
% Default to the notebook output style

    


% Inherit from the specified cell style.




    
\documentclass[11pt]{article}

    
    
    \usepackage[T1]{fontenc}
    % Nicer default font (+ math font) than Computer Modern for most use cases
    \usepackage{mathpazo}

    % Basic figure setup, for now with no caption control since it's done
    % automatically by Pandoc (which extracts ![](path) syntax from Markdown).
    \usepackage{graphicx}
    % We will generate all images so they have a width \maxwidth. This means
    % that they will get their normal width if they fit onto the page, but
    % are scaled down if they would overflow the margins.
    \makeatletter
    \def\maxwidth{\ifdim\Gin@nat@width>\linewidth\linewidth
    \else\Gin@nat@width\fi}
    \makeatother
    \let\Oldincludegraphics\includegraphics
    % Set max figure width to be 80% of text width, for now hardcoded.
    \renewcommand{\includegraphics}[1]{\Oldincludegraphics[width=.8\maxwidth]{#1}}
    % Ensure that by default, figures have no caption (until we provide a
    % proper Figure object with a Caption API and a way to capture that
    % in the conversion process - todo).
    \usepackage{caption}
    \DeclareCaptionLabelFormat{nolabel}{}
    \captionsetup{labelformat=nolabel}

    \usepackage{adjustbox} % Used to constrain images to a maximum size 
    \usepackage{xcolor} % Allow colors to be defined
    \usepackage{enumerate} % Needed for markdown enumerations to work
    \usepackage{geometry} % Used to adjust the document margins
    \usepackage{amsmath} % Equations
    \usepackage{amssymb} % Equations
    \usepackage{textcomp} % defines textquotesingle
    % Hack from http://tex.stackexchange.com/a/47451/13684:
    \AtBeginDocument{%
        \def\PYZsq{\textquotesingle}% Upright quotes in Pygmentized code
    }
    \usepackage{upquote} % Upright quotes for verbatim code
    \usepackage{eurosym} % defines \euro
    \usepackage[mathletters]{ucs} % Extended unicode (utf-8) support
    \usepackage[utf8x]{inputenc} % Allow utf-8 characters in the tex document
    \usepackage{fancyvrb} % verbatim replacement that allows latex
    \usepackage{grffile} % extends the file name processing of package graphics 
                         % to support a larger range 
    % The hyperref package gives us a pdf with properly built
    % internal navigation ('pdf bookmarks' for the table of contents,
    % internal cross-reference links, web links for URLs, etc.)
    \usepackage{hyperref}
    \usepackage{longtable} % longtable support required by pandoc >1.10
    \usepackage{booktabs}  % table support for pandoc > 1.12.2
    \usepackage[inline]{enumitem} % IRkernel/repr support (it uses the enumerate* environment)
    \usepackage[normalem]{ulem} % ulem is needed to support strikethroughs (\sout)
                                % normalem makes italics be italics, not underlines
    

    
    
    % Colors for the hyperref package
    \definecolor{urlcolor}{rgb}{0,.145,.698}
    \definecolor{linkcolor}{rgb}{.71,0.21,0.01}
    \definecolor{citecolor}{rgb}{.12,.54,.11}

    % ANSI colors
    \definecolor{ansi-black}{HTML}{3E424D}
    \definecolor{ansi-black-intense}{HTML}{282C36}
    \definecolor{ansi-red}{HTML}{E75C58}
    \definecolor{ansi-red-intense}{HTML}{B22B31}
    \definecolor{ansi-green}{HTML}{00A250}
    \definecolor{ansi-green-intense}{HTML}{007427}
    \definecolor{ansi-yellow}{HTML}{DDB62B}
    \definecolor{ansi-yellow-intense}{HTML}{B27D12}
    \definecolor{ansi-blue}{HTML}{208FFB}
    \definecolor{ansi-blue-intense}{HTML}{0065CA}
    \definecolor{ansi-magenta}{HTML}{D160C4}
    \definecolor{ansi-magenta-intense}{HTML}{A03196}
    \definecolor{ansi-cyan}{HTML}{60C6C8}
    \definecolor{ansi-cyan-intense}{HTML}{258F8F}
    \definecolor{ansi-white}{HTML}{C5C1B4}
    \definecolor{ansi-white-intense}{HTML}{A1A6B2}

    % commands and environments needed by pandoc snippets
    % extracted from the output of `pandoc -s`
    \providecommand{\tightlist}{%
      \setlength{\itemsep}{0pt}\setlength{\parskip}{0pt}}
    \DefineVerbatimEnvironment{Highlighting}{Verbatim}{commandchars=\\\{\}}
    % Add ',fontsize=\small' for more characters per line
    \newenvironment{Shaded}{}{}
    \newcommand{\KeywordTok}[1]{\textcolor[rgb]{0.00,0.44,0.13}{\textbf{{#1}}}}
    \newcommand{\DataTypeTok}[1]{\textcolor[rgb]{0.56,0.13,0.00}{{#1}}}
    \newcommand{\DecValTok}[1]{\textcolor[rgb]{0.25,0.63,0.44}{{#1}}}
    \newcommand{\BaseNTok}[1]{\textcolor[rgb]{0.25,0.63,0.44}{{#1}}}
    \newcommand{\FloatTok}[1]{\textcolor[rgb]{0.25,0.63,0.44}{{#1}}}
    \newcommand{\CharTok}[1]{\textcolor[rgb]{0.25,0.44,0.63}{{#1}}}
    \newcommand{\StringTok}[1]{\textcolor[rgb]{0.25,0.44,0.63}{{#1}}}
    \newcommand{\CommentTok}[1]{\textcolor[rgb]{0.38,0.63,0.69}{\textit{{#1}}}}
    \newcommand{\OtherTok}[1]{\textcolor[rgb]{0.00,0.44,0.13}{{#1}}}
    \newcommand{\AlertTok}[1]{\textcolor[rgb]{1.00,0.00,0.00}{\textbf{{#1}}}}
    \newcommand{\FunctionTok}[1]{\textcolor[rgb]{0.02,0.16,0.49}{{#1}}}
    \newcommand{\RegionMarkerTok}[1]{{#1}}
    \newcommand{\ErrorTok}[1]{\textcolor[rgb]{1.00,0.00,0.00}{\textbf{{#1}}}}
    \newcommand{\NormalTok}[1]{{#1}}
    
    % Additional commands for more recent versions of Pandoc
    \newcommand{\ConstantTok}[1]{\textcolor[rgb]{0.53,0.00,0.00}{{#1}}}
    \newcommand{\SpecialCharTok}[1]{\textcolor[rgb]{0.25,0.44,0.63}{{#1}}}
    \newcommand{\VerbatimStringTok}[1]{\textcolor[rgb]{0.25,0.44,0.63}{{#1}}}
    \newcommand{\SpecialStringTok}[1]{\textcolor[rgb]{0.73,0.40,0.53}{{#1}}}
    \newcommand{\ImportTok}[1]{{#1}}
    \newcommand{\DocumentationTok}[1]{\textcolor[rgb]{0.73,0.13,0.13}{\textit{{#1}}}}
    \newcommand{\AnnotationTok}[1]{\textcolor[rgb]{0.38,0.63,0.69}{\textbf{\textit{{#1}}}}}
    \newcommand{\CommentVarTok}[1]{\textcolor[rgb]{0.38,0.63,0.69}{\textbf{\textit{{#1}}}}}
    \newcommand{\VariableTok}[1]{\textcolor[rgb]{0.10,0.09,0.49}{{#1}}}
    \newcommand{\ControlFlowTok}[1]{\textcolor[rgb]{0.00,0.44,0.13}{\textbf{{#1}}}}
    \newcommand{\OperatorTok}[1]{\textcolor[rgb]{0.40,0.40,0.40}{{#1}}}
    \newcommand{\BuiltInTok}[1]{{#1}}
    \newcommand{\ExtensionTok}[1]{{#1}}
    \newcommand{\PreprocessorTok}[1]{\textcolor[rgb]{0.74,0.48,0.00}{{#1}}}
    \newcommand{\AttributeTok}[1]{\textcolor[rgb]{0.49,0.56,0.16}{{#1}}}
    \newcommand{\InformationTok}[1]{\textcolor[rgb]{0.38,0.63,0.69}{\textbf{\textit{{#1}}}}}
    \newcommand{\WarningTok}[1]{\textcolor[rgb]{0.38,0.63,0.69}{\textbf{\textit{{#1}}}}}
    
    
    % Define a nice break command that doesn't care if a line doesn't already
    % exist.
    \def\br{\hspace*{\fill} \\* }
    % Math Jax compatability definitions
    \def\gt{>}
    \def\lt{<}
    % Document parameters
    \title{Kmeans GM Sprectral Clustering }
    
    
    

    % Pygments definitions
    
\makeatletter
\def\PY@reset{\let\PY@it=\relax \let\PY@bf=\relax%
    \let\PY@ul=\relax \let\PY@tc=\relax%
    \let\PY@bc=\relax \let\PY@ff=\relax}
\def\PY@tok#1{\csname PY@tok@#1\endcsname}
\def\PY@toks#1+{\ifx\relax#1\empty\else%
    \PY@tok{#1}\expandafter\PY@toks\fi}
\def\PY@do#1{\PY@bc{\PY@tc{\PY@ul{%
    \PY@it{\PY@bf{\PY@ff{#1}}}}}}}
\def\PY#1#2{\PY@reset\PY@toks#1+\relax+\PY@do{#2}}

\expandafter\def\csname PY@tok@gd\endcsname{\def\PY@tc##1{\textcolor[rgb]{0.63,0.00,0.00}{##1}}}
\expandafter\def\csname PY@tok@gu\endcsname{\let\PY@bf=\textbf\def\PY@tc##1{\textcolor[rgb]{0.50,0.00,0.50}{##1}}}
\expandafter\def\csname PY@tok@gt\endcsname{\def\PY@tc##1{\textcolor[rgb]{0.00,0.27,0.87}{##1}}}
\expandafter\def\csname PY@tok@gs\endcsname{\let\PY@bf=\textbf}
\expandafter\def\csname PY@tok@gr\endcsname{\def\PY@tc##1{\textcolor[rgb]{1.00,0.00,0.00}{##1}}}
\expandafter\def\csname PY@tok@cm\endcsname{\let\PY@it=\textit\def\PY@tc##1{\textcolor[rgb]{0.25,0.50,0.50}{##1}}}
\expandafter\def\csname PY@tok@vg\endcsname{\def\PY@tc##1{\textcolor[rgb]{0.10,0.09,0.49}{##1}}}
\expandafter\def\csname PY@tok@vi\endcsname{\def\PY@tc##1{\textcolor[rgb]{0.10,0.09,0.49}{##1}}}
\expandafter\def\csname PY@tok@vm\endcsname{\def\PY@tc##1{\textcolor[rgb]{0.10,0.09,0.49}{##1}}}
\expandafter\def\csname PY@tok@mh\endcsname{\def\PY@tc##1{\textcolor[rgb]{0.40,0.40,0.40}{##1}}}
\expandafter\def\csname PY@tok@cs\endcsname{\let\PY@it=\textit\def\PY@tc##1{\textcolor[rgb]{0.25,0.50,0.50}{##1}}}
\expandafter\def\csname PY@tok@ge\endcsname{\let\PY@it=\textit}
\expandafter\def\csname PY@tok@vc\endcsname{\def\PY@tc##1{\textcolor[rgb]{0.10,0.09,0.49}{##1}}}
\expandafter\def\csname PY@tok@il\endcsname{\def\PY@tc##1{\textcolor[rgb]{0.40,0.40,0.40}{##1}}}
\expandafter\def\csname PY@tok@go\endcsname{\def\PY@tc##1{\textcolor[rgb]{0.53,0.53,0.53}{##1}}}
\expandafter\def\csname PY@tok@cp\endcsname{\def\PY@tc##1{\textcolor[rgb]{0.74,0.48,0.00}{##1}}}
\expandafter\def\csname PY@tok@gi\endcsname{\def\PY@tc##1{\textcolor[rgb]{0.00,0.63,0.00}{##1}}}
\expandafter\def\csname PY@tok@gh\endcsname{\let\PY@bf=\textbf\def\PY@tc##1{\textcolor[rgb]{0.00,0.00,0.50}{##1}}}
\expandafter\def\csname PY@tok@ni\endcsname{\let\PY@bf=\textbf\def\PY@tc##1{\textcolor[rgb]{0.60,0.60,0.60}{##1}}}
\expandafter\def\csname PY@tok@nl\endcsname{\def\PY@tc##1{\textcolor[rgb]{0.63,0.63,0.00}{##1}}}
\expandafter\def\csname PY@tok@nn\endcsname{\let\PY@bf=\textbf\def\PY@tc##1{\textcolor[rgb]{0.00,0.00,1.00}{##1}}}
\expandafter\def\csname PY@tok@no\endcsname{\def\PY@tc##1{\textcolor[rgb]{0.53,0.00,0.00}{##1}}}
\expandafter\def\csname PY@tok@na\endcsname{\def\PY@tc##1{\textcolor[rgb]{0.49,0.56,0.16}{##1}}}
\expandafter\def\csname PY@tok@nb\endcsname{\def\PY@tc##1{\textcolor[rgb]{0.00,0.50,0.00}{##1}}}
\expandafter\def\csname PY@tok@nc\endcsname{\let\PY@bf=\textbf\def\PY@tc##1{\textcolor[rgb]{0.00,0.00,1.00}{##1}}}
\expandafter\def\csname PY@tok@nd\endcsname{\def\PY@tc##1{\textcolor[rgb]{0.67,0.13,1.00}{##1}}}
\expandafter\def\csname PY@tok@ne\endcsname{\let\PY@bf=\textbf\def\PY@tc##1{\textcolor[rgb]{0.82,0.25,0.23}{##1}}}
\expandafter\def\csname PY@tok@nf\endcsname{\def\PY@tc##1{\textcolor[rgb]{0.00,0.00,1.00}{##1}}}
\expandafter\def\csname PY@tok@si\endcsname{\let\PY@bf=\textbf\def\PY@tc##1{\textcolor[rgb]{0.73,0.40,0.53}{##1}}}
\expandafter\def\csname PY@tok@s2\endcsname{\def\PY@tc##1{\textcolor[rgb]{0.73,0.13,0.13}{##1}}}
\expandafter\def\csname PY@tok@nt\endcsname{\let\PY@bf=\textbf\def\PY@tc##1{\textcolor[rgb]{0.00,0.50,0.00}{##1}}}
\expandafter\def\csname PY@tok@nv\endcsname{\def\PY@tc##1{\textcolor[rgb]{0.10,0.09,0.49}{##1}}}
\expandafter\def\csname PY@tok@s1\endcsname{\def\PY@tc##1{\textcolor[rgb]{0.73,0.13,0.13}{##1}}}
\expandafter\def\csname PY@tok@dl\endcsname{\def\PY@tc##1{\textcolor[rgb]{0.73,0.13,0.13}{##1}}}
\expandafter\def\csname PY@tok@ch\endcsname{\let\PY@it=\textit\def\PY@tc##1{\textcolor[rgb]{0.25,0.50,0.50}{##1}}}
\expandafter\def\csname PY@tok@m\endcsname{\def\PY@tc##1{\textcolor[rgb]{0.40,0.40,0.40}{##1}}}
\expandafter\def\csname PY@tok@gp\endcsname{\let\PY@bf=\textbf\def\PY@tc##1{\textcolor[rgb]{0.00,0.00,0.50}{##1}}}
\expandafter\def\csname PY@tok@sh\endcsname{\def\PY@tc##1{\textcolor[rgb]{0.73,0.13,0.13}{##1}}}
\expandafter\def\csname PY@tok@ow\endcsname{\let\PY@bf=\textbf\def\PY@tc##1{\textcolor[rgb]{0.67,0.13,1.00}{##1}}}
\expandafter\def\csname PY@tok@sx\endcsname{\def\PY@tc##1{\textcolor[rgb]{0.00,0.50,0.00}{##1}}}
\expandafter\def\csname PY@tok@bp\endcsname{\def\PY@tc##1{\textcolor[rgb]{0.00,0.50,0.00}{##1}}}
\expandafter\def\csname PY@tok@c1\endcsname{\let\PY@it=\textit\def\PY@tc##1{\textcolor[rgb]{0.25,0.50,0.50}{##1}}}
\expandafter\def\csname PY@tok@fm\endcsname{\def\PY@tc##1{\textcolor[rgb]{0.00,0.00,1.00}{##1}}}
\expandafter\def\csname PY@tok@o\endcsname{\def\PY@tc##1{\textcolor[rgb]{0.40,0.40,0.40}{##1}}}
\expandafter\def\csname PY@tok@kc\endcsname{\let\PY@bf=\textbf\def\PY@tc##1{\textcolor[rgb]{0.00,0.50,0.00}{##1}}}
\expandafter\def\csname PY@tok@c\endcsname{\let\PY@it=\textit\def\PY@tc##1{\textcolor[rgb]{0.25,0.50,0.50}{##1}}}
\expandafter\def\csname PY@tok@mf\endcsname{\def\PY@tc##1{\textcolor[rgb]{0.40,0.40,0.40}{##1}}}
\expandafter\def\csname PY@tok@err\endcsname{\def\PY@bc##1{\setlength{\fboxsep}{0pt}\fcolorbox[rgb]{1.00,0.00,0.00}{1,1,1}{\strut ##1}}}
\expandafter\def\csname PY@tok@mb\endcsname{\def\PY@tc##1{\textcolor[rgb]{0.40,0.40,0.40}{##1}}}
\expandafter\def\csname PY@tok@ss\endcsname{\def\PY@tc##1{\textcolor[rgb]{0.10,0.09,0.49}{##1}}}
\expandafter\def\csname PY@tok@sr\endcsname{\def\PY@tc##1{\textcolor[rgb]{0.73,0.40,0.53}{##1}}}
\expandafter\def\csname PY@tok@mo\endcsname{\def\PY@tc##1{\textcolor[rgb]{0.40,0.40,0.40}{##1}}}
\expandafter\def\csname PY@tok@kd\endcsname{\let\PY@bf=\textbf\def\PY@tc##1{\textcolor[rgb]{0.00,0.50,0.00}{##1}}}
\expandafter\def\csname PY@tok@mi\endcsname{\def\PY@tc##1{\textcolor[rgb]{0.40,0.40,0.40}{##1}}}
\expandafter\def\csname PY@tok@kn\endcsname{\let\PY@bf=\textbf\def\PY@tc##1{\textcolor[rgb]{0.00,0.50,0.00}{##1}}}
\expandafter\def\csname PY@tok@cpf\endcsname{\let\PY@it=\textit\def\PY@tc##1{\textcolor[rgb]{0.25,0.50,0.50}{##1}}}
\expandafter\def\csname PY@tok@kr\endcsname{\let\PY@bf=\textbf\def\PY@tc##1{\textcolor[rgb]{0.00,0.50,0.00}{##1}}}
\expandafter\def\csname PY@tok@s\endcsname{\def\PY@tc##1{\textcolor[rgb]{0.73,0.13,0.13}{##1}}}
\expandafter\def\csname PY@tok@kp\endcsname{\def\PY@tc##1{\textcolor[rgb]{0.00,0.50,0.00}{##1}}}
\expandafter\def\csname PY@tok@w\endcsname{\def\PY@tc##1{\textcolor[rgb]{0.73,0.73,0.73}{##1}}}
\expandafter\def\csname PY@tok@kt\endcsname{\def\PY@tc##1{\textcolor[rgb]{0.69,0.00,0.25}{##1}}}
\expandafter\def\csname PY@tok@sc\endcsname{\def\PY@tc##1{\textcolor[rgb]{0.73,0.13,0.13}{##1}}}
\expandafter\def\csname PY@tok@sb\endcsname{\def\PY@tc##1{\textcolor[rgb]{0.73,0.13,0.13}{##1}}}
\expandafter\def\csname PY@tok@sa\endcsname{\def\PY@tc##1{\textcolor[rgb]{0.73,0.13,0.13}{##1}}}
\expandafter\def\csname PY@tok@k\endcsname{\let\PY@bf=\textbf\def\PY@tc##1{\textcolor[rgb]{0.00,0.50,0.00}{##1}}}
\expandafter\def\csname PY@tok@se\endcsname{\let\PY@bf=\textbf\def\PY@tc##1{\textcolor[rgb]{0.73,0.40,0.13}{##1}}}
\expandafter\def\csname PY@tok@sd\endcsname{\let\PY@it=\textit\def\PY@tc##1{\textcolor[rgb]{0.73,0.13,0.13}{##1}}}

\def\PYZbs{\char`\\}
\def\PYZus{\char`\_}
\def\PYZob{\char`\{}
\def\PYZcb{\char`\}}
\def\PYZca{\char`\^}
\def\PYZam{\char`\&}
\def\PYZlt{\char`\<}
\def\PYZgt{\char`\>}
\def\PYZsh{\char`\#}
\def\PYZpc{\char`\%}
\def\PYZdl{\char`\$}
\def\PYZhy{\char`\-}
\def\PYZsq{\char`\'}
\def\PYZdq{\char`\"}
\def\PYZti{\char`\~}
% for compatibility with earlier versions
\def\PYZat{@}
\def\PYZlb{[}
\def\PYZrb{]}
\makeatother


    % Exact colors from NB
    \definecolor{incolor}{rgb}{0.0, 0.0, 0.5}
    \definecolor{outcolor}{rgb}{0.545, 0.0, 0.0}



    
    % Prevent overflowing lines due to hard-to-break entities
    \sloppy 
    % Setup hyperref package
    \hypersetup{
      breaklinks=true,  % so long urls are correctly broken across lines
      colorlinks=true,
      urlcolor=urlcolor,
      linkcolor=linkcolor,
      citecolor=citecolor,
      }
    % Slightly bigger margins than the latex defaults
    
    \geometry{verbose,tmargin=1in,bmargin=1in,lmargin=1in,rmargin=1in}
    
    

    \begin{document}
    
    
    \maketitle
    
    

    
    \begin{Verbatim}[commandchars=\\\{\}]
{\color{incolor}In [{\color{incolor}1}]:} \PY{k+kn}{from} \PY{n+nn}{sklearn.datasets} \PY{k+kn}{import} \PY{n}{load\PYZus{}iris}
        \PY{k+kn}{from} \PY{n+nn}{sklearn.cluster} \PY{k+kn}{import} \PY{n}{KMeans}\PY{p}{,} \PY{n}{SpectralClustering}
        \PY{k+kn}{from} \PY{n+nn}{sklearn.mixture} \PY{k+kn}{import} \PY{n}{GaussianMixture}
        \PY{k+kn}{import} \PY{n+nn}{matplotlib.pyplot} \PY{k+kn}{as} \PY{n+nn}{plt}
        \PY{k+kn}{from} \PY{n+nn}{sklearn.metrics} \PY{k+kn}{import} \PY{n}{confusion\PYZus{}matrix}
        
        \PY{c+c1}{\PYZsh{}Lo que viene en el dataset}
        \PY{c+c1}{\PYZsh{} 150 instancias \PYZhy{} 3 Clases de Iris \PYZhy{} 50 Iris Setosa, 50 Iris Versicolor, 50 Iris Virginica}
        \PY{c+c1}{\PYZsh{} 4 Atributos en cm \PYZhy{} Largo del Sepalo, Ancho del Sepalo, Largo del Petalo, Ancho del Petalo }
        \PY{c+c1}{\PYZsh{} No hay valores nulos, por lo que no tenemos que preocuparnos por eso.}
        
        \PY{n}{data} \PY{o}{=} \PY{n}{load\PYZus{}iris}\PY{p}{(}\PY{p}{)}
        \PY{c+c1}{\PYZsh{}print(data)}
\end{Verbatim}


    \section{Kmeans description}\label{kmeans-description}

Una breve descripción de ventajas y desventajas de hacer clustering con
esté algoritmo, y tener en cuenta para su óptimo funcionamiento.

\subsubsection{Suposición}\label{suposiciuxf3n}

\begin{enumerate}
\def\labelenumi{\arabic{enumi})}
\item
  Asumir un tamaño de cluster equilibrado dentro del conjunto de datos.
\item
  Se supone que la distribución conjunta de las características dentro
  de cada grupo es esférica: esto significa que las características
  dentro de un grupo tienen la misma varianza, y también las
  características son independientes entre sí.
\item
  los grupos tienen una densidad similar.
\end{enumerate}

\subsubsection{Contras}\label{contras}

\begin{enumerate}
\def\labelenumi{\arabic{enumi})}
\item
  efecto uniforme: a menudo producen grupos con un tamaño relativamente
  uniforme, incluso si los datos de entrada tienen un tamaño de grupo
  diferente.
\item
  Suposición esférica difícil de satisfacer: la correlación entre las
  características la rompe, pondría pesos adicionales en las
  características correlacionadas (debería tomar medidas dependiendo de
  los problemas); no puede encontrar clústeres o conglomerados no
  convexos con formas inusuales.
\item
  densidades diferentes: pueden funcionar mal con agrupaciones con
  diferentes densidades pero forma esférica.
\item
  Valor K desconocido: ¿cómo resolver K?
\end{enumerate}

\begin{itemize}
\tightlist
\item
  para un rango pequeño de valor K, digamos 2-10, para cada valor K
  ejecutado muchas veces (20-100 veces), tome el resultado de
  agrupamiento con el valor J más bajo entre todos los valores K;
\item
  usando el método de codo para decidir el valor de K;
\item
  GAPs;
\item
  decidir los flujos descendentes K: decidir por los propósitos /
  objetivos de los proyectos
\end{itemize}

\begin{enumerate}
\def\labelenumi{\arabic{enumi})}
\setcounter{enumi}{4}
\item
  sensible a valores atípicos;
\item
  sensible a los puntos iniciales y óptimo local, y no hay una solución
  única para un cierto valor K: así ejecute K media para un valor K
  muchas veces (20-100 veces), luego elija los resultados con la J más
  baja
\end{enumerate}

\subsubsection{Pros}\label{pros}

\begin{enumerate}
\def\labelenumi{\arabic{enumi})}
\item
  prácticamente funciona bien incluso algunas suposiciones se rompen;
\item
  simple, fácil de implementar;
\item
  fácil de interpretar los resultados de la agrupación;
\item
  rápido y eficiente en términos de costo computacional, típicamente O
  (K * n * d);
\end{enumerate}

    \section{Mixture Guassian}\label{mixture-guassian}

En lugar de asignar datos a un clúster, si no estamos seguros de los
puntos de datos a los que pertenecen o a qué grupo, utilizamos este
método. Utiliza la probabilidad de una muestra para determinar la
viabilidad de pertenecer a un clúster.

\subsubsection{Ventajas}\label{ventajas}

\begin{enumerate}
\def\labelenumi{\arabic{enumi}.}
\item
  No supone que los clústeres sean de ninguna geometría. Funciona bien
  con distribuciones geométricas no lineales también.
\item
  No sesga los tamaños de conglomerados para que tengan estructuras
  específicas como lo hace K-Means (Circular).
\end{enumerate}

\subsubsection{Desventajas}\label{desventajas}

\begin{enumerate}
\def\labelenumi{\arabic{enumi}.}
\item
  Utiliza todos los componentes a los que tiene acceso, por lo que la
  inicialización de clusters será difícil cuando la dimensionalidad de
  los datos sea alta.
\item
  Difícil de interpretar.
\end{enumerate}

\section{GMM description vs K means}\label{gmm-description-vs-k-means}

GMM es mucho más flexible en términos de agrupamiento en covarianza.
k-means es en realidad un caso especial de GMM en el que la covarianza
de cada agrupación a lo largo de todas las dimensiones se acerca a 0.
Esto implica que un punto se asignará solo al clúster más cercano. Con
GMM, cada grupo puede tener una estructura de covarianza no restringida.
Piense en la distribución girada y / o alargada de puntos en un grupo,
en lugar de esférica como en kmeans. Como resultado, la asignación de
clústeres es mucho más flexible en GMM que en k-means.

\paragraph{El modelo de GMM acomoda la membresía
mixta}\label{el-modelo-de-gmm-acomoda-la-membresuxeda-mixta}

Otra implicación de su estructura de covarianza es que GMM permite la
membresía mixta de puntos a clusters. En kmeans, un punto pertenece a un
solo cluster, mientras que en GMM un punto pertenece a cada cluster en
un grado diferente. El grado se basa en la probabilidad de que el punto
se genere a partir de la distribución normal (multivariante) de cada
agrupación, con el centro del clúster como la media de la distribución y
la covarianza del clúster como su covarianza. Dependiendo de la tarea,
la membresía mixta puede ser más apropiada (por ejemplo, los artículos
de noticias pueden pertenecer a múltiples grupos de temas) o no (por
ejemplo, los organismos pueden pertenecer a una sola especie).

    \section{Spectral clustering}\label{spectral-clustering}

La agrupación espectral funciona primero transformando los datos del
espacio cartesiano en espacio de similitud y luego agrupando en un
espacio de similitud. Los datos originales se proyectan en el nuevo
espacio de coordenadas que codifica información sobre cómo son los
puntos de datos cercanos. La transformación de similitud reduce la
dimensionalidad del espacio y, en términos generales, pre-agrupa los
datos en dimensiones ortogonales. Este pre-clustering no es lineal y
permite geometrías no convexas conectadas arbitrariamente, que es la
principal ventaja de la agrupación espectral.

El mapeo del espacio cartesiano al espacio de similitud es facilitado
por la creación y diagonalización de la matriz de similitud. En el caso
en que tenga k agrupamientos espacialmente separados y bien definidos,
independientemente de la forma geométrica del grupo, la matriz de
similitud resultante es diagonal de bloques. Cada bloque corresponderá a
un grupo diferente.

Cuando apila los k-vectores propios más bajos de esta matriz como
columnas en una nueva matriz y los normaliza, las filas de la matriz son
las nuevas coordenadas para cada punto de datos en el nuevo espacio.
Ignorando la degeneración, si inspecciona estas nuevas coordenadas verá
que los datos se encuentran a lo largo de cada uno de los ejes de su
nuevo espacio. Estas son las coordenadas que se utilizan para agrupar y
asignar las etiquetas de los grupos de datos originales.

La parte K-means se ejecuta porque los autovectores pueden estar
degenerados, y los clústeres no tienen que estar tan claramente
separados. Los vectores propios abarcan el espacio lineal definido por
los clusters. Pero, los conglomerados pueden sentarse en cualquier
coordenadas en el espacio, siempre y cuando se roten 90 grados entre sí
en relación con el origen.

\section{Spectral clustering vs
Kmeans}\label{spectral-clustering-vs-kmeans}

Eche un vistazo a estos seis conjuntos de datos (de juguete), donde se
aplica la agrupación espectral para su agrupación en clústeres:

\begin{figure}
\centering
\includegraphics{attachment:image.png}
\caption{image.png}
\end{figure}

Los K-means no podrán agruparlos eficazmente, incluso cuando el
algoritmo conozca la verdadera cantidad de clústeres K.

Esto se debe a que K-means, como algoritmo de agrupamiento de datos, es
ideal para descubrir cúmulos globulares como los que se muestran a
continuación, donde todos los miembros de cada grupo se encuentran muy
cerca el uno del otro (en el sentido euclidiano).

En contraste con el agrupamiento de datos, tenemos técnicas de
agrupamiento de gráficos como la agrupación espectral, donde no agrupa
los puntos de datos directamente en su espacio de datos nativo sino que
forma una matriz de similitud donde la entrada (i, j) es algo distancia
de similitud que define entre los puntos de datos i-th y j-th en su
conjunto de datos.

Entonces, en cierto sentido, la agrupación espectral es más general (y
poderosa) porque siempre que K-means sea apropiado para su uso, también
lo es la agrupación espectral (basta con usar una simple distancia
euclidiana como medida de similitud). Lo contrario no es cierto.

\subsubsection{Consideraciones}\label{consideraciones}

También hay consideraciones prácticas que debe tener en cuenta al elegir
uno de estos métodos sobre el otro. Con K-means se factoriza la matriz
de datos de entrada, mientras que con la agrupación espectral se
factoriza la matriz de Laplacian (una matriz derivada de la matriz de
similitud) ¿Por qué eso importa?

Supongamos que tiene puntos de datos P, cada uno con N dimensiones /
características. Luego, usando K-means, tratará con una matriz N por P,
mientras que la matriz de entrada a la agrupación espectral tendrá el
tamaño P por P. Ahora debería ver las implicaciones prácticas: la
agrupación espectral es indiferente a la cantidad de características que
usa (El núcleo gaussiano, que puede considerarse como una transformación
de características infinitamente dimensional, es particularmente popular
cuando se usan agrupaciones espectrales). Sin embargo, se enfrentarán
dificultades para aplicar la agrupación espectral (al menos la versión
estándar) a conjuntos de datos muy grandes (P grande).

    \begin{Verbatim}[commandchars=\\\{\}]
{\color{incolor}In [{\color{incolor}2}]:} \PY{c+c1}{\PYZsh{}funciones Kmeans, GaussianMixture SpectralClustering}
        \PY{k}{def} \PY{n+nf}{clustering}\PY{p}{(}\PY{n}{X}\PY{p}{,} \PY{n}{cluster}\PY{p}{)}\PY{p}{:}
            \PY{c+c1}{\PYZsh{}\PYZsh{}\PYZsh{}\PYZsh{}\PYZsh{}\PYZsh{}\PYZsh{}\PYZsh{}\PYZsh{}\PYZsh{}\PYZsh{}\PYZsh{}\PYZsh{}\PYZsh{}\PYZsh{}\PYZsh{}\PYZsh{}\PYZsh{}\PYZsh{}\PYZsh{}\PYZsh{}\PYZsh{}\PYZsh{}\PYZsh{}\PYZsh{}\PYZsh{}\PYZsh{}\PYZsh{}\PYZsh{}\PYZsh{}\PYZsh{}\PYZsh{}\PYZsh{}\PYZsh{}\PYZsh{} Kmeans}
        
            \PY{n}{kmeans} \PY{o}{=} \PY{n}{KMeans}\PY{p}{(}\PY{n}{n\PYZus{}clusters}\PY{o}{=}\PY{n}{cluster}\PY{p}{)}\PY{o}{.}\PY{n}{fit}\PY{p}{(}\PY{n}{X}\PY{p}{)}
            \PY{n}{result\PYZus{}kmeans} \PY{o}{=} \PY{n}{kmeans}\PY{o}{.}\PY{n}{labels\PYZus{}}
        
            \PY{k}{print}\PY{p}{(}\PY{l+s+s1}{\PYZsq{}}\PY{l+s+se}{\PYZbs{}n}\PY{l+s+s1}{Kmeans}\PY{l+s+s1}{\PYZsq{}}\PY{p}{)}
            \PY{k}{print}\PY{p}{(}\PY{n}{result\PYZus{}kmeans}\PY{p}{)}
        
            \PY{n}{cMat} \PY{o}{=} \PY{n}{confusion\PYZus{}matrix}\PY{p}{(}\PY{n}{t}\PY{p}{,} \PY{n}{result\PYZus{}kmeans}\PY{p}{)}
            \PY{k}{print}\PY{p}{(}\PY{n}{cMat}\PY{p}{)}
        
            \PY{n}{accuracy} \PY{o}{=} \PY{l+m+mi}{0}
            \PY{k}{for} \PY{n}{l} \PY{o+ow}{in} \PY{n}{cMat}\PY{p}{:}
                \PY{n}{m} \PY{o}{=} \PY{n+nb}{max}\PY{p}{(}\PY{n}{l}\PY{p}{)}
                \PY{n}{accuracy} \PY{o}{+}\PY{o}{=} \PY{n}{m}
            \PY{k}{print}\PY{p}{(}\PY{l+s+s1}{\PYZsq{}}\PY{l+s+s1}{Accuracy \PYZob{}\PYZcb{}}\PY{l+s+s1}{\PYZpc{}}\PY{l+s+s1}{\PYZsq{}}\PY{o}{.}\PY{n}{format}\PY{p}{(}\PY{n}{accuracy}\PY{o}{/}\PY{n}{cant}\PY{o}{*}\PY{l+m+mi}{100}\PY{p}{)}\PY{p}{)}
        
        
            \PY{c+c1}{\PYZsh{}\PYZsh{}\PYZsh{}\PYZsh{}\PYZsh{}\PYZsh{}\PYZsh{}\PYZsh{}\PYZsh{}\PYZsh{}\PYZsh{}\PYZsh{}\PYZsh{}\PYZsh{}\PYZsh{}\PYZsh{}\PYZsh{}\PYZsh{}\PYZsh{}\PYZsh{}\PYZsh{}\PYZsh{}\PYZsh{}\PYZsh{}\PYZsh{}\PYZsh{}\PYZsh{}\PYZsh{}\PYZsh{}\PYZsh{}\PYZsh{}\PYZsh{}\PYZsh{}\PYZsh{}\PYZsh{} GaussianMixture}
            \PY{n}{gnm} \PY{o}{=} \PY{n}{GaussianMixture}\PY{p}{(}\PY{n}{n\PYZus{}components}\PY{o}{=}\PY{n}{cluster}\PY{p}{)}\PY{o}{.}\PY{n}{fit}\PY{p}{(}\PY{n}{X}\PY{p}{)}
            \PY{n}{result\PYZus{}gnm} \PY{o}{=} \PY{n}{gnm}\PY{o}{.}\PY{n}{predict}\PY{p}{(}\PY{n}{X}\PY{p}{)}
        
            \PY{k}{print}\PY{p}{(}\PY{l+s+s1}{\PYZsq{}}\PY{l+s+se}{\PYZbs{}n}\PY{l+s+s1}{Gaussian Mixture}\PY{l+s+s1}{\PYZsq{}}\PY{p}{)}
            \PY{k}{print}\PY{p}{(}\PY{n}{result\PYZus{}gnm}\PY{p}{)}
        
            \PY{n}{cMat} \PY{o}{=} \PY{n}{confusion\PYZus{}matrix}\PY{p}{(}\PY{n}{t}\PY{p}{,} \PY{n}{result\PYZus{}gnm}\PY{p}{)}
            \PY{k}{print}\PY{p}{(}\PY{n}{cMat}\PY{p}{)}
        
            \PY{n}{accuracy} \PY{o}{=} \PY{l+m+mi}{0}
            \PY{k}{for} \PY{n}{l} \PY{o+ow}{in} \PY{n}{cMat}\PY{p}{:}
                \PY{n}{m} \PY{o}{=} \PY{n+nb}{max}\PY{p}{(}\PY{n}{l}\PY{p}{)}
                \PY{n}{accuracy} \PY{o}{+}\PY{o}{=} \PY{n}{m}
            \PY{k}{print}\PY{p}{(}\PY{l+s+s1}{\PYZsq{}}\PY{l+s+s1}{Accuracy \PYZob{}\PYZcb{}}\PY{l+s+s1}{\PYZpc{}}\PY{l+s+s1}{\PYZsq{}}\PY{o}{.}\PY{n}{format}\PY{p}{(}\PY{n}{accuracy}\PY{o}{/}\PY{n}{cant}\PY{o}{*}\PY{l+m+mi}{100}\PY{p}{)}\PY{p}{)}
        
        
            \PY{c+c1}{\PYZsh{}\PYZsh{}\PYZsh{}\PYZsh{}\PYZsh{}\PYZsh{}\PYZsh{}\PYZsh{}\PYZsh{}\PYZsh{}\PYZsh{}\PYZsh{}\PYZsh{}\PYZsh{}\PYZsh{}\PYZsh{}\PYZsh{}\PYZsh{}\PYZsh{}\PYZsh{}\PYZsh{}\PYZsh{}\PYZsh{}\PYZsh{}\PYZsh{}\PYZsh{}\PYZsh{}\PYZsh{}\PYZsh{}\PYZsh{}\PYZsh{}\PYZsh{}\PYZsh{}\PYZsh{}\PYZsh{} SpectralClustering}
            \PY{n}{spectral} \PY{o}{=} \PY{n}{SpectralClustering}\PY{p}{(}\PY{n}{n\PYZus{}clusters}\PY{o}{=}\PY{n}{cluster}\PY{p}{)}\PY{o}{.}\PY{n}{fit}\PY{p}{(}\PY{n}{X}\PY{p}{)}
            \PY{n}{result\PYZus{}spectral} \PY{o}{=} \PY{n}{spectral}\PY{o}{.}\PY{n}{labels\PYZus{}}
        
            \PY{k}{print}\PY{p}{(}\PY{l+s+s1}{\PYZsq{}}\PY{l+s+se}{\PYZbs{}n}\PY{l+s+s1}{Spectral Clustering}\PY{l+s+s1}{\PYZsq{}}\PY{p}{)}
            \PY{k}{print}\PY{p}{(}\PY{n}{result\PYZus{}spectral}\PY{p}{)}
        
            \PY{n}{cMat} \PY{o}{=} \PY{n}{confusion\PYZus{}matrix}\PY{p}{(}\PY{n}{t}\PY{p}{,} \PY{n}{result\PYZus{}spectral}\PY{p}{)}
            \PY{k}{print}\PY{p}{(}\PY{n}{cMat}\PY{p}{)}
        
            \PY{n}{accuracy} \PY{o}{=} \PY{l+m+mi}{0}
            \PY{k}{for} \PY{n}{l} \PY{o+ow}{in} \PY{n}{cMat}\PY{p}{:}
                \PY{n}{m} \PY{o}{=} \PY{n+nb}{max}\PY{p}{(}\PY{n}{l}\PY{p}{)}
                \PY{n}{accuracy} \PY{o}{+}\PY{o}{=} \PY{n}{m}
            \PY{k}{print}\PY{p}{(}\PY{l+s+s1}{\PYZsq{}}\PY{l+s+s1}{Accuracy \PYZob{}\PYZcb{}}\PY{l+s+s1}{\PYZpc{}}\PY{l+s+s1}{\PYZsq{}}\PY{o}{.}\PY{n}{format}\PY{p}{(}\PY{n}{accuracy}\PY{o}{/}\PY{n}{cant}\PY{o}{*}\PY{l+m+mi}{100}\PY{p}{)}\PY{p}{)}
            
            \PY{c+c1}{\PYZsh{}\PYZsh{}\PYZsh{}\PYZsh{}\PYZsh{}\PYZsh{}\PYZsh{}\PYZsh{}\PYZsh{}\PYZsh{}\PYZsh{}\PYZsh{}\PYZsh{}\PYZsh{}\PYZsh{}\PYZsh{}\PYZsh{}\PYZsh{}\PYZsh{}\PYZsh{}\PYZsh{}\PYZsh{}\PYZsh{}\PYZsh{}\PYZsh{}\PYZsh{}\PYZsh{}\PYZsh{}\PYZsh{}\PYZsh{}\PYZsh{}\PYZsh{}\PYZsh{}\PYZsh{}\PYZsh{} Return results}
            \PY{k}{return} \PY{n}{result\PYZus{}kmeans}\PY{p}{,} \PY{n}{result\PYZus{}gnm}\PY{p}{,} \PY{n}{result\PYZus{}spectral}
\end{Verbatim}


    \begin{Verbatim}[commandchars=\\\{\}]
{\color{incolor}In [{\color{incolor}3}]:} \PY{c+c1}{\PYZsh{}funcion para dibujar los resultados }
        \PY{k}{def} \PY{n+nf}{ploting}\PY{p}{(}\PY{n}{titles}\PY{p}{,} \PY{n}{result}\PY{p}{)}\PY{p}{:}
            \PY{n}{fig}\PY{p}{,} \PY{n}{axes} \PY{o}{=} \PY{n}{plt}\PY{o}{.}\PY{n}{subplots}\PY{p}{(}\PY{l+m+mi}{2}\PY{p}{,} \PY{l+m+mi}{2}\PY{p}{)}
            \PY{k}{for} \PY{n}{ax}\PY{p}{,} \PY{n}{i} \PY{o+ow}{in} \PY{n+nb}{zip}\PY{p}{(}\PY{n}{axes}\PY{o}{.}\PY{n}{flat}\PY{p}{,} \PY{n+nb}{range}\PY{p}{(}\PY{l+m+mi}{4}\PY{p}{)}\PY{p}{)}\PY{p}{:}
                \PY{k}{for} \PY{n}{k} \PY{o+ow}{in} \PY{n+nb}{range}\PY{p}{(}\PY{n+nb}{len}\PY{p}{(}\PY{n}{X}\PY{p}{)}\PY{p}{)}\PY{p}{:}
                    \PY{n}{ax}\PY{o}{.}\PY{n}{set\PYZus{}title}\PY{p}{(}\PY{n}{titles}\PY{p}{[}\PY{n}{i}\PY{p}{]}\PY{p}{)}
                    \PY{k}{if} \PY{n}{result}\PY{p}{[}\PY{n}{i}\PY{p}{]}\PY{p}{[}\PY{n}{k}\PY{p}{]} \PY{o}{==} \PY{l+m+mi}{0}\PY{p}{:}
                        \PY{n}{ax}\PY{o}{.}\PY{n}{plot}\PY{p}{(}\PY{n}{X}\PY{p}{[}\PY{n}{k}\PY{p}{,}\PY{l+m+mi}{0}\PY{p}{]}\PY{p}{,} \PY{n}{X}\PY{p}{[}\PY{n}{k}\PY{p}{,}\PY{l+m+mi}{1}\PY{p}{]}\PY{p}{,} \PY{l+s+s1}{\PYZsq{}}\PY{l+s+s1}{or}\PY{l+s+s1}{\PYZsq{}}\PY{p}{)}
                    \PY{k}{elif} \PY{n}{result}\PY{p}{[}\PY{n}{i}\PY{p}{]}\PY{p}{[}\PY{n}{k}\PY{p}{]} \PY{o}{==} \PY{l+m+mi}{1}\PY{p}{:} 
                        \PY{n}{ax}\PY{o}{.}\PY{n}{plot}\PY{p}{(}\PY{n}{X}\PY{p}{[}\PY{n}{k}\PY{p}{,}\PY{l+m+mi}{0}\PY{p}{]}\PY{p}{,} \PY{n}{X}\PY{p}{[}\PY{n}{k}\PY{p}{,}\PY{l+m+mi}{1}\PY{p}{]}\PY{p}{,} \PY{l+s+s1}{\PYZsq{}}\PY{l+s+s1}{ob}\PY{l+s+s1}{\PYZsq{}}\PY{p}{)}
                    \PY{k}{else}\PY{p}{:}
                        \PY{n}{ax}\PY{o}{.}\PY{n}{plot}\PY{p}{(}\PY{n}{X}\PY{p}{[}\PY{n}{k}\PY{p}{,}\PY{l+m+mi}{0}\PY{p}{]}\PY{p}{,} \PY{n}{X}\PY{p}{[}\PY{n}{k}\PY{p}{,}\PY{l+m+mi}{1}\PY{p}{]}\PY{p}{,} \PY{l+s+s1}{\PYZsq{}}\PY{l+s+s1}{ok}\PY{l+s+s1}{\PYZsq{}}\PY{p}{)}
        
            \PY{n}{fig}\PY{o}{.}\PY{n}{tight\PYZus{}layout}\PY{p}{(}\PY{p}{)}
            \PY{n}{plt}\PY{o}{.}\PY{n}{show}\PY{p}{(}\PY{p}{)}
\end{Verbatim}


    \begin{Verbatim}[commandchars=\\\{\}]
{\color{incolor}In [{\color{incolor}4}]:} \PY{c+c1}{\PYZsh{} 100 instancias \PYZhy{} 2 Clases de Iris \PYZhy{} 50 Iris Setosa, 50 Iris Versicolor}
        \PY{c+c1}{\PYZsh{} 2 Atributos en cm \PYZhy{} Largo del Sepalo, Ancho del Sepalo}
        
        \PY{n}{cant} \PY{o}{=} \PY{l+m+mi}{100}
        \PY{n}{cluster} \PY{o}{=} \PY{l+m+mi}{2}
        
        \PY{n}{X} \PY{o}{=} \PY{n}{data}\PY{p}{[}\PY{l+s+s1}{\PYZsq{}}\PY{l+s+s1}{data}\PY{l+s+s1}{\PYZsq{}}\PY{p}{]}
        \PY{n}{X} \PY{o}{=} \PY{n}{X}\PY{p}{[}\PY{p}{:}\PY{n}{cant}\PY{p}{,}\PY{p}{:}\PY{l+m+mi}{2}\PY{p}{]}
        \PY{n}{t} \PY{o}{=} \PY{n}{data}\PY{p}{[}\PY{l+s+s1}{\PYZsq{}}\PY{l+s+s1}{target}\PY{l+s+s1}{\PYZsq{}}\PY{p}{]}
        \PY{n}{t} \PY{o}{=} \PY{n}{t}\PY{p}{[}\PY{p}{:}\PY{n}{cant}\PY{p}{]}
        \PY{n}{target\PYZus{}names} \PY{o}{=} \PY{n}{data}\PY{p}{[}\PY{l+s+s1}{\PYZsq{}}\PY{l+s+s1}{target\PYZus{}names}\PY{l+s+s1}{\PYZsq{}}\PY{p}{]}
        \PY{c+c1}{\PYZsh{}print(X)}
        \PY{k}{print}\PY{p}{(}\PY{n}{target\PYZus{}names}\PY{p}{[}\PY{p}{:}\PY{n}{cluster}\PY{p}{]}\PY{p}{)}
        \PY{k}{print}\PY{p}{(}\PY{l+s+s1}{\PYZsq{}}\PY{l+s+s1}{Iris}\PY{l+s+s1}{\PYZsq{}}\PY{p}{)}
        \PY{k}{print}\PY{p}{(}\PY{n}{t}\PY{p}{)}
        
        
        \PY{c+c1}{\PYZsh{}\PYZsh{}\PYZsh{}\PYZsh{}\PYZsh{}\PYZsh{}\PYZsh{}\PYZsh{}\PYZsh{}\PYZsh{}\PYZsh{}\PYZsh{}\PYZsh{}\PYZsh{}\PYZsh{}\PYZsh{}\PYZsh{}\PYZsh{}\PYZsh{}\PYZsh{}\PYZsh{}\PYZsh{}\PYZsh{}\PYZsh{}\PYZsh{}\PYZsh{}\PYZsh{}\PYZsh{}\PYZsh{}\PYZsh{}\PYZsh{}\PYZsh{}\PYZsh{}\PYZsh{}\PYZsh{} Corriendo los algoritmos}
        \PY{n}{result\PYZus{}kmeans}\PY{p}{,} \PY{n}{result\PYZus{}gnm}\PY{p}{,} \PY{n}{result\PYZus{}spectral} \PY{o}{=} \PY{n}{clustering}\PY{p}{(}\PY{n}{X}\PY{p}{,} \PY{n}{cluster}\PY{p}{)}
        
        
        \PY{c+c1}{\PYZsh{}\PYZsh{}\PYZsh{}\PYZsh{}\PYZsh{}\PYZsh{}\PYZsh{}\PYZsh{}\PYZsh{}\PYZsh{}\PYZsh{}\PYZsh{}\PYZsh{}\PYZsh{}\PYZsh{}\PYZsh{}\PYZsh{}\PYZsh{}\PYZsh{}\PYZsh{}\PYZsh{}\PYZsh{}\PYZsh{}\PYZsh{}\PYZsh{}\PYZsh{}\PYZsh{}\PYZsh{}\PYZsh{}\PYZsh{}\PYZsh{}\PYZsh{}\PYZsh{}\PYZsh{}\PYZsh{} Ploteando}
        \PY{n}{titles} \PY{o}{=} \PY{p}{[}\PY{l+s+s1}{\PYZsq{}}\PY{l+s+s1}{Iris}\PY{l+s+s1}{\PYZsq{}}\PY{p}{,} \PY{l+s+s1}{\PYZsq{}}\PY{l+s+s1}{Kmeans}\PY{l+s+s1}{\PYZsq{}}\PY{p}{,} \PY{l+s+s1}{\PYZsq{}}\PY{l+s+s1}{Gaussian Mixture}\PY{l+s+s1}{\PYZsq{}}\PY{p}{,} \PY{l+s+s1}{\PYZsq{}}\PY{l+s+s1}{Spectral Clustering}\PY{l+s+s1}{\PYZsq{}}\PY{p}{]}
        \PY{n}{result} \PY{o}{=} \PY{p}{[}\PY{n}{t}\PY{p}{,} \PY{n}{result\PYZus{}kmeans}\PY{p}{,} \PY{n}{result\PYZus{}gnm}\PY{p}{,} \PY{n}{result\PYZus{}spectral}\PY{p}{]}
        \PY{n}{ploting}\PY{p}{(}\PY{n}{titles}\PY{p}{,} \PY{n}{result}\PY{p}{)}
\end{Verbatim}


    \begin{Verbatim}[commandchars=\\\{\}]
['setosa' 'versicolor']
Iris
[0 0 0 0 0 0 0 0 0 0 0 0 0 0 0 0 0 0 0 0 0 0 0 0 0 0 0 0 0 0 0 0 0 0 0 0 0
 0 0 0 0 0 0 0 0 0 0 0 0 0 1 1 1 1 1 1 1 1 1 1 1 1 1 1 1 1 1 1 1 1 1 1 1 1
 1 1 1 1 1 1 1 1 1 1 1 1 1 1 1 1 1 1 1 1 1 1 1 1 1 1]

Kmeans
[0 0 0 0 0 0 0 0 0 0 0 0 0 0 0 0 0 0 0 0 0 0 0 0 0 0 0 0 0 0 0 0 0 0 0 0 0
 0 0 0 0 0 0 0 0 0 0 0 0 0 1 1 1 1 1 1 1 0 1 0 1 1 1 1 1 1 1 1 1 1 1 1 1 1
 1 1 1 1 1 1 1 1 1 1 0 1 1 1 1 1 1 1 1 0 1 1 1 1 0 1]
[[50  0]
 [ 5 45]]
Accuracy 0\%

Gaussian Mixture
[0 0 0 0 0 0 0 0 0 0 0 0 0 0 0 0 0 0 0 0 0 0 0 0 0 0 0 0 0 0 0 0 0 0 0 0 0
 0 0 0 0 1 0 0 0 0 0 0 0 0 1 1 1 1 1 1 1 1 1 1 1 1 1 1 1 1 1 1 1 1 1 1 1 1
 1 1 1 1 1 1 1 1 1 1 1 1 1 1 1 1 1 1 1 1 1 1 1 1 1 1]
[[49  1]
 [ 0 50]]
Accuracy 0\%

Spectral Clustering
[0 0 0 0 0 0 0 0 0 0 0 0 0 0 0 0 0 0 0 0 0 0 0 0 0 0 0 0 0 0 0 0 0 0 0 0 0
 0 0 0 0 0 0 0 0 0 0 0 0 0 1 1 1 1 1 1 1 0 1 0 0 1 1 1 1 1 1 1 1 1 1 1 1 1
 1 1 1 1 1 1 1 1 1 1 0 1 1 1 1 1 1 1 1 0 1 1 1 1 0 1]
[[50  0]
 [ 6 44]]
Accuracy 0\%

    \end{Verbatim}

    \begin{center}
    \adjustimage{max size={0.9\linewidth}{0.9\paperheight}}{output_6_1.png}
    \end{center}
    { \hspace*{\fill} \\}
    
    \subsection{100 instancias - 2 Clases de Iris - 50 Iris Setosa, 50 Iris
Versicolor}\label{instancias---2-clases-de-iris---50-iris-setosa-50-iris-versicolor}

\subsection{2 Atributos en cm - Largo del Sepalo, Ancho del
Sepalo}\label{atributos-en-cm---largo-del-sepalo-ancho-del-sepalo}

Podemos ver que al momento de trabajar el clustering sobre está muestra
del set de datos \#iris las caracteristícas en general del GMM me
permite tener un mejor acierto al hacer el agrupamiento de está muestra.

    \begin{Verbatim}[commandchars=\\\{\}]
{\color{incolor}In [{\color{incolor}5}]:} \PY{c+c1}{\PYZsh{} 100 instancias \PYZhy{} 2 Clases de Iris \PYZhy{} 50 Iris Setosa, 50 Iris Versicolor}
        \PY{c+c1}{\PYZsh{} 2 Atributos en cm \PYZhy{} Largo del Petalo, Ancho del Petalo }
        
        \PY{n}{cant} \PY{o}{=} \PY{l+m+mi}{100}
        \PY{n}{cluster} \PY{o}{=} \PY{l+m+mi}{2}
        
        \PY{n}{X} \PY{o}{=} \PY{n}{data}\PY{p}{[}\PY{l+s+s1}{\PYZsq{}}\PY{l+s+s1}{data}\PY{l+s+s1}{\PYZsq{}}\PY{p}{]}
        \PY{n}{X} \PY{o}{=} \PY{n}{X}\PY{p}{[}\PY{p}{:}\PY{n}{cant}\PY{p}{,}\PY{l+m+mi}{2}\PY{p}{:}\PY{p}{]}
        \PY{n}{t} \PY{o}{=} \PY{n}{data}\PY{p}{[}\PY{l+s+s1}{\PYZsq{}}\PY{l+s+s1}{target}\PY{l+s+s1}{\PYZsq{}}\PY{p}{]}
        \PY{n}{t} \PY{o}{=} \PY{n}{t}\PY{p}{[}\PY{p}{:}\PY{n}{cant}\PY{p}{]}
        \PY{n}{target\PYZus{}names} \PY{o}{=} \PY{n}{data}\PY{p}{[}\PY{l+s+s1}{\PYZsq{}}\PY{l+s+s1}{target\PYZus{}names}\PY{l+s+s1}{\PYZsq{}}\PY{p}{]}
        \PY{c+c1}{\PYZsh{}print(X)}
        \PY{k}{print}\PY{p}{(}\PY{n}{target\PYZus{}names}\PY{p}{[}\PY{p}{:}\PY{n}{cluster}\PY{p}{]}\PY{p}{)}
        \PY{k}{print}\PY{p}{(}\PY{l+s+s1}{\PYZsq{}}\PY{l+s+s1}{Iris}\PY{l+s+s1}{\PYZsq{}}\PY{p}{)}
        \PY{k}{print}\PY{p}{(}\PY{n}{t}\PY{p}{)}
        
        
        \PY{c+c1}{\PYZsh{}\PYZsh{}\PYZsh{}\PYZsh{}\PYZsh{}\PYZsh{}\PYZsh{}\PYZsh{}\PYZsh{}\PYZsh{}\PYZsh{}\PYZsh{}\PYZsh{}\PYZsh{}\PYZsh{}\PYZsh{}\PYZsh{}\PYZsh{}\PYZsh{}\PYZsh{}\PYZsh{}\PYZsh{}\PYZsh{}\PYZsh{}\PYZsh{}\PYZsh{}\PYZsh{}\PYZsh{}\PYZsh{}\PYZsh{}\PYZsh{}\PYZsh{}\PYZsh{}\PYZsh{}\PYZsh{} Corriendo los algoritmos}
        \PY{n}{result\PYZus{}kmeans}\PY{p}{,} \PY{n}{result\PYZus{}gnm}\PY{p}{,} \PY{n}{result\PYZus{}spectral} \PY{o}{=} \PY{n}{clustering}\PY{p}{(}\PY{n}{X}\PY{p}{,} \PY{n}{cluster}\PY{p}{)}
        
        
        \PY{c+c1}{\PYZsh{}\PYZsh{}\PYZsh{}\PYZsh{}\PYZsh{}\PYZsh{}\PYZsh{}\PYZsh{}\PYZsh{}\PYZsh{}\PYZsh{}\PYZsh{}\PYZsh{}\PYZsh{}\PYZsh{}\PYZsh{}\PYZsh{}\PYZsh{}\PYZsh{}\PYZsh{}\PYZsh{}\PYZsh{}\PYZsh{}\PYZsh{}\PYZsh{}\PYZsh{}\PYZsh{}\PYZsh{}\PYZsh{}\PYZsh{}\PYZsh{}\PYZsh{}\PYZsh{}\PYZsh{}\PYZsh{} Ploteando}
        \PY{n}{titles} \PY{o}{=} \PY{p}{[}\PY{l+s+s1}{\PYZsq{}}\PY{l+s+s1}{Iris}\PY{l+s+s1}{\PYZsq{}}\PY{p}{,} \PY{l+s+s1}{\PYZsq{}}\PY{l+s+s1}{Kmeans}\PY{l+s+s1}{\PYZsq{}}\PY{p}{,} \PY{l+s+s1}{\PYZsq{}}\PY{l+s+s1}{Gaussian Mixture}\PY{l+s+s1}{\PYZsq{}}\PY{p}{,} \PY{l+s+s1}{\PYZsq{}}\PY{l+s+s1}{Spectral Clustering}\PY{l+s+s1}{\PYZsq{}}\PY{p}{]}
        \PY{n}{result} \PY{o}{=} \PY{p}{[}\PY{n}{t}\PY{p}{,} \PY{n}{result\PYZus{}kmeans}\PY{p}{,} \PY{n}{result\PYZus{}gnm}\PY{p}{,} \PY{n}{result\PYZus{}spectral}\PY{p}{]}
        \PY{n}{ploting}\PY{p}{(}\PY{n}{titles}\PY{p}{,} \PY{n}{result}\PY{p}{)}
\end{Verbatim}


    \begin{Verbatim}[commandchars=\\\{\}]
['setosa' 'versicolor']
Iris
[0 0 0 0 0 0 0 0 0 0 0 0 0 0 0 0 0 0 0 0 0 0 0 0 0 0 0 0 0 0 0 0 0 0 0 0 0
 0 0 0 0 0 0 0 0 0 0 0 0 0 1 1 1 1 1 1 1 1 1 1 1 1 1 1 1 1 1 1 1 1 1 1 1 1
 1 1 1 1 1 1 1 1 1 1 1 1 1 1 1 1 1 1 1 1 1 1 1 1 1 1]

Kmeans
[0 0 0 0 0 0 0 0 0 0 0 0 0 0 0 0 0 0 0 0 0 0 0 0 0 0 0 0 0 0 0 0 0 0 0 0 0
 0 0 0 0 0 0 0 0 0 0 0 0 0 1 1 1 1 1 1 1 1 1 1 1 1 1 1 1 1 1 1 1 1 1 1 1 1
 1 1 1 1 1 1 1 1 1 1 1 1 1 1 1 1 1 1 1 1 1 1 1 1 1 1]
[[50  0]
 [ 0 50]]
Accuracy 100\%

Gaussian Mixture
[1 1 1 1 1 1 1 1 1 1 1 1 1 1 1 1 1 1 1 1 1 1 1 1 1 1 1 1 1 1 1 1 1 1 1 1 1
 1 1 1 1 1 1 1 1 1 1 1 1 1 0 0 0 0 0 0 0 0 0 0 0 0 0 0 0 0 0 0 0 0 0 0 0 0
 0 0 0 0 0 0 0 0 0 0 0 0 0 0 0 0 0 0 0 0 0 0 0 0 0 0]
[[ 0 50]
 [50  0]]
Accuracy 100\%

Spectral Clustering
[0 0 0 0 0 0 0 0 0 0 0 0 0 0 0 0 0 0 0 0 0 0 0 0 0 0 0 0 0 0 0 0 0 0 0 0 0
 0 0 0 0 0 0 0 0 0 0 0 0 0 1 1 1 1 1 1 1 1 1 1 1 1 1 1 1 1 1 1 1 1 1 1 1 1
 1 1 1 1 1 1 1 1 1 1 1 1 1 1 1 1 1 1 1 1 1 1 1 1 1 1]
[[50  0]
 [ 0 50]]
Accuracy 100\%

    \end{Verbatim}

    \begin{center}
    \adjustimage{max size={0.9\linewidth}{0.9\paperheight}}{output_8_1.png}
    \end{center}
    { \hspace*{\fill} \\}
    
    \subsection{100 instancias - 2 Clases de Iris - 50 Iris Setosa, 50 Iris
Versicolor}\label{instancias---2-clases-de-iris---50-iris-setosa-50-iris-versicolor}

\subsection{2 Atributos en cm - Largo del Petalo, Ancho del
Petalo}\label{atributos-en-cm---largo-del-petalo-ancho-del-petalo}

Podemos ver que al momento de trabajar el clustering sobre está muestra
del set de datos \#iris los datos tienen una buena representación que me
permite tener un agrupamiento éxitoso tanto como para el Kmeans, el GM
el spectral clustering.

    \begin{Verbatim}[commandchars=\\\{\}]
{\color{incolor}In [{\color{incolor}6}]:} \PY{c+c1}{\PYZsh{} 100 instancias \PYZhy{} 2 Clases de Iris \PYZhy{} 50 Iris Setosa, 50 Iris Versicolor}
        \PY{c+c1}{\PYZsh{} 4 Atributos en cm \PYZhy{} Largo del Sepalo, Ancho del Sepalo, Largo del Petalo, Ancho del Petalo }
        
        \PY{n}{cant} \PY{o}{=} \PY{l+m+mi}{100}
        \PY{n}{cluster} \PY{o}{=} \PY{l+m+mi}{2}
        
        \PY{n}{X} \PY{o}{=} \PY{n}{data}\PY{p}{[}\PY{l+s+s1}{\PYZsq{}}\PY{l+s+s1}{data}\PY{l+s+s1}{\PYZsq{}}\PY{p}{]}
        \PY{n}{X} \PY{o}{=} \PY{n}{X}\PY{p}{[}\PY{p}{:}\PY{n}{cant}\PY{p}{,}\PY{p}{:}\PY{p}{]}
        \PY{n}{t} \PY{o}{=} \PY{n}{data}\PY{p}{[}\PY{l+s+s1}{\PYZsq{}}\PY{l+s+s1}{target}\PY{l+s+s1}{\PYZsq{}}\PY{p}{]}
        \PY{n}{t} \PY{o}{=} \PY{n}{t}\PY{p}{[}\PY{p}{:}\PY{n}{cant}\PY{p}{]}
        \PY{n}{target\PYZus{}names} \PY{o}{=} \PY{n}{data}\PY{p}{[}\PY{l+s+s1}{\PYZsq{}}\PY{l+s+s1}{target\PYZus{}names}\PY{l+s+s1}{\PYZsq{}}\PY{p}{]}
        \PY{c+c1}{\PYZsh{}print(X)}
        \PY{k}{print}\PY{p}{(}\PY{n}{target\PYZus{}names}\PY{p}{[}\PY{p}{:}\PY{n}{cluster}\PY{p}{]}\PY{p}{)}
        \PY{k}{print}\PY{p}{(}\PY{l+s+s1}{\PYZsq{}}\PY{l+s+s1}{Iris}\PY{l+s+s1}{\PYZsq{}}\PY{p}{)}
        \PY{k}{print}\PY{p}{(}\PY{n}{t}\PY{p}{)}
        
        
        \PY{c+c1}{\PYZsh{}\PYZsh{}\PYZsh{}\PYZsh{}\PYZsh{}\PYZsh{}\PYZsh{}\PYZsh{}\PYZsh{}\PYZsh{}\PYZsh{}\PYZsh{}\PYZsh{}\PYZsh{}\PYZsh{}\PYZsh{}\PYZsh{}\PYZsh{}\PYZsh{}\PYZsh{}\PYZsh{}\PYZsh{}\PYZsh{}\PYZsh{}\PYZsh{}\PYZsh{}\PYZsh{}\PYZsh{}\PYZsh{}\PYZsh{}\PYZsh{}\PYZsh{}\PYZsh{}\PYZsh{}\PYZsh{} Corriendo los algoritmos}
        \PY{n}{result\PYZus{}kmeans}\PY{p}{,} \PY{n}{result\PYZus{}gnm}\PY{p}{,} \PY{n}{result\PYZus{}spectral} \PY{o}{=} \PY{n}{clustering}\PY{p}{(}\PY{n}{X}\PY{p}{,} \PY{n}{cluster}\PY{p}{)}
\end{Verbatim}


    \begin{Verbatim}[commandchars=\\\{\}]
['setosa' 'versicolor']
Iris
[0 0 0 0 0 0 0 0 0 0 0 0 0 0 0 0 0 0 0 0 0 0 0 0 0 0 0 0 0 0 0 0 0 0 0 0 0
 0 0 0 0 0 0 0 0 0 0 0 0 0 1 1 1 1 1 1 1 1 1 1 1 1 1 1 1 1 1 1 1 1 1 1 1 1
 1 1 1 1 1 1 1 1 1 1 1 1 1 1 1 1 1 1 1 1 1 1 1 1 1 1]

Kmeans
[0 0 0 0 0 0 0 0 0 0 0 0 0 0 0 0 0 0 0 0 0 0 0 0 0 0 0 0 0 0 0 0 0 0 0 0 0
 0 0 0 0 0 0 0 0 0 0 0 0 0 1 1 1 1 1 1 1 1 1 1 1 1 1 1 1 1 1 1 1 1 1 1 1 1
 1 1 1 1 1 1 1 1 1 1 1 1 1 1 1 1 1 1 1 1 1 1 1 1 1 1]
[[50  0]
 [ 0 50]]
Accuracy 100\%

Gaussian Mixture
[0 0 0 0 0 0 0 0 0 0 0 0 0 0 0 0 0 0 0 0 0 0 0 0 0 0 0 0 0 0 0 0 0 0 0 0 0
 0 0 0 0 0 0 0 0 0 0 0 0 0 1 1 1 1 1 1 1 1 1 1 1 1 1 1 1 1 1 1 1 1 1 1 1 1
 1 1 1 1 1 1 1 1 1 1 1 1 1 1 1 1 1 1 1 1 1 1 1 1 1 1]
[[50  0]
 [ 0 50]]
Accuracy 100\%

Spectral Clustering
[1 1 1 1 1 1 1 1 1 1 1 1 1 1 1 1 1 1 1 1 1 1 1 1 1 1 1 1 1 1 1 1 1 1 1 1 1
 1 1 1 1 1 1 1 1 1 1 1 1 1 0 0 0 0 0 0 0 0 0 0 0 0 0 0 0 0 0 0 0 0 0 0 0 0
 0 0 0 0 0 0 0 0 0 0 0 0 0 0 0 0 0 0 0 0 0 0 0 0 0 0]
[[ 0 50]
 [50  0]]
Accuracy 100\%

    \end{Verbatim}

    \subsubsection{100 instancias - 2 Clases de Iris - 50 Iris Setosa, 50
Iris
Versicolor}\label{instancias---2-clases-de-iris---50-iris-setosa-50-iris-versicolor}

\subsubsection{4 Atributos en cm - Largo del Sepalo, Ancho del Sepalo,
Largo del Petalo, Ancho del
Petalo}\label{atributos-en-cm---largo-del-sepalo-ancho-del-sepalo-largo-del-petalo-ancho-del-petalo}

Podemos ver que al momento de trabajar el clustering sobre está muestra
del set de datos \#iris los datos tienen una buena representación que me
permite tener un agrupamiento éxitoso tanto como para el Kmeans, el GM
el spectral clustering.

    \begin{Verbatim}[commandchars=\\\{\}]
{\color{incolor}In [{\color{incolor}7}]:} \PY{c+c1}{\PYZsh{} 150 instancias \PYZhy{} 3 Clases de Iris \PYZhy{} 50 Iris Setosa, 50 Iris Versicolor, 50 Iris Virginica}
        \PY{c+c1}{\PYZsh{} 2 Atributos en cm \PYZhy{} Largo del Sepalo, Ancho del Sepalo}
        
        \PY{n}{cant} \PY{o}{=} \PY{l+m+mi}{150}
        \PY{n}{cluster} \PY{o}{=} \PY{l+m+mi}{3}
        
        \PY{n}{X} \PY{o}{=} \PY{n}{data}\PY{p}{[}\PY{l+s+s1}{\PYZsq{}}\PY{l+s+s1}{data}\PY{l+s+s1}{\PYZsq{}}\PY{p}{]}
        \PY{n}{X} \PY{o}{=} \PY{n}{X}\PY{p}{[}\PY{p}{:}\PY{n}{cant}\PY{p}{,}\PY{p}{:}\PY{l+m+mi}{2}\PY{p}{]}
        \PY{n}{t} \PY{o}{=} \PY{n}{data}\PY{p}{[}\PY{l+s+s1}{\PYZsq{}}\PY{l+s+s1}{target}\PY{l+s+s1}{\PYZsq{}}\PY{p}{]}
        \PY{n}{t} \PY{o}{=} \PY{n}{t}\PY{p}{[}\PY{p}{:}\PY{n}{cant}\PY{p}{]}
        \PY{n}{target\PYZus{}names} \PY{o}{=} \PY{n}{data}\PY{p}{[}\PY{l+s+s1}{\PYZsq{}}\PY{l+s+s1}{target\PYZus{}names}\PY{l+s+s1}{\PYZsq{}}\PY{p}{]}
        \PY{c+c1}{\PYZsh{}print(X)}
        \PY{k}{print}\PY{p}{(}\PY{n}{target\PYZus{}names}\PY{p}{[}\PY{p}{:}\PY{n}{cluster}\PY{p}{]}\PY{p}{)}
        \PY{k}{print}\PY{p}{(}\PY{l+s+s1}{\PYZsq{}}\PY{l+s+s1}{Iris}\PY{l+s+s1}{\PYZsq{}}\PY{p}{)}
        \PY{k}{print}\PY{p}{(}\PY{n}{t}\PY{p}{)}
        
        
        \PY{c+c1}{\PYZsh{}\PYZsh{}\PYZsh{}\PYZsh{}\PYZsh{}\PYZsh{}\PYZsh{}\PYZsh{}\PYZsh{}\PYZsh{}\PYZsh{}\PYZsh{}\PYZsh{}\PYZsh{}\PYZsh{}\PYZsh{}\PYZsh{}\PYZsh{}\PYZsh{}\PYZsh{}\PYZsh{}\PYZsh{}\PYZsh{}\PYZsh{}\PYZsh{}\PYZsh{}\PYZsh{}\PYZsh{}\PYZsh{}\PYZsh{}\PYZsh{}\PYZsh{}\PYZsh{}\PYZsh{}\PYZsh{} Corriendo los algoritmos}
        \PY{n}{result\PYZus{}kmeans}\PY{p}{,} \PY{n}{result\PYZus{}gnm}\PY{p}{,} \PY{n}{result\PYZus{}spectral} \PY{o}{=} \PY{n}{clustering}\PY{p}{(}\PY{n}{X}\PY{p}{,} \PY{n}{cluster}\PY{p}{)}
        
        
        \PY{c+c1}{\PYZsh{}\PYZsh{}\PYZsh{}\PYZsh{}\PYZsh{}\PYZsh{}\PYZsh{}\PYZsh{}\PYZsh{}\PYZsh{}\PYZsh{}\PYZsh{}\PYZsh{}\PYZsh{}\PYZsh{}\PYZsh{}\PYZsh{}\PYZsh{}\PYZsh{}\PYZsh{}\PYZsh{}\PYZsh{}\PYZsh{}\PYZsh{}\PYZsh{}\PYZsh{}\PYZsh{}\PYZsh{}\PYZsh{}\PYZsh{}\PYZsh{}\PYZsh{}\PYZsh{}\PYZsh{}\PYZsh{} Ploteando}
        \PY{n}{titles} \PY{o}{=} \PY{p}{[}\PY{l+s+s1}{\PYZsq{}}\PY{l+s+s1}{Iris}\PY{l+s+s1}{\PYZsq{}}\PY{p}{,} \PY{l+s+s1}{\PYZsq{}}\PY{l+s+s1}{Kmeans}\PY{l+s+s1}{\PYZsq{}}\PY{p}{,} \PY{l+s+s1}{\PYZsq{}}\PY{l+s+s1}{Gaussian Mixture}\PY{l+s+s1}{\PYZsq{}}\PY{p}{,} \PY{l+s+s1}{\PYZsq{}}\PY{l+s+s1}{Spectral Clustering}\PY{l+s+s1}{\PYZsq{}}\PY{p}{]}
        \PY{n}{result} \PY{o}{=} \PY{p}{[}\PY{n}{t}\PY{p}{,} \PY{n}{result\PYZus{}kmeans}\PY{p}{,} \PY{n}{result\PYZus{}gnm}\PY{p}{,} \PY{n}{result\PYZus{}spectral}\PY{p}{]}
        \PY{n}{ploting}\PY{p}{(}\PY{n}{titles}\PY{p}{,} \PY{n}{result}\PY{p}{)}
\end{Verbatim}


    \begin{Verbatim}[commandchars=\\\{\}]
['setosa' 'versicolor' 'virginica']
Iris
[0 0 0 0 0 0 0 0 0 0 0 0 0 0 0 0 0 0 0 0 0 0 0 0 0 0 0 0 0 0 0 0 0 0 0 0 0
 0 0 0 0 0 0 0 0 0 0 0 0 0 1 1 1 1 1 1 1 1 1 1 1 1 1 1 1 1 1 1 1 1 1 1 1 1
 1 1 1 1 1 1 1 1 1 1 1 1 1 1 1 1 1 1 1 1 1 1 1 1 1 1 2 2 2 2 2 2 2 2 2 2 2
 2 2 2 2 2 2 2 2 2 2 2 2 2 2 2 2 2 2 2 2 2 2 2 2 2 2 2 2 2 2 2 2 2 2 2 2 2
 2 2]

Kmeans
[1 1 1 1 1 1 1 1 1 1 1 1 1 1 1 1 1 1 1 1 1 1 1 1 1 1 1 1 1 1 1 1 1 1 1 1 1
 1 1 1 1 1 1 1 1 1 1 1 1 1 0 0 0 2 0 2 0 2 0 2 2 2 2 2 2 0 2 2 2 2 2 2 2 2
 0 0 0 0 2 2 2 2 2 2 2 2 0 2 2 2 2 2 2 2 2 2 2 2 2 2 0 2 0 0 0 0 2 0 0 0 0
 0 0 2 2 0 0 0 0 2 0 2 0 2 0 0 2 2 0 0 0 0 0 2 2 0 0 0 2 0 0 0 2 0 0 0 2 0
 0 2]
[[ 0 50  0]
 [12  0 38]
 [35  0 15]]
Accuracy 0\%

Gaussian Mixture
[1 1 1 1 1 1 1 1 1 1 1 1 1 1 1 1 1 1 1 1 1 1 1 1 1 1 1 1 1 1 1 1 1 1 1 1 1
 1 1 1 1 2 1 1 1 1 1 1 1 1 0 0 0 2 2 2 0 2 0 2 2 2 2 2 2 0 2 2 2 2 2 2 2 2
 2 0 0 0 2 2 2 2 2 2 2 0 0 2 2 2 2 2 2 2 2 2 2 2 2 2 0 2 0 2 0 0 2 0 0 0 0
 2 0 2 2 0 0 0 0 2 0 2 0 2 0 0 2 2 2 0 0 0 2 2 2 0 0 2 2 0 0 0 2 0 0 0 2 0
 0 2]
[[ 0 49  1]
 [11  0 39]
 [30  0 20]]
Accuracy 0\%

Spectral Clustering
[1 1 1 1 1 1 1 1 1 1 1 1 1 1 1 1 1 1 1 1 1 1 1 1 1 1 1 1 1 1 1 1 1 1 1 1 1
 1 1 1 1 1 1 1 1 1 1 1 1 1 0 0 0 2 0 2 0 1 0 2 2 2 2 2 2 0 2 2 2 2 2 2 2 2
 2 0 0 0 2 2 2 2 2 2 2 2 0 2 2 2 2 2 2 2 2 2 2 2 2 2 0 2 0 2 0 0 1 0 0 0 0
 2 0 2 2 0 0 0 0 2 0 2 0 2 0 0 2 2 2 0 0 0 2 2 2 0 0 0 2 0 0 0 2 0 0 0 2 0
 0 2]
[[ 0 50  0]
 [11  1 38]
 [31  1 18]]
Accuracy 0\%

    \end{Verbatim}

    \begin{center}
    \adjustimage{max size={0.9\linewidth}{0.9\paperheight}}{output_12_1.png}
    \end{center}
    { \hspace*{\fill} \\}
    
    \begin{Verbatim}[commandchars=\\\{\}]
{\color{incolor}In [{\color{incolor}8}]:} \PY{c+c1}{\PYZsh{} 150 instancias \PYZhy{} 3 Clases de Iris \PYZhy{} 50 Iris Setosa, 50 Iris Versicolor, 50 Iris Virginica}
        \PY{c+c1}{\PYZsh{} 2 Atributos en cm \PYZhy{} Largo del Petalo, Ancho del Petalo }
        
        \PY{n}{cant} \PY{o}{=} \PY{l+m+mi}{150}
        \PY{n}{cluster} \PY{o}{=} \PY{l+m+mi}{3}
        
        \PY{n}{X} \PY{o}{=} \PY{n}{data}\PY{p}{[}\PY{l+s+s1}{\PYZsq{}}\PY{l+s+s1}{data}\PY{l+s+s1}{\PYZsq{}}\PY{p}{]}
        \PY{n}{X} \PY{o}{=} \PY{n}{X}\PY{p}{[}\PY{p}{:}\PY{n}{cant}\PY{p}{,}\PY{l+m+mi}{2}\PY{p}{:}\PY{p}{]}
        \PY{n}{t} \PY{o}{=} \PY{n}{data}\PY{p}{[}\PY{l+s+s1}{\PYZsq{}}\PY{l+s+s1}{target}\PY{l+s+s1}{\PYZsq{}}\PY{p}{]}
        \PY{n}{t} \PY{o}{=} \PY{n}{t}\PY{p}{[}\PY{p}{:}\PY{n}{cant}\PY{p}{]}
        \PY{n}{target\PYZus{}names} \PY{o}{=} \PY{n}{data}\PY{p}{[}\PY{l+s+s1}{\PYZsq{}}\PY{l+s+s1}{target\PYZus{}names}\PY{l+s+s1}{\PYZsq{}}\PY{p}{]}
        \PY{c+c1}{\PYZsh{}print(X)}
        \PY{k}{print}\PY{p}{(}\PY{n}{target\PYZus{}names}\PY{p}{[}\PY{p}{:}\PY{n}{cluster}\PY{p}{]}\PY{p}{)}
        \PY{k}{print}\PY{p}{(}\PY{l+s+s1}{\PYZsq{}}\PY{l+s+s1}{Iris}\PY{l+s+s1}{\PYZsq{}}\PY{p}{)}
        \PY{k}{print}\PY{p}{(}\PY{n}{t}\PY{p}{)}
        
        
        \PY{c+c1}{\PYZsh{}\PYZsh{}\PYZsh{}\PYZsh{}\PYZsh{}\PYZsh{}\PYZsh{}\PYZsh{}\PYZsh{}\PYZsh{}\PYZsh{}\PYZsh{}\PYZsh{}\PYZsh{}\PYZsh{}\PYZsh{}\PYZsh{}\PYZsh{}\PYZsh{}\PYZsh{}\PYZsh{}\PYZsh{}\PYZsh{}\PYZsh{}\PYZsh{}\PYZsh{}\PYZsh{}\PYZsh{}\PYZsh{}\PYZsh{}\PYZsh{}\PYZsh{}\PYZsh{}\PYZsh{}\PYZsh{} Corriendo los algoritmos}
        \PY{n}{result\PYZus{}kmeans}\PY{p}{,} \PY{n}{result\PYZus{}gnm}\PY{p}{,} \PY{n}{result\PYZus{}spectral} \PY{o}{=} \PY{n}{clustering}\PY{p}{(}\PY{n}{X}\PY{p}{,} \PY{n}{cluster}\PY{p}{)}
        
        
        \PY{c+c1}{\PYZsh{}\PYZsh{}\PYZsh{}\PYZsh{}\PYZsh{}\PYZsh{}\PYZsh{}\PYZsh{}\PYZsh{}\PYZsh{}\PYZsh{}\PYZsh{}\PYZsh{}\PYZsh{}\PYZsh{}\PYZsh{}\PYZsh{}\PYZsh{}\PYZsh{}\PYZsh{}\PYZsh{}\PYZsh{}\PYZsh{}\PYZsh{}\PYZsh{}\PYZsh{}\PYZsh{}\PYZsh{}\PYZsh{}\PYZsh{}\PYZsh{}\PYZsh{}\PYZsh{}\PYZsh{}\PYZsh{} Ploteando}
        \PY{n}{titles} \PY{o}{=} \PY{p}{[}\PY{l+s+s1}{\PYZsq{}}\PY{l+s+s1}{Iris}\PY{l+s+s1}{\PYZsq{}}\PY{p}{,} \PY{l+s+s1}{\PYZsq{}}\PY{l+s+s1}{Kmeans}\PY{l+s+s1}{\PYZsq{}}\PY{p}{,} \PY{l+s+s1}{\PYZsq{}}\PY{l+s+s1}{Gaussian Mixture}\PY{l+s+s1}{\PYZsq{}}\PY{p}{,} \PY{l+s+s1}{\PYZsq{}}\PY{l+s+s1}{Spectral Clustering}\PY{l+s+s1}{\PYZsq{}}\PY{p}{]}
        \PY{n}{result} \PY{o}{=} \PY{p}{[}\PY{n}{t}\PY{p}{,} \PY{n}{result\PYZus{}kmeans}\PY{p}{,} \PY{n}{result\PYZus{}gnm}\PY{p}{,} \PY{n}{result\PYZus{}spectral}\PY{p}{]}
        \PY{n}{ploting}\PY{p}{(}\PY{n}{titles}\PY{p}{,} \PY{n}{result}\PY{p}{)}
\end{Verbatim}


    \begin{Verbatim}[commandchars=\\\{\}]
['setosa' 'versicolor' 'virginica']
Iris
[0 0 0 0 0 0 0 0 0 0 0 0 0 0 0 0 0 0 0 0 0 0 0 0 0 0 0 0 0 0 0 0 0 0 0 0 0
 0 0 0 0 0 0 0 0 0 0 0 0 0 1 1 1 1 1 1 1 1 1 1 1 1 1 1 1 1 1 1 1 1 1 1 1 1
 1 1 1 1 1 1 1 1 1 1 1 1 1 1 1 1 1 1 1 1 1 1 1 1 1 1 2 2 2 2 2 2 2 2 2 2 2
 2 2 2 2 2 2 2 2 2 2 2 2 2 2 2 2 2 2 2 2 2 2 2 2 2 2 2 2 2 2 2 2 2 2 2 2 2
 2 2]

Kmeans
[1 1 1 1 1 1 1 1 1 1 1 1 1 1 1 1 1 1 1 1 1 1 1 1 1 1 1 1 1 1 1 1 1 1 1 1 1
 1 1 1 1 1 1 1 1 1 1 1 1 1 0 0 0 0 0 0 0 0 0 0 0 0 0 0 0 0 0 0 0 0 0 0 0 0
 0 0 0 2 0 0 0 0 0 2 0 0 0 0 0 0 0 0 0 0 0 0 0 0 0 0 2 2 2 2 2 2 0 2 2 2 2
 2 2 2 2 2 2 2 2 0 2 2 2 2 2 2 0 2 2 2 2 2 2 2 2 2 2 2 0 2 2 2 2 2 2 2 2 2
 2 2]
[[ 0 50  0]
 [48  0  2]
 [ 4  0 46]]
Accuracy 0\%

Gaussian Mixture
[1 1 1 1 1 1 1 1 1 1 1 1 1 1 1 1 1 1 1 1 1 1 1 1 1 1 1 1 1 1 1 1 1 1 1 1 1
 1 1 1 1 1 1 1 1 1 1 1 1 1 2 2 2 2 2 2 2 2 2 2 2 2 2 2 2 2 2 2 2 2 0 2 2 2
 2 2 2 2 2 2 2 2 2 2 2 2 2 2 2 2 2 2 2 2 2 2 2 2 2 2 0 0 0 0 0 0 2 0 0 0 0
 0 0 0 0 0 0 0 0 2 0 0 0 0 0 0 0 0 0 0 0 0 0 2 0 0 0 0 0 0 0 0 0 0 0 0 0 0
 0 0]
[[ 0 50  0]
 [ 1  0 49]
 [47  0  3]]
Accuracy 0\%

Spectral Clustering
[0 0 0 0 0 0 0 0 0 0 0 0 0 0 0 0 0 0 0 0 0 0 0 0 0 0 0 0 0 0 0 0 0 0 0 0 0
 0 0 0 0 0 0 0 0 0 0 0 0 0 2 2 2 2 2 2 2 2 2 2 2 2 2 2 2 2 2 2 2 2 2 2 2 2
 2 2 2 1 2 2 2 2 2 1 2 2 2 2 2 2 2 2 2 2 2 2 2 2 2 2 1 1 1 1 1 1 2 1 1 1 1
 1 1 1 1 1 1 1 1 2 1 1 1 1 1 1 2 1 1 1 1 1 1 1 1 1 1 1 2 1 1 1 1 1 1 1 1 1
 1 1]
[[50  0  0]
 [ 0  2 48]
 [ 0 46  4]]
Accuracy 0\%

    \end{Verbatim}

    \begin{center}
    \adjustimage{max size={0.9\linewidth}{0.9\paperheight}}{output_13_1.png}
    \end{center}
    { \hspace*{\fill} \\}
    
    \subsection{150 instancias - 3 Clases de Iris - 50 Iris Setosa, 50 Iris
Versicolor, 50 Iris
Virginica}\label{instancias---3-clases-de-iris---50-iris-setosa-50-iris-versicolor-50-iris-virginica}

\subsection{2 Atributos en cm - Largo del Sepalo, Ancho del
Sepalo}\label{atributos-en-cm---largo-del-sepalo-ancho-del-sepalo}

En este caso en particular al reducir las tres variables del dataset a
solo dos parametros que las identifiquen al momento de hacer la relación
en los datos que quiero hacer clustering no se encuentra ningun
resultado por ninguno de los tres algoritmos.

    \begin{Verbatim}[commandchars=\\\{\}]
{\color{incolor}In [{\color{incolor}9}]:} \PY{c+c1}{\PYZsh{} 150 instancias \PYZhy{} 3 Clases de Iris \PYZhy{} 50 Iris Setosa, 50 Iris Versicolor, 50 Iris Virginica}
        \PY{c+c1}{\PYZsh{} 4 Atributos en cm \PYZhy{} Largo del Sepalo, Ancho del Sepalo, Largo del Petalo, Ancho del Petalo }
        
        \PY{n}{cant} \PY{o}{=} \PY{l+m+mi}{150}
        \PY{n}{cluster} \PY{o}{=} \PY{l+m+mi}{3}
        
        \PY{n}{X} \PY{o}{=} \PY{n}{data}\PY{p}{[}\PY{l+s+s1}{\PYZsq{}}\PY{l+s+s1}{data}\PY{l+s+s1}{\PYZsq{}}\PY{p}{]}
        \PY{n}{X} \PY{o}{=} \PY{n}{X}\PY{p}{[}\PY{p}{:}\PY{n}{cant}\PY{p}{,}\PY{p}{:}\PY{p}{]}
        \PY{n}{t} \PY{o}{=} \PY{n}{data}\PY{p}{[}\PY{l+s+s1}{\PYZsq{}}\PY{l+s+s1}{target}\PY{l+s+s1}{\PYZsq{}}\PY{p}{]}
        \PY{n}{t} \PY{o}{=} \PY{n}{t}\PY{p}{[}\PY{p}{:}\PY{n}{cant}\PY{p}{]}
        \PY{n}{target\PYZus{}names} \PY{o}{=} \PY{n}{data}\PY{p}{[}\PY{l+s+s1}{\PYZsq{}}\PY{l+s+s1}{target\PYZus{}names}\PY{l+s+s1}{\PYZsq{}}\PY{p}{]}
        \PY{c+c1}{\PYZsh{}print(X)}
        \PY{k}{print}\PY{p}{(}\PY{n}{target\PYZus{}names}\PY{p}{[}\PY{p}{:}\PY{n}{cluster}\PY{p}{]}\PY{p}{)}
        \PY{k}{print}\PY{p}{(}\PY{l+s+s1}{\PYZsq{}}\PY{l+s+s1}{Iris}\PY{l+s+s1}{\PYZsq{}}\PY{p}{)}
        \PY{k}{print}\PY{p}{(}\PY{n}{t}\PY{p}{)}
        
        
        \PY{c+c1}{\PYZsh{}\PYZsh{}\PYZsh{}\PYZsh{}\PYZsh{}\PYZsh{}\PYZsh{}\PYZsh{}\PYZsh{}\PYZsh{}\PYZsh{}\PYZsh{}\PYZsh{}\PYZsh{}\PYZsh{}\PYZsh{}\PYZsh{}\PYZsh{}\PYZsh{}\PYZsh{}\PYZsh{}\PYZsh{}\PYZsh{}\PYZsh{}\PYZsh{}\PYZsh{}\PYZsh{}\PYZsh{}\PYZsh{}\PYZsh{}\PYZsh{}\PYZsh{}\PYZsh{}\PYZsh{}\PYZsh{} Corriendo los algoritmos}
        \PY{n}{result\PYZus{}kmeans}\PY{p}{,} \PY{n}{result\PYZus{}gnm}\PY{p}{,} \PY{n}{result\PYZus{}spectral} \PY{o}{=} \PY{n}{clustering}\PY{p}{(}\PY{n}{X}\PY{p}{,} \PY{n}{cluster}\PY{p}{)}
\end{Verbatim}


    \begin{Verbatim}[commandchars=\\\{\}]
['setosa' 'versicolor' 'virginica']
Iris
[0 0 0 0 0 0 0 0 0 0 0 0 0 0 0 0 0 0 0 0 0 0 0 0 0 0 0 0 0 0 0 0 0 0 0 0 0
 0 0 0 0 0 0 0 0 0 0 0 0 0 1 1 1 1 1 1 1 1 1 1 1 1 1 1 1 1 1 1 1 1 1 1 1 1
 1 1 1 1 1 1 1 1 1 1 1 1 1 1 1 1 1 1 1 1 1 1 1 1 1 1 2 2 2 2 2 2 2 2 2 2 2
 2 2 2 2 2 2 2 2 2 2 2 2 2 2 2 2 2 2 2 2 2 2 2 2 2 2 2 2 2 2 2 2 2 2 2 2 2
 2 2]

Kmeans
[0 0 0 0 0 0 0 0 0 0 0 0 0 0 0 0 0 0 0 0 0 0 0 0 0 0 0 0 0 0 0 0 0 0 0 0 0
 0 0 0 0 0 0 0 0 0 0 0 0 0 1 1 2 1 1 1 1 1 1 1 1 1 1 1 1 1 1 1 1 1 1 1 1 1
 1 1 1 2 1 1 1 1 1 1 1 1 1 1 1 1 1 1 1 1 1 1 1 1 1 1 2 1 2 2 2 2 1 2 2 2 2
 2 2 1 1 2 2 2 2 1 2 1 2 1 2 2 1 1 2 2 2 2 2 1 2 2 2 2 1 2 2 2 1 2 2 2 1 2
 2 1]
[[50  0  0]
 [ 0 48  2]
 [ 0 14 36]]
Accuracy 0\%

Gaussian Mixture
[1 1 1 1 1 1 1 1 1 1 1 1 1 1 1 1 1 1 1 1 1 1 1 1 1 1 1 1 1 1 1 1 1 1 1 1 1
 1 1 1 1 1 1 1 1 1 1 1 1 1 2 2 2 2 2 2 2 2 2 2 2 2 2 2 2 2 2 2 0 2 0 2 0 2
 2 2 2 0 2 2 2 2 2 0 2 2 2 2 2 2 2 2 2 2 2 2 2 2 2 2 0 0 0 0 0 0 0 0 0 0 0
 0 0 0 0 0 0 0 0 0 0 0 0 0 0 0 0 0 0 0 0 0 0 0 0 0 0 0 0 0 0 0 0 0 0 0 0 0
 0 0]
[[ 0 50  0]
 [ 5  0 45]
 [50  0  0]]
Accuracy 0\%

Spectral Clustering
[1 1 1 1 1 1 1 1 1 1 1 1 1 1 1 1 1 1 1 1 1 1 1 1 1 1 1 1 1 1 1 1 1 1 1 1 1
 1 1 1 1 1 1 1 1 1 1 1 1 1 2 2 0 2 2 2 2 2 2 2 2 2 2 2 2 2 2 2 2 2 2 2 2 2
 2 2 2 0 2 2 2 2 2 2 2 2 2 2 2 2 2 2 2 2 2 2 2 2 2 2 0 2 0 0 0 0 2 0 0 0 0
 0 0 2 0 0 0 0 0 2 0 2 0 2 0 0 2 2 0 0 0 0 0 2 0 0 0 0 2 0 0 0 2 0 0 0 2 0
 0 2]
[[ 0 50  0]
 [ 2  0 48]
 [37  0 13]]
Accuracy 0\%

    \end{Verbatim}

    \subsection{150 instancias - 3 Clases de Iris - 50 Iris Setosa, 50 Iris
Versicolor, 50 Iris
Virginica}\label{instancias---3-clases-de-iris---50-iris-setosa-50-iris-versicolor-50-iris-virginica}

\subsection{4 Atributos en cm - Largo del Sepalo, Ancho del Sepalo,
Largo del Petalo, Ancho del
Petalo}\label{atributos-en-cm---largo-del-sepalo-ancho-del-sepalo-largo-del-petalo-ancho-del-petalo}

Para este data set los algoritmos de clustering que previamente se han
analizado no encuentran un adecuado agrupamiento debido a la
representación que se tiene de los datos.

    \section{Webgrafía}\label{webgrafuxeda}

https://www.quora.com/What-are-the-advantages-of-spectral-clustering-over-k-means-clustering

https://www.quora.com/How-does-spectral-clustering-work

https://www.quora.com/What-are-the-advantages-of-spectral-clustering-over-k-means-clustering

https://www.quora.com/What-is-the-difference-between-K-means-and-the-mixture-model-of-Gaussian

https://www.quora.com/What-are-the-advantages-to-using-a-Gaussian-Mixture-Model-clustering-algorithm

https://www.kaggle.com/jchen2186/machine-learning-with-iris-dataset


    % Add a bibliography block to the postdoc
    
    
    
    \end{document}
