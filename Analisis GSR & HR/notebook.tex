
% Default to the notebook output style

    


% Inherit from the specified cell style.




    
\documentclass[11pt]{article}

    
    
    \usepackage[T1]{fontenc}
    % Nicer default font (+ math font) than Computer Modern for most use cases
    \usepackage{mathpazo}

    % Basic figure setup, for now with no caption control since it's done
    % automatically by Pandoc (which extracts ![](path) syntax from Markdown).
    \usepackage{graphicx}
    % We will generate all images so they have a width \maxwidth. This means
    % that they will get their normal width if they fit onto the page, but
    % are scaled down if they would overflow the margins.
    \makeatletter
    \def\maxwidth{\ifdim\Gin@nat@width>\linewidth\linewidth
    \else\Gin@nat@width\fi}
    \makeatother
    \let\Oldincludegraphics\includegraphics
    % Set max figure width to be 80% of text width, for now hardcoded.
    \renewcommand{\includegraphics}[1]{\Oldincludegraphics[width=.8\maxwidth]{#1}}
    % Ensure that by default, figures have no caption (until we provide a
    % proper Figure object with a Caption API and a way to capture that
    % in the conversion process - todo).
    \usepackage{caption}
    \DeclareCaptionLabelFormat{nolabel}{}
    \captionsetup{labelformat=nolabel}

    \usepackage{adjustbox} % Used to constrain images to a maximum size 
    \usepackage{xcolor} % Allow colors to be defined
    \usepackage{enumerate} % Needed for markdown enumerations to work
    \usepackage{geometry} % Used to adjust the document margins
    \usepackage{amsmath} % Equations
    \usepackage{amssymb} % Equations
    \usepackage{textcomp} % defines textquotesingle
    % Hack from http://tex.stackexchange.com/a/47451/13684:
    \AtBeginDocument{%
        \def\PYZsq{\textquotesingle}% Upright quotes in Pygmentized code
    }
    \usepackage{upquote} % Upright quotes for verbatim code
    \usepackage{eurosym} % defines \euro
    \usepackage[mathletters]{ucs} % Extended unicode (utf-8) support
    \usepackage[utf8x]{inputenc} % Allow utf-8 characters in the tex document
    \usepackage{fancyvrb} % verbatim replacement that allows latex
    \usepackage{grffile} % extends the file name processing of package graphics 
                         % to support a larger range 
    % The hyperref package gives us a pdf with properly built
    % internal navigation ('pdf bookmarks' for the table of contents,
    % internal cross-reference links, web links for URLs, etc.)
    \usepackage{hyperref}
    \usepackage{longtable} % longtable support required by pandoc >1.10
    \usepackage{booktabs}  % table support for pandoc > 1.12.2
    \usepackage[inline]{enumitem} % IRkernel/repr support (it uses the enumerate* environment)
    \usepackage[normalem]{ulem} % ulem is needed to support strikethroughs (\sout)
                                % normalem makes italics be italics, not underlines
    

    
    
    % Colors for the hyperref package
    \definecolor{urlcolor}{rgb}{0,.145,.698}
    \definecolor{linkcolor}{rgb}{.71,0.21,0.01}
    \definecolor{citecolor}{rgb}{.12,.54,.11}

    % ANSI colors
    \definecolor{ansi-black}{HTML}{3E424D}
    \definecolor{ansi-black-intense}{HTML}{282C36}
    \definecolor{ansi-red}{HTML}{E75C58}
    \definecolor{ansi-red-intense}{HTML}{B22B31}
    \definecolor{ansi-green}{HTML}{00A250}
    \definecolor{ansi-green-intense}{HTML}{007427}
    \definecolor{ansi-yellow}{HTML}{DDB62B}
    \definecolor{ansi-yellow-intense}{HTML}{B27D12}
    \definecolor{ansi-blue}{HTML}{208FFB}
    \definecolor{ansi-blue-intense}{HTML}{0065CA}
    \definecolor{ansi-magenta}{HTML}{D160C4}
    \definecolor{ansi-magenta-intense}{HTML}{A03196}
    \definecolor{ansi-cyan}{HTML}{60C6C8}
    \definecolor{ansi-cyan-intense}{HTML}{258F8F}
    \definecolor{ansi-white}{HTML}{C5C1B4}
    \definecolor{ansi-white-intense}{HTML}{A1A6B2}

    % commands and environments needed by pandoc snippets
    % extracted from the output of `pandoc -s`
    \providecommand{\tightlist}{%
      \setlength{\itemsep}{0pt}\setlength{\parskip}{0pt}}
    \DefineVerbatimEnvironment{Highlighting}{Verbatim}{commandchars=\\\{\}}
    % Add ',fontsize=\small' for more characters per line
    \newenvironment{Shaded}{}{}
    \newcommand{\KeywordTok}[1]{\textcolor[rgb]{0.00,0.44,0.13}{\textbf{{#1}}}}
    \newcommand{\DataTypeTok}[1]{\textcolor[rgb]{0.56,0.13,0.00}{{#1}}}
    \newcommand{\DecValTok}[1]{\textcolor[rgb]{0.25,0.63,0.44}{{#1}}}
    \newcommand{\BaseNTok}[1]{\textcolor[rgb]{0.25,0.63,0.44}{{#1}}}
    \newcommand{\FloatTok}[1]{\textcolor[rgb]{0.25,0.63,0.44}{{#1}}}
    \newcommand{\CharTok}[1]{\textcolor[rgb]{0.25,0.44,0.63}{{#1}}}
    \newcommand{\StringTok}[1]{\textcolor[rgb]{0.25,0.44,0.63}{{#1}}}
    \newcommand{\CommentTok}[1]{\textcolor[rgb]{0.38,0.63,0.69}{\textit{{#1}}}}
    \newcommand{\OtherTok}[1]{\textcolor[rgb]{0.00,0.44,0.13}{{#1}}}
    \newcommand{\AlertTok}[1]{\textcolor[rgb]{1.00,0.00,0.00}{\textbf{{#1}}}}
    \newcommand{\FunctionTok}[1]{\textcolor[rgb]{0.02,0.16,0.49}{{#1}}}
    \newcommand{\RegionMarkerTok}[1]{{#1}}
    \newcommand{\ErrorTok}[1]{\textcolor[rgb]{1.00,0.00,0.00}{\textbf{{#1}}}}
    \newcommand{\NormalTok}[1]{{#1}}
    
    % Additional commands for more recent versions of Pandoc
    \newcommand{\ConstantTok}[1]{\textcolor[rgb]{0.53,0.00,0.00}{{#1}}}
    \newcommand{\SpecialCharTok}[1]{\textcolor[rgb]{0.25,0.44,0.63}{{#1}}}
    \newcommand{\VerbatimStringTok}[1]{\textcolor[rgb]{0.25,0.44,0.63}{{#1}}}
    \newcommand{\SpecialStringTok}[1]{\textcolor[rgb]{0.73,0.40,0.53}{{#1}}}
    \newcommand{\ImportTok}[1]{{#1}}
    \newcommand{\DocumentationTok}[1]{\textcolor[rgb]{0.73,0.13,0.13}{\textit{{#1}}}}
    \newcommand{\AnnotationTok}[1]{\textcolor[rgb]{0.38,0.63,0.69}{\textbf{\textit{{#1}}}}}
    \newcommand{\CommentVarTok}[1]{\textcolor[rgb]{0.38,0.63,0.69}{\textbf{\textit{{#1}}}}}
    \newcommand{\VariableTok}[1]{\textcolor[rgb]{0.10,0.09,0.49}{{#1}}}
    \newcommand{\ControlFlowTok}[1]{\textcolor[rgb]{0.00,0.44,0.13}{\textbf{{#1}}}}
    \newcommand{\OperatorTok}[1]{\textcolor[rgb]{0.40,0.40,0.40}{{#1}}}
    \newcommand{\BuiltInTok}[1]{{#1}}
    \newcommand{\ExtensionTok}[1]{{#1}}
    \newcommand{\PreprocessorTok}[1]{\textcolor[rgb]{0.74,0.48,0.00}{{#1}}}
    \newcommand{\AttributeTok}[1]{\textcolor[rgb]{0.49,0.56,0.16}{{#1}}}
    \newcommand{\InformationTok}[1]{\textcolor[rgb]{0.38,0.63,0.69}{\textbf{\textit{{#1}}}}}
    \newcommand{\WarningTok}[1]{\textcolor[rgb]{0.38,0.63,0.69}{\textbf{\textit{{#1}}}}}
    
    
    % Define a nice break command that doesn't care if a line doesn't already
    % exist.
    \def\br{\hspace*{\fill} \\* }
    % Math Jax compatability definitions
    \def\gt{>}
    \def\lt{<}
    % Document parameters
    \title{GSR-HR}
    
    
    

    % Pygments definitions
    
\makeatletter
\def\PY@reset{\let\PY@it=\relax \let\PY@bf=\relax%
    \let\PY@ul=\relax \let\PY@tc=\relax%
    \let\PY@bc=\relax \let\PY@ff=\relax}
\def\PY@tok#1{\csname PY@tok@#1\endcsname}
\def\PY@toks#1+{\ifx\relax#1\empty\else%
    \PY@tok{#1}\expandafter\PY@toks\fi}
\def\PY@do#1{\PY@bc{\PY@tc{\PY@ul{%
    \PY@it{\PY@bf{\PY@ff{#1}}}}}}}
\def\PY#1#2{\PY@reset\PY@toks#1+\relax+\PY@do{#2}}

\expandafter\def\csname PY@tok@gd\endcsname{\def\PY@tc##1{\textcolor[rgb]{0.63,0.00,0.00}{##1}}}
\expandafter\def\csname PY@tok@gu\endcsname{\let\PY@bf=\textbf\def\PY@tc##1{\textcolor[rgb]{0.50,0.00,0.50}{##1}}}
\expandafter\def\csname PY@tok@gt\endcsname{\def\PY@tc##1{\textcolor[rgb]{0.00,0.27,0.87}{##1}}}
\expandafter\def\csname PY@tok@gs\endcsname{\let\PY@bf=\textbf}
\expandafter\def\csname PY@tok@gr\endcsname{\def\PY@tc##1{\textcolor[rgb]{1.00,0.00,0.00}{##1}}}
\expandafter\def\csname PY@tok@cm\endcsname{\let\PY@it=\textit\def\PY@tc##1{\textcolor[rgb]{0.25,0.50,0.50}{##1}}}
\expandafter\def\csname PY@tok@vg\endcsname{\def\PY@tc##1{\textcolor[rgb]{0.10,0.09,0.49}{##1}}}
\expandafter\def\csname PY@tok@vi\endcsname{\def\PY@tc##1{\textcolor[rgb]{0.10,0.09,0.49}{##1}}}
\expandafter\def\csname PY@tok@vm\endcsname{\def\PY@tc##1{\textcolor[rgb]{0.10,0.09,0.49}{##1}}}
\expandafter\def\csname PY@tok@mh\endcsname{\def\PY@tc##1{\textcolor[rgb]{0.40,0.40,0.40}{##1}}}
\expandafter\def\csname PY@tok@cs\endcsname{\let\PY@it=\textit\def\PY@tc##1{\textcolor[rgb]{0.25,0.50,0.50}{##1}}}
\expandafter\def\csname PY@tok@ge\endcsname{\let\PY@it=\textit}
\expandafter\def\csname PY@tok@vc\endcsname{\def\PY@tc##1{\textcolor[rgb]{0.10,0.09,0.49}{##1}}}
\expandafter\def\csname PY@tok@il\endcsname{\def\PY@tc##1{\textcolor[rgb]{0.40,0.40,0.40}{##1}}}
\expandafter\def\csname PY@tok@go\endcsname{\def\PY@tc##1{\textcolor[rgb]{0.53,0.53,0.53}{##1}}}
\expandafter\def\csname PY@tok@cp\endcsname{\def\PY@tc##1{\textcolor[rgb]{0.74,0.48,0.00}{##1}}}
\expandafter\def\csname PY@tok@gi\endcsname{\def\PY@tc##1{\textcolor[rgb]{0.00,0.63,0.00}{##1}}}
\expandafter\def\csname PY@tok@gh\endcsname{\let\PY@bf=\textbf\def\PY@tc##1{\textcolor[rgb]{0.00,0.00,0.50}{##1}}}
\expandafter\def\csname PY@tok@ni\endcsname{\let\PY@bf=\textbf\def\PY@tc##1{\textcolor[rgb]{0.60,0.60,0.60}{##1}}}
\expandafter\def\csname PY@tok@nl\endcsname{\def\PY@tc##1{\textcolor[rgb]{0.63,0.63,0.00}{##1}}}
\expandafter\def\csname PY@tok@nn\endcsname{\let\PY@bf=\textbf\def\PY@tc##1{\textcolor[rgb]{0.00,0.00,1.00}{##1}}}
\expandafter\def\csname PY@tok@no\endcsname{\def\PY@tc##1{\textcolor[rgb]{0.53,0.00,0.00}{##1}}}
\expandafter\def\csname PY@tok@na\endcsname{\def\PY@tc##1{\textcolor[rgb]{0.49,0.56,0.16}{##1}}}
\expandafter\def\csname PY@tok@nb\endcsname{\def\PY@tc##1{\textcolor[rgb]{0.00,0.50,0.00}{##1}}}
\expandafter\def\csname PY@tok@nc\endcsname{\let\PY@bf=\textbf\def\PY@tc##1{\textcolor[rgb]{0.00,0.00,1.00}{##1}}}
\expandafter\def\csname PY@tok@nd\endcsname{\def\PY@tc##1{\textcolor[rgb]{0.67,0.13,1.00}{##1}}}
\expandafter\def\csname PY@tok@ne\endcsname{\let\PY@bf=\textbf\def\PY@tc##1{\textcolor[rgb]{0.82,0.25,0.23}{##1}}}
\expandafter\def\csname PY@tok@nf\endcsname{\def\PY@tc##1{\textcolor[rgb]{0.00,0.00,1.00}{##1}}}
\expandafter\def\csname PY@tok@si\endcsname{\let\PY@bf=\textbf\def\PY@tc##1{\textcolor[rgb]{0.73,0.40,0.53}{##1}}}
\expandafter\def\csname PY@tok@s2\endcsname{\def\PY@tc##1{\textcolor[rgb]{0.73,0.13,0.13}{##1}}}
\expandafter\def\csname PY@tok@nt\endcsname{\let\PY@bf=\textbf\def\PY@tc##1{\textcolor[rgb]{0.00,0.50,0.00}{##1}}}
\expandafter\def\csname PY@tok@nv\endcsname{\def\PY@tc##1{\textcolor[rgb]{0.10,0.09,0.49}{##1}}}
\expandafter\def\csname PY@tok@s1\endcsname{\def\PY@tc##1{\textcolor[rgb]{0.73,0.13,0.13}{##1}}}
\expandafter\def\csname PY@tok@dl\endcsname{\def\PY@tc##1{\textcolor[rgb]{0.73,0.13,0.13}{##1}}}
\expandafter\def\csname PY@tok@ch\endcsname{\let\PY@it=\textit\def\PY@tc##1{\textcolor[rgb]{0.25,0.50,0.50}{##1}}}
\expandafter\def\csname PY@tok@m\endcsname{\def\PY@tc##1{\textcolor[rgb]{0.40,0.40,0.40}{##1}}}
\expandafter\def\csname PY@tok@gp\endcsname{\let\PY@bf=\textbf\def\PY@tc##1{\textcolor[rgb]{0.00,0.00,0.50}{##1}}}
\expandafter\def\csname PY@tok@sh\endcsname{\def\PY@tc##1{\textcolor[rgb]{0.73,0.13,0.13}{##1}}}
\expandafter\def\csname PY@tok@ow\endcsname{\let\PY@bf=\textbf\def\PY@tc##1{\textcolor[rgb]{0.67,0.13,1.00}{##1}}}
\expandafter\def\csname PY@tok@sx\endcsname{\def\PY@tc##1{\textcolor[rgb]{0.00,0.50,0.00}{##1}}}
\expandafter\def\csname PY@tok@bp\endcsname{\def\PY@tc##1{\textcolor[rgb]{0.00,0.50,0.00}{##1}}}
\expandafter\def\csname PY@tok@c1\endcsname{\let\PY@it=\textit\def\PY@tc##1{\textcolor[rgb]{0.25,0.50,0.50}{##1}}}
\expandafter\def\csname PY@tok@fm\endcsname{\def\PY@tc##1{\textcolor[rgb]{0.00,0.00,1.00}{##1}}}
\expandafter\def\csname PY@tok@o\endcsname{\def\PY@tc##1{\textcolor[rgb]{0.40,0.40,0.40}{##1}}}
\expandafter\def\csname PY@tok@kc\endcsname{\let\PY@bf=\textbf\def\PY@tc##1{\textcolor[rgb]{0.00,0.50,0.00}{##1}}}
\expandafter\def\csname PY@tok@c\endcsname{\let\PY@it=\textit\def\PY@tc##1{\textcolor[rgb]{0.25,0.50,0.50}{##1}}}
\expandafter\def\csname PY@tok@mf\endcsname{\def\PY@tc##1{\textcolor[rgb]{0.40,0.40,0.40}{##1}}}
\expandafter\def\csname PY@tok@err\endcsname{\def\PY@bc##1{\setlength{\fboxsep}{0pt}\fcolorbox[rgb]{1.00,0.00,0.00}{1,1,1}{\strut ##1}}}
\expandafter\def\csname PY@tok@mb\endcsname{\def\PY@tc##1{\textcolor[rgb]{0.40,0.40,0.40}{##1}}}
\expandafter\def\csname PY@tok@ss\endcsname{\def\PY@tc##1{\textcolor[rgb]{0.10,0.09,0.49}{##1}}}
\expandafter\def\csname PY@tok@sr\endcsname{\def\PY@tc##1{\textcolor[rgb]{0.73,0.40,0.53}{##1}}}
\expandafter\def\csname PY@tok@mo\endcsname{\def\PY@tc##1{\textcolor[rgb]{0.40,0.40,0.40}{##1}}}
\expandafter\def\csname PY@tok@kd\endcsname{\let\PY@bf=\textbf\def\PY@tc##1{\textcolor[rgb]{0.00,0.50,0.00}{##1}}}
\expandafter\def\csname PY@tok@mi\endcsname{\def\PY@tc##1{\textcolor[rgb]{0.40,0.40,0.40}{##1}}}
\expandafter\def\csname PY@tok@kn\endcsname{\let\PY@bf=\textbf\def\PY@tc##1{\textcolor[rgb]{0.00,0.50,0.00}{##1}}}
\expandafter\def\csname PY@tok@cpf\endcsname{\let\PY@it=\textit\def\PY@tc##1{\textcolor[rgb]{0.25,0.50,0.50}{##1}}}
\expandafter\def\csname PY@tok@kr\endcsname{\let\PY@bf=\textbf\def\PY@tc##1{\textcolor[rgb]{0.00,0.50,0.00}{##1}}}
\expandafter\def\csname PY@tok@s\endcsname{\def\PY@tc##1{\textcolor[rgb]{0.73,0.13,0.13}{##1}}}
\expandafter\def\csname PY@tok@kp\endcsname{\def\PY@tc##1{\textcolor[rgb]{0.00,0.50,0.00}{##1}}}
\expandafter\def\csname PY@tok@w\endcsname{\def\PY@tc##1{\textcolor[rgb]{0.73,0.73,0.73}{##1}}}
\expandafter\def\csname PY@tok@kt\endcsname{\def\PY@tc##1{\textcolor[rgb]{0.69,0.00,0.25}{##1}}}
\expandafter\def\csname PY@tok@sc\endcsname{\def\PY@tc##1{\textcolor[rgb]{0.73,0.13,0.13}{##1}}}
\expandafter\def\csname PY@tok@sb\endcsname{\def\PY@tc##1{\textcolor[rgb]{0.73,0.13,0.13}{##1}}}
\expandafter\def\csname PY@tok@sa\endcsname{\def\PY@tc##1{\textcolor[rgb]{0.73,0.13,0.13}{##1}}}
\expandafter\def\csname PY@tok@k\endcsname{\let\PY@bf=\textbf\def\PY@tc##1{\textcolor[rgb]{0.00,0.50,0.00}{##1}}}
\expandafter\def\csname PY@tok@se\endcsname{\let\PY@bf=\textbf\def\PY@tc##1{\textcolor[rgb]{0.73,0.40,0.13}{##1}}}
\expandafter\def\csname PY@tok@sd\endcsname{\let\PY@it=\textit\def\PY@tc##1{\textcolor[rgb]{0.73,0.13,0.13}{##1}}}

\def\PYZbs{\char`\\}
\def\PYZus{\char`\_}
\def\PYZob{\char`\{}
\def\PYZcb{\char`\}}
\def\PYZca{\char`\^}
\def\PYZam{\char`\&}
\def\PYZlt{\char`\<}
\def\PYZgt{\char`\>}
\def\PYZsh{\char`\#}
\def\PYZpc{\char`\%}
\def\PYZdl{\char`\$}
\def\PYZhy{\char`\-}
\def\PYZsq{\char`\'}
\def\PYZdq{\char`\"}
\def\PYZti{\char`\~}
% for compatibility with earlier versions
\def\PYZat{@}
\def\PYZlb{[}
\def\PYZrb{]}
\makeatother


    % Exact colors from NB
    \definecolor{incolor}{rgb}{0.0, 0.0, 0.5}
    \definecolor{outcolor}{rgb}{0.545, 0.0, 0.0}



    
    % Prevent overflowing lines due to hard-to-break entities
    \sloppy 
    % Setup hyperref package
    \hypersetup{
      breaklinks=true,  % so long urls are correctly broken across lines
      colorlinks=true,
      urlcolor=urlcolor,
      linkcolor=linkcolor,
      citecolor=citecolor,
      }
    % Slightly bigger margins than the latex defaults
    
    \geometry{verbose,tmargin=1in,bmargin=1in,lmargin=1in,rmargin=1in}
    
    

    \begin{document}
    
    
    \maketitle
    
    

    
    \section{Proyecto I : Regresión
Lineal}\label{proyecto-i-regresiuxf3n-lineal}

    Integrantes: - Aldair Bernal Betancur - Juan José Vera Arango

    En este laboratorio vamos a analizar dos tipos de señales. GSR que es
una señal tomada la cual representa la respuesta galvanica de la piel
medida en microvoltios durante 1 segundo y otra señal llamada HR la cual
representa el Pulso Cardiaco.

A continuación, importamos las librerias necesarias, tales como
matplitlib con la función pyplot para graficar las señales y visualizar
los resultados, numpy para el manejo de las matrices y calculos de
estas, math para operaciones matematicas ya definidas y scipy para leer
los datos previamente conseguidos.

    \begin{Verbatim}[commandchars=\\\{\}]
{\color{incolor}In [{\color{incolor}1}]:} \PY{k+kn}{import} \PY{n+nn}{matplotlib.pyplot} \PY{k+kn}{as} \PY{n+nn}{plt} \PY{c+c1}{\PYZsh{} Libreria para graficar y visualizar resultados}
        \PY{k+kn}{import} \PY{n+nn}{numpy} \PY{k+kn}{as} \PY{n+nn}{np} \PY{c+c1}{\PYZsh{} libreria de manejo de datos matriciales y operaciones multivariadas}
        \PY{k+kn}{import} \PY{n+nn}{math} \PY{c+c1}{\PYZsh{} Libreria de opreaciones matematicas}
        \PY{k+kn}{import} \PY{n+nn}{scipy.io} \PY{k+kn}{as} \PY{n+nn}{sio} \PY{c+c1}{\PYZsh{} Libreria para cargar o escribir datos (Se utilizara para cargar nuestros DATASETS)}
        \PY{o}{\PYZpc{}}\PY{k}{matplotlib} inline 
        
        \PY{n}{data} \PY{o}{=} \PY{n}{sio}\PY{o}{.}\PY{n}{loadmat}\PY{p}{(}\PY{l+s+s1}{\PYZsq{}}\PY{l+s+s1}{Biosignals.mat}\PY{l+s+s1}{\PYZsq{}}\PY{p}{)}
        \PY{c+c1}{\PYZsh{} los datos pueden ser verificados con la funcion print datos a mostrar}
        \PY{k}{print} \PY{n}{data}
\end{Verbatim}


    \begin{Verbatim}[commandchars=\\\{\}]
\{'GSR': array([[-0.00660929],
       [-0.00660929],
       [-0.00660929],
       {\ldots},
       [-0.0526087 ],
       [-0.0526087 ],
       [-0.0526087 ]]), 'HR': array([[ 764.43494819],
       [ 636.02893546],
       [1071.12188151],
       {\ldots},
       [ 607.99773725],
       [ 477.8104778 ],
       [ 675.59136235]]), '\_\_header\_\_': 'MATLAB 5.0 MAT-file, Platform: PCWIN64, Created on: Wed Feb 28 11:20:27 2018', '\_\_globals\_\_': [], 'x': array([[0.00000000e+00, 1.53209744e-04, 3.06419488e-04, {\ldots},
        9.99693581e-01, 9.99846790e-01, 1.00000000e+00]]), '\_\_version\_\_': '1.0'\}

    \end{Verbatim}

    Creamos las variables con las que vamos a desarrollar el laboratorio. N
representa el total de datos que tenemos en el DATASET.

    \begin{Verbatim}[commandchars=\\\{\}]
{\color{incolor}In [{\color{incolor}2}]:} \PY{n}{GSR} \PY{o}{=} \PY{n}{data}\PY{p}{[}\PY{l+s+s1}{\PYZsq{}}\PY{l+s+s1}{GSR}\PY{l+s+s1}{\PYZsq{}}\PY{p}{]} \PY{c+c1}{\PYZsh{}Se accede al campo GSR la cual son los valores de la Respuesta Galvanica de la Piel}
        \PY{n}{HR} \PY{o}{=} \PY{n}{data}\PY{p}{[}\PY{l+s+s1}{\PYZsq{}}\PY{l+s+s1}{HR}\PY{l+s+s1}{\PYZsq{}}\PY{p}{]} \PY{c+c1}{\PYZsh{}Se accede al campo GSR la cual son los valores sobre el Ritmo Cardiaco}
        \PY{n}{x} \PY{o}{=} \PY{n}{data}\PY{p}{[}\PY{l+s+s1}{\PYZsq{}}\PY{l+s+s1}{x}\PY{l+s+s1}{\PYZsq{}}\PY{p}{]} \PY{c+c1}{\PYZsh{}Se accede al campo x donde se encuentran las características del conjunto de entrenamiento}
        \PY{n}{x} \PY{o}{=} \PY{n}{x}\PY{o}{.}\PY{n}{T}
        \PY{n}{N}\PY{p}{,} \PY{n}{D} \PY{o}{=} \PY{n}{x}\PY{o}{.}\PY{n}{shape}
\end{Verbatim}


    \subsection{Analizando GSR}\label{analizando-gsr}

    Comenzaremos analizando la señal GSR, partimos la señal en dos partes,
una que será para entrenar nuestro modelo y la otra parte para
verificarlo, estas partes se dividiran en 70\% para entrenamiento y 30\%
para test.

\begin{itemize}
\tightlist
\item
  xTr = Representa el conjunto de las caracteristicas de entrenamiento
\item
  xTe = Representa el conjunto de las caracteristicas de testing
\item
  GSRTr = Representa los datos recogidos de la respuesta galvanica de la
  piel para el entrenamiento
\item
  GSRTe = Representa los datos recogidos de la respuesta galvanica de la
  piel para el testing
\end{itemize}

Graficamos los datos de entramiento y los de testing.

    \begin{Verbatim}[commandchars=\\\{\}]
{\color{incolor}In [{\color{incolor}3}]:} \PY{n+nb}{id} \PY{o}{=} \PY{n}{np}\PY{o}{.}\PY{n}{random}\PY{o}{.}\PY{n}{permutation}\PY{p}{(}\PY{n}{N}\PY{p}{)}
        \PY{n}{perTrain} \PY{o}{=} \PY{l+m+mf}{0.7}
        \PY{n}{NTr} \PY{o}{=} \PY{n+nb}{int}\PY{p}{(}\PY{n+nb}{round}\PY{p}{(}\PY{n}{N}\PY{o}{*}\PY{n}{perTrain}\PY{p}{)}\PY{p}{)}
        \PY{n}{idTr} \PY{o}{=} \PY{n+nb}{id}\PY{p}{[}\PY{p}{:}\PY{n}{NTr}\PY{p}{]}
        \PY{n}{idTe} \PY{o}{=} \PY{n+nb}{id}\PY{p}{[}\PY{n}{NTr}\PY{p}{:}\PY{p}{]}
        \PY{n}{xTr} \PY{o}{=} \PY{n}{x}\PY{p}{[}\PY{n}{idTr}\PY{p}{]}
        \PY{n}{xTe} \PY{o}{=} \PY{n}{x}\PY{p}{[}\PY{n}{idTe}\PY{p}{]}
        \PY{n}{GSRTr} \PY{o}{=} \PY{n}{GSR}\PY{p}{[}\PY{n}{idTr}\PY{p}{]}
        \PY{n}{GSRTe} \PY{o}{=} \PY{n}{GSR}\PY{p}{[}\PY{n}{idTe}\PY{p}{]}
        
        \PY{n}{plt}\PY{o}{.}\PY{n}{plot}\PY{p}{(}\PY{n}{xTr}\PY{p}{,} \PY{n}{GSRTr}\PY{p}{,} \PY{l+s+s1}{\PYZsq{}}\PY{l+s+s1}{or}\PY{l+s+s1}{\PYZsq{}}\PY{p}{)}
        \PY{n}{plt}\PY{o}{.}\PY{n}{xlabel}\PY{p}{(}\PY{l+s+s1}{\PYZsq{}}\PY{l+s+s1}{x}\PY{l+s+s1}{\PYZsq{}}\PY{p}{)}
        \PY{n}{GSRTr}\PY{o}{.}\PY{n}{shape}
\end{Verbatim}


\begin{Verbatim}[commandchars=\\\{\}]
{\color{outcolor}Out[{\color{outcolor}3}]:} (4570L, 1L)
\end{Verbatim}
            
    \begin{center}
    \adjustimage{max size={0.9\linewidth}{0.9\paperheight}}{output_8_1.png}
    \end{center}
    { \hspace*{\fill} \\}
    
    \begin{Verbatim}[commandchars=\\\{\}]
{\color{incolor}In [{\color{incolor}4}]:} \PY{n}{plt}\PY{o}{.}\PY{n}{plot}\PY{p}{(}\PY{n}{xTe}\PY{p}{,} \PY{n}{GSRTe}\PY{p}{,} \PY{l+s+s1}{\PYZsq{}}\PY{l+s+s1}{og}\PY{l+s+s1}{\PYZsq{}}\PY{p}{)}
        \PY{n}{plt}\PY{o}{.}\PY{n}{xlabel}\PY{p}{(}\PY{l+s+s1}{\PYZsq{}}\PY{l+s+s1}{x}\PY{l+s+s1}{\PYZsq{}}\PY{p}{)}
        \PY{n}{GSRTe}\PY{o}{.}\PY{n}{shape}
\end{Verbatim}


\begin{Verbatim}[commandchars=\\\{\}]
{\color{outcolor}Out[{\color{outcolor}4}]:} (1958L, 1L)
\end{Verbatim}
            
    \begin{center}
    \adjustimage{max size={0.9\linewidth}{0.9\paperheight}}{output_9_1.png}
    \end{center}
    { \hspace*{\fill} \\}
    
    \subsection{Funcion de regresión lineal y
error}\label{funcion-de-regresiuxf3n-lineal-y-error}

    Luego de tener nuestros datos debidamente almacenados en sus respectivas
variables debemos recordar que para una tarea de machine learning,
nuestro conjunto de datos serán:

\(\bullet\) Observaciones para
\(\bf{X}=[\bf{x}_1,\bf{x}_2,\cdots, \bf{x}_N]\in \mathbb{R}^{N\times D}\),
donde cada vector observado es de dimensionaledad
\(\bf{x}\in \mathbb{R}^{D\times 1}\)

\(\bullet\) Etiquetas (flotantes) para
\(\bf{t}\in \mathbb{R}^{N\times 1}\)

    \begin{Verbatim}[commandchars=\\\{\}]
{\color{incolor}In [{\color{incolor}5}]:} \PY{k}{def} \PY{n+nf}{LS}\PY{p}{(}\PY{n}{X}\PY{p}{,}\PY{n}{t}\PY{p}{,}\PY{n}{basisFNC}\PY{p}{,}\PY{n}{NbF}\PY{p}{)}\PY{p}{:}
            \PY{n}{Ndata}\PY{p}{,}\PY{n}{D} \PY{o}{=} \PY{n}{X}\PY{o}{.}\PY{n}{shape}
            \PY{c+c1}{\PYZsh{}print Ndata,D}
            \PY{n}{yEst} \PY{o}{=} \PY{n}{np}\PY{o}{.}\PY{n}{zeros}\PY{p}{(}\PY{p}{(}\PY{n}{Ndata}\PY{p}{,}\PY{l+m+mi}{1}\PY{p}{)}\PY{p}{)}
            \PY{c+c1}{\PYZsh{} Calculo de la matriz PHI de funciones base}
            \PY{n}{PHI} \PY{o}{=} \PY{n}{np}\PY{o}{.}\PY{n}{zeros}\PY{p}{(}\PY{p}{(}\PY{n}{Ndata}\PY{p}{,}\PY{n}{NbF}\PY{o}{+}\PY{l+m+mi}{1}\PY{p}{)}\PY{p}{)}
            \PY{n}{PHI}\PY{p}{[}\PY{p}{:}\PY{p}{,}\PY{l+m+mi}{0}\PY{p}{]} \PY{o}{=} \PY{l+m+mi}{1}
            \PY{n}{mu} \PY{o}{=} \PY{n}{np}\PY{o}{.}\PY{n}{linspace}\PY{p}{(}\PY{l+m+mf}{0.0}\PY{p}{,} \PY{l+m+mf}{1.0}\PY{p}{,} \PY{n}{NbF}\PY{o}{+}\PY{l+m+mi}{1}\PY{p}{)}
            \PY{n}{s2} \PY{o}{=} \PY{n}{np}\PY{o}{.}\PY{n}{var}\PY{p}{(}\PY{n}{mu}\PY{p}{)}
            \PY{k}{for} \PY{n}{n} \PY{o+ow}{in} \PY{n+nb}{range}\PY{p}{(}\PY{l+m+mi}{0}\PY{p}{,}\PY{n}{Ndata}\PY{p}{)}\PY{p}{:}
                \PY{c+c1}{\PYZsh{}print X[n]}
                \PY{k}{for} \PY{n}{i} \PY{o+ow}{in} \PY{n+nb}{range}\PY{p}{(}\PY{l+m+mi}{1}\PY{p}{,}\PY{n}{NbF}\PY{o}{+}\PY{l+m+mi}{1}\PY{p}{)}\PY{p}{:}
                    \PY{k}{if} \PY{n}{basisFNC} \PY{o}{==} \PY{l+s+s1}{\PYZsq{}}\PY{l+s+s1}{pol}\PY{l+s+s1}{\PYZsq{}}\PY{p}{:}                
                        \PY{n}{PHI}\PY{p}{[}\PY{n}{n}\PY{p}{]}\PY{p}{[}\PY{n}{i}\PY{p}{]} \PY{o}{=} \PY{n}{X}\PY{p}{[}\PY{n}{n}\PY{p}{]}\PY{o}{*}\PY{o}{*}\PY{p}{(}\PY{n}{i}\PY{p}{)}
                    \PY{k}{if} \PY{n}{basisFNC} \PY{o}{==} \PY{l+s+s1}{\PYZsq{}}\PY{l+s+s1}{exp}\PY{l+s+s1}{\PYZsq{}}\PY{p}{:}
                        \PY{n}{PHI}\PY{p}{[}\PY{n}{n}\PY{p}{]}\PY{p}{[}\PY{n}{i}\PY{p}{]} \PY{o}{=} \PY{n}{math}\PY{o}{.}\PY{n}{exp}\PY{p}{(}\PY{o}{\PYZhy{}}\PY{p}{(}\PY{p}{(}\PY{n}{X}\PY{p}{[}\PY{n}{n}\PY{p}{]}\PY{o}{\PYZhy{}}\PY{n}{mu}\PY{p}{[}\PY{n}{i}\PY{p}{]}\PY{p}{)}\PY{o}{*}\PY{o}{*}\PY{l+m+mi}{2}\PY{p}{)}\PY{o}{/}\PY{p}{(}\PY{l+m+mi}{2}\PY{o}{*}\PY{n}{s2}\PY{p}{)}\PY{p}{)}
                    \PY{k}{if} \PY{n}{basisFNC} \PY{o}{==} \PY{l+s+s1}{\PYZsq{}}\PY{l+s+s1}{sig}\PY{l+s+s1}{\PYZsq{}}\PY{p}{:}
                        \PY{n}{PHI}\PY{p}{[}\PY{n}{n}\PY{p}{]}\PY{p}{[}\PY{n}{i}\PY{p}{]} \PY{o}{=} \PY{p}{(}\PY{l+m+mi}{1} \PY{o}{/} \PY{p}{(}\PY{l+m+mi}{1} \PY{o}{+} \PY{n}{np}\PY{o}{.}\PY{n}{exp}\PY{p}{(}\PY{o}{\PYZhy{}}\PY{p}{(}\PY{p}{(}\PY{n}{X}\PY{p}{[}\PY{n}{n}\PY{p}{]}\PY{o}{\PYZhy{}}\PY{n}{mu}\PY{p}{[}\PY{n}{i}\PY{p}{]}\PY{p}{)}\PY{o}{/}\PY{n}{math}\PY{o}{.}\PY{n}{sqrt}\PY{p}{(}\PY{n}{s2}\PY{p}{)}\PY{p}{)}\PY{p}{)}\PY{p}{)}\PY{p}{)}
            
            \PY{c+c1}{\PYZsh{} Luego se estima el mejor W que maximiza la verosimilitud utilizando minimos cuadrados}
            \PY{n}{PHIT} \PY{o}{=} \PY{n}{PHI}\PY{o}{.}\PY{n}{T}
            \PY{n}{w\PYZus{}ml} \PY{o}{=} \PY{n}{np}\PY{o}{.}\PY{n}{linalg}\PY{o}{.}\PY{n}{inv}\PY{p}{(}\PY{n}{PHIT}\PY{o}{.}\PY{n}{dot}\PY{p}{(}\PY{n}{PHI}\PY{p}{)}\PY{p}{)}\PY{o}{.}\PY{n}{dot}\PY{p}{(}\PY{n}{PHIT}\PY{o}{.}\PY{n}{dot}\PY{p}{(}\PY{n}{t}\PY{p}{)}\PY{p}{)}
            \PY{n}{yEst} \PY{o}{=} \PY{n}{PHI}\PY{o}{.}\PY{n}{dot}\PY{p}{(}\PY{n}{w\PYZus{}ml}\PY{p}{)}
            \PY{c+c1}{\PYZsh{}print w\PYZus{}ml}
            \PY{k}{return} \PY{n}{PHI}\PY{p}{,}\PY{n}{w\PYZus{}ml}\PY{p}{,}\PY{n}{yEst}\PY{p}{,} \PY{n}{s2}  
        
        \PY{k}{def} \PY{n+nf}{Erms}\PY{p}{(}\PY{n}{to}\PY{p}{,} \PY{n}{te}\PY{p}{)}\PY{p}{:}
            \PY{n}{N}\PY{p}{,} \PY{n}{d} \PY{o}{=} \PY{n}{to}\PY{o}{.}\PY{n}{shape}
            \PY{n}{eRMS} \PY{o}{=} \PY{n}{np}\PY{o}{.}\PY{n}{sqrt}\PY{p}{(}\PY{p}{(}\PY{l+m+mf}{1.0}\PY{o}{/}\PY{n}{N}\PY{p}{)}\PY{o}{*}\PY{n}{np}\PY{o}{.}\PY{n}{sum}\PY{p}{(}\PY{p}{(}\PY{n}{to}\PY{o}{\PYZhy{}}\PY{n}{te}\PY{p}{)}\PY{o}{*}\PY{o}{*}\PY{l+m+mi}{2}\PY{p}{)}\PY{p}{)}
            \PY{k}{return} \PY{n}{eRMS}
\end{Verbatim}


    Este módulo realiza la tarea de ajusta un conjunto de observaciones
\(\bf X\) a unas etiquetas \(\bf t\) a partir de la selección del mapeo
de una cantidad \(M\) de funciones base que pueden ser de tipo
polinomial, exponencial y sigmoidal. Estas funciones base permiten
llevar nuestras observaciones a un espacio de representación mas
relevante en el cual el ajuste de los datos presenta un mejor desempeño.

El módulo denominado \(\bf LS\) se ejecuta de mediante el siguiente
comando

\(W\_ML, EtiquetaEstimada\) \(=\)
\(\bf LS\)\((observaciones,etiquetas,tipoFCNbase,M)\)

Dicha función recibe cómo argumentos de entrada:

\(\bullet\) observaciones: matriz de dimensión
\(\mathbb{R}^{N\times D}\) que contiene los patrones a modelar

\(\bullet\) etiquetas: vector de dimensión \(\mathbb{R}^{N\times 1}\)
que contiene las etiquetas correspondientes a cada dato

\(\bullet\) tipoFCNbase: dato tipo cadena para identificar el tipo de
funcion base a utilizar. Esta cadena puede ser:

\(-\) 'pol': para una funcion de base polinomial

\(-\) 'exp': para una funcion de base exponencial

\(-\) 'sig': para una funcion de base sigmoidal

Dicha función tiene cómo argumentos de salida:

\(\bullet\) \(W\_ML\): vector de coeficientes que realiza la estimación
de los datos de la forma
\(\hat{\bf{y}} = \bf{w}_{ML}^\top \phi(\bf{x}_n)\)

\(\bullet\) EtiquetaEstimada: vector de tamaño \$\mathbb{R}\^{}\{N
\times 1\} \$ que contiene las etiquetas estimadas para los respectivos
\(\bf X\)

    \subsubsection{Regresion con funciones bases
polinomial}\label{regresion-con-funciones-bases-polinomial}

    Hacemos la regresion lineal con bases polinomial, despues de un proceso
empirico de buscar el mejor numero de funciones bases, se establece que
con 10 es el que mas se acerca a la funcion GSR. Se calcula el Error del
Train y del Test y se puede ver que con tiene un muy bajo porcentaje de
error

    \begin{Verbatim}[commandchars=\\\{\}]
{\color{incolor}In [{\color{incolor}6}]:} \PY{n}{PHIbTr}\PY{p}{,}\PY{n}{w\PYZus{}ML}\PY{p}{,}\PY{n}{yEstimadoTr}\PY{p}{,} \PY{n}{s2} \PY{o}{=} \PY{n}{LS}\PY{p}{(}\PY{n}{xTr}\PY{p}{,}\PY{n}{GSRTr}\PY{p}{,}\PY{l+s+s1}{\PYZsq{}}\PY{l+s+s1}{pol}\PY{l+s+s1}{\PYZsq{}}\PY{p}{,}\PY{l+m+mi}{10}\PY{p}{)}
        \PY{n}{PHIbTe}\PY{p}{,}\PY{n}{w\PYZus{}MLTe}\PY{p}{,}\PY{n}{yEstimadoTe}\PY{p}{,} \PY{n}{s2} \PY{o}{=} \PY{n}{LS}\PY{p}{(}\PY{n}{xTe}\PY{p}{,}\PY{n}{GSRTe}\PY{p}{,}\PY{l+s+s1}{\PYZsq{}}\PY{l+s+s1}{pol}\PY{l+s+s1}{\PYZsq{}}\PY{p}{,} \PY{l+m+mi}{10}\PY{p}{)}
        \PY{n}{yEstimadoTe} \PY{o}{=} \PY{n}{PHIbTe}\PY{o}{.}\PY{n}{dot}\PY{p}{(}\PY{n}{w\PYZus{}ML}\PY{p}{)}
        \PY{n}{plt}\PY{o}{.}\PY{n}{plot}\PY{p}{(}\PY{n}{xTr}\PY{p}{,}\PY{n}{yEstimadoTr}\PY{p}{,} \PY{l+s+s1}{\PYZsq{}}\PY{l+s+s1}{og}\PY{l+s+s1}{\PYZsq{}}\PY{p}{)}
        \PY{n}{plt}\PY{o}{.}\PY{n}{plot}\PY{p}{(}\PY{n}{xTe}\PY{p}{,}\PY{n}{yEstimadoTe}\PY{p}{,} \PY{l+s+s1}{\PYZsq{}}\PY{l+s+s1}{or}\PY{l+s+s1}{\PYZsq{}}\PY{p}{)}
        \PY{n}{plt}\PY{o}{.}\PY{n}{plot}\PY{p}{(}\PY{n}{x}\PY{p}{,}\PY{n}{GSR}\PY{p}{,} \PY{l+s+s1}{\PYZsq{}}\PY{l+s+s1}{\PYZhy{}k}\PY{l+s+s1}{\PYZsq{}}\PY{p}{)}
        
        \PY{n}{RMStest} \PY{o}{=} \PY{n}{Erms}\PY{p}{(}\PY{n}{GSRTr}\PY{p}{,} \PY{n}{yEstimadoTr}\PY{p}{)}
        \PY{n}{RMStrain} \PY{o}{=} \PY{n}{Erms}\PY{p}{(}\PY{n}{GSRTe}\PY{p}{,} \PY{n}{yEstimadoTe}\PY{p}{)}
        
        \PY{k}{print} \PY{l+s+s2}{\PYZdq{}}\PY{l+s+s2}{Error Test}\PY{l+s+s2}{\PYZdq{}}\PY{p}{,} \PY{n}{RMStest}
        \PY{k}{print} \PY{l+s+s2}{\PYZdq{}}\PY{l+s+s2}{Error Train}\PY{l+s+s2}{\PYZdq{}}\PY{p}{,} \PY{n}{RMStrain}
\end{Verbatim}


    \begin{Verbatim}[commandchars=\\\{\}]
Error Test 0.002192609363493034
Error Train 0.002164658131864438

    \end{Verbatim}

    \begin{center}
    \adjustimage{max size={0.9\linewidth}{0.9\paperheight}}{output_16_1.png}
    \end{center}
    { \hspace*{\fill} \\}
    
    \subsubsection{Regresion con funciones bases
exponencial}\label{regresion-con-funciones-bases-exponencial}

    Ahora analizaremos la función GSR, pero esta vez las funciones bases
seran exponenciales. En esta ocasion la cantidad de funciones bases son
menores pero tiene un aumento muy poco significativo en el error, casi
imperceptible.

    \begin{Verbatim}[commandchars=\\\{\}]
{\color{incolor}In [{\color{incolor}7}]:} \PY{n}{PHIbTr}\PY{p}{,}\PY{n}{w\PYZus{}ML}\PY{p}{,}\PY{n}{yEstimadoTr}\PY{p}{,} \PY{n}{s2} \PY{o}{=} \PY{n}{LS}\PY{p}{(}\PY{n}{xTr}\PY{p}{,}\PY{n}{GSRTr}\PY{p}{,}\PY{l+s+s1}{\PYZsq{}}\PY{l+s+s1}{exp}\PY{l+s+s1}{\PYZsq{}}\PY{p}{,}\PY{l+m+mi}{9}\PY{p}{)}
        \PY{n}{PHIbTe}\PY{p}{,}\PY{n}{w\PYZus{}MLTe}\PY{p}{,}\PY{n}{yEstimadoTe}\PY{p}{,} \PY{n}{s2} \PY{o}{=} \PY{n}{LS}\PY{p}{(}\PY{n}{xTe}\PY{p}{,}\PY{n}{GSRTe}\PY{p}{,}\PY{l+s+s1}{\PYZsq{}}\PY{l+s+s1}{exp}\PY{l+s+s1}{\PYZsq{}}\PY{p}{,}\PY{l+m+mi}{9}\PY{p}{)}
        \PY{n}{yEstimadoTe} \PY{o}{=} \PY{n}{PHIbTe}\PY{o}{.}\PY{n}{dot}\PY{p}{(}\PY{n}{w\PYZus{}ML}\PY{p}{)}
        \PY{n}{plt}\PY{o}{.}\PY{n}{plot}\PY{p}{(}\PY{n}{xTr}\PY{p}{,}\PY{n}{yEstimadoTr}\PY{p}{,} \PY{l+s+s1}{\PYZsq{}}\PY{l+s+s1}{og}\PY{l+s+s1}{\PYZsq{}}\PY{p}{)}
        \PY{n}{plt}\PY{o}{.}\PY{n}{plot}\PY{p}{(}\PY{n}{xTe}\PY{p}{,}\PY{n}{yEstimadoTe}\PY{p}{,} \PY{l+s+s1}{\PYZsq{}}\PY{l+s+s1}{or}\PY{l+s+s1}{\PYZsq{}}\PY{p}{)}
        \PY{n}{plt}\PY{o}{.}\PY{n}{plot}\PY{p}{(}\PY{n}{x}\PY{p}{,}\PY{n}{GSR}\PY{p}{,} \PY{l+s+s1}{\PYZsq{}}\PY{l+s+s1}{\PYZhy{}k}\PY{l+s+s1}{\PYZsq{}}\PY{p}{)}
        
        \PY{n}{RMStest} \PY{o}{=} \PY{n}{Erms}\PY{p}{(}\PY{n}{GSRTr}\PY{p}{,} \PY{n}{yEstimadoTr}\PY{p}{)}
        \PY{n}{RMStrain} \PY{o}{=} \PY{n}{Erms}\PY{p}{(}\PY{n}{GSRTe}\PY{p}{,} \PY{n}{yEstimadoTe}\PY{p}{)}
        
        \PY{k}{print} \PY{l+s+s2}{\PYZdq{}}\PY{l+s+s2}{Error Test}\PY{l+s+s2}{\PYZdq{}}\PY{p}{,} \PY{n}{RMStest}
        \PY{k}{print} \PY{l+s+s2}{\PYZdq{}}\PY{l+s+s2}{Error Train}\PY{l+s+s2}{\PYZdq{}}\PY{p}{,} \PY{n}{RMStrain}
\end{Verbatim}


    \begin{Verbatim}[commandchars=\\\{\}]
Error Test 0.0049320961667611435
Error Train 0.0047952158930751314

    \end{Verbatim}

    \begin{center}
    \adjustimage{max size={0.9\linewidth}{0.9\paperheight}}{output_19_1.png}
    \end{center}
    { \hspace*{\fill} \\}
    
    \subsubsection{Regresion con funciones bases
sigmoidal}\label{regresion-con-funciones-bases-sigmoidal}

    Ahora analizaremos la función GSR, pero esta vez las funciones bases
seran sigmoidal y ocurre lo mismo, se reduje la cantidad de funciones
bases, pero aumento el error, tambien muy poco significativo En
conclusion, con este modelo de regresión lineal y para esta señal si se
quiere menor error, viene siendo mejor trabajar con funciones bases
polinomiales, pero si lo que se quiere es disminuir la complejidad
computacional, lo mejor vendria siendo trabajar con funciones bases
sigmoidal.

    \begin{Verbatim}[commandchars=\\\{\}]
{\color{incolor}In [{\color{incolor}8}]:} \PY{n}{PHIbTr}\PY{p}{,}\PY{n}{w\PYZus{}ML}\PY{p}{,}\PY{n}{yEstimadoTr}\PY{p}{,} \PY{n}{s2} \PY{o}{=} \PY{n}{LS}\PY{p}{(}\PY{n}{xTr}\PY{p}{,}\PY{n}{GSRTr}\PY{p}{,}\PY{l+s+s1}{\PYZsq{}}\PY{l+s+s1}{sig}\PY{l+s+s1}{\PYZsq{}}\PY{p}{,}\PY{l+m+mi}{8}\PY{p}{)}
        \PY{n}{PHIbTe}\PY{p}{,}\PY{n}{w\PYZus{}MLTe}\PY{p}{,}\PY{n}{yEstimadoTe}\PY{p}{,} \PY{n}{s2} \PY{o}{=} \PY{n}{LS}\PY{p}{(}\PY{n}{xTe}\PY{p}{,}\PY{n}{GSRTe}\PY{p}{,}\PY{l+s+s1}{\PYZsq{}}\PY{l+s+s1}{sig}\PY{l+s+s1}{\PYZsq{}}\PY{p}{,}\PY{l+m+mi}{8}\PY{p}{)}
        \PY{n}{yEstimadoTe} \PY{o}{=} \PY{n}{PHIbTe}\PY{o}{.}\PY{n}{dot}\PY{p}{(}\PY{n}{w\PYZus{}ML}\PY{p}{)}
        \PY{n}{plt}\PY{o}{.}\PY{n}{plot}\PY{p}{(}\PY{n}{xTr}\PY{p}{,}\PY{n}{yEstimadoTr}\PY{p}{,} \PY{l+s+s1}{\PYZsq{}}\PY{l+s+s1}{og}\PY{l+s+s1}{\PYZsq{}}\PY{p}{)}
        \PY{n}{plt}\PY{o}{.}\PY{n}{plot}\PY{p}{(}\PY{n}{xTe}\PY{p}{,}\PY{n}{yEstimadoTe}\PY{p}{,} \PY{l+s+s1}{\PYZsq{}}\PY{l+s+s1}{or}\PY{l+s+s1}{\PYZsq{}}\PY{p}{)}
        \PY{n}{plt}\PY{o}{.}\PY{n}{plot}\PY{p}{(}\PY{n}{x}\PY{p}{,}\PY{n}{GSR}\PY{p}{,} \PY{l+s+s1}{\PYZsq{}}\PY{l+s+s1}{\PYZhy{}k}\PY{l+s+s1}{\PYZsq{}}\PY{p}{)}
        
        \PY{n}{RMStest} \PY{o}{=} \PY{n}{Erms}\PY{p}{(}\PY{n}{GSRTr}\PY{p}{,} \PY{n}{yEstimadoTr}\PY{p}{)}
        \PY{n}{RMStrain} \PY{o}{=} \PY{n}{Erms}\PY{p}{(}\PY{n}{GSRTe}\PY{p}{,} \PY{n}{yEstimadoTe}\PY{p}{)}
        
        \PY{k}{print} \PY{l+s+s2}{\PYZdq{}}\PY{l+s+s2}{Error Test}\PY{l+s+s2}{\PYZdq{}}\PY{p}{,} \PY{n}{RMStest}
        \PY{k}{print} \PY{l+s+s2}{\PYZdq{}}\PY{l+s+s2}{Error Train}\PY{l+s+s2}{\PYZdq{}}\PY{p}{,} \PY{n}{RMStrain}
\end{Verbatim}


    \begin{Verbatim}[commandchars=\\\{\}]
Error Test 0.0023317265622117608
Error Train 0.002293375626876105

    \end{Verbatim}

    \begin{center}
    \adjustimage{max size={0.9\linewidth}{0.9\paperheight}}{output_22_1.png}
    \end{center}
    { \hspace*{\fill} \\}
    
    \subsubsection{Funcion de Regresion lineal con regularización y
Error}\label{funcion-de-regresion-lineal-con-regularizaciuxf3n-y-error}

    Controlar el sobre entrenamiento.

    \begin{Verbatim}[commandchars=\\\{\}]
{\color{incolor}In [{\color{incolor}9}]:} \PY{k}{def} \PY{n+nf}{LS\PYZus{}Reg}\PY{p}{(}\PY{n}{X}\PY{p}{,}\PY{n}{t}\PY{p}{,}\PY{n}{basisFNC}\PY{p}{,}\PY{n}{NbF}\PY{p}{,}\PY{n}{lambdaI}\PY{p}{)}\PY{p}{:}
            \PY{n}{Ndata}\PY{p}{,}\PY{n}{D} \PY{o}{=} \PY{n}{X}\PY{o}{.}\PY{n}{shape}
            \PY{n}{I} \PY{o}{=} \PY{n}{np}\PY{o}{.}\PY{n}{eye}\PY{p}{(}\PY{n}{NbF}\PY{o}{+}\PY{l+m+mi}{1}\PY{p}{)}
            \PY{c+c1}{\PYZsh{}print Ndata,D}
            \PY{n}{yEst} \PY{o}{=} \PY{n}{np}\PY{o}{.}\PY{n}{zeros}\PY{p}{(}\PY{p}{(}\PY{n}{Ndata}\PY{p}{,}\PY{l+m+mi}{1}\PY{p}{)}\PY{p}{)}
            \PY{c+c1}{\PYZsh{} Calculo de la matriz PHI de funciones base}
            \PY{n}{PHI} \PY{o}{=} \PY{n}{np}\PY{o}{.}\PY{n}{zeros}\PY{p}{(}\PY{p}{(}\PY{n}{Ndata}\PY{p}{,}\PY{n}{NbF}\PY{o}{+}\PY{l+m+mi}{1}\PY{p}{)}\PY{p}{)}
            \PY{n}{PHI}\PY{p}{[}\PY{p}{:}\PY{p}{,}\PY{l+m+mi}{0}\PY{p}{]} \PY{o}{=} \PY{l+m+mi}{1}
            \PY{n}{mu} \PY{o}{=} \PY{n}{np}\PY{o}{.}\PY{n}{linspace}\PY{p}{(}\PY{l+m+mf}{0.0}\PY{p}{,} \PY{l+m+mf}{1.0}\PY{p}{,} \PY{n}{NbF}\PY{o}{+}\PY{l+m+mi}{1}\PY{p}{)}
            \PY{n}{s2} \PY{o}{=} \PY{n}{np}\PY{o}{.}\PY{n}{var}\PY{p}{(}\PY{n}{mu}\PY{p}{)}
            \PY{k}{for} \PY{n}{n} \PY{o+ow}{in} \PY{n+nb}{range}\PY{p}{(}\PY{l+m+mi}{0}\PY{p}{,}\PY{n}{Ndata}\PY{p}{)}\PY{p}{:}
                \PY{c+c1}{\PYZsh{}print X[n]}
                \PY{k}{for} \PY{n}{i} \PY{o+ow}{in} \PY{n+nb}{range}\PY{p}{(}\PY{l+m+mi}{1}\PY{p}{,}\PY{n}{NbF}\PY{o}{+}\PY{l+m+mi}{1}\PY{p}{)}\PY{p}{:}
                    \PY{k}{if} \PY{n}{basisFNC} \PY{o}{==} \PY{l+s+s1}{\PYZsq{}}\PY{l+s+s1}{pol}\PY{l+s+s1}{\PYZsq{}}\PY{p}{:}                
                        \PY{n}{PHI}\PY{p}{[}\PY{n}{n}\PY{p}{]}\PY{p}{[}\PY{n}{i}\PY{p}{]} \PY{o}{=} \PY{n}{X}\PY{p}{[}\PY{n}{n}\PY{p}{]}\PY{o}{*}\PY{o}{*}\PY{p}{(}\PY{n}{i}\PY{p}{)}
                    \PY{k}{if} \PY{n}{basisFNC} \PY{o}{==} \PY{l+s+s1}{\PYZsq{}}\PY{l+s+s1}{exp}\PY{l+s+s1}{\PYZsq{}}\PY{p}{:}
                        \PY{n}{PHI}\PY{p}{[}\PY{n}{n}\PY{p}{]}\PY{p}{[}\PY{n}{i}\PY{p}{]} \PY{o}{=} \PY{n}{math}\PY{o}{.}\PY{n}{exp}\PY{p}{(}\PY{o}{\PYZhy{}}\PY{p}{(}\PY{p}{(}\PY{n}{X}\PY{p}{[}\PY{n}{n}\PY{p}{]}\PY{o}{\PYZhy{}}\PY{n}{mu}\PY{p}{[}\PY{n}{i}\PY{p}{]}\PY{p}{)}\PY{o}{*}\PY{o}{*}\PY{l+m+mi}{2}\PY{p}{)}\PY{o}{/}\PY{p}{(}\PY{l+m+mi}{2}\PY{o}{*}\PY{n}{s2}\PY{p}{)}\PY{p}{)}
                    \PY{k}{if} \PY{n}{basisFNC} \PY{o}{==} \PY{l+s+s1}{\PYZsq{}}\PY{l+s+s1}{sig}\PY{l+s+s1}{\PYZsq{}}\PY{p}{:}
                        \PY{n}{PHI}\PY{p}{[}\PY{n}{n}\PY{p}{]}\PY{p}{[}\PY{n}{i}\PY{p}{]} \PY{o}{=} \PY{p}{(}\PY{l+m+mi}{1} \PY{o}{/} \PY{p}{(}\PY{l+m+mi}{1} \PY{o}{+} \PY{n}{np}\PY{o}{.}\PY{n}{exp}\PY{p}{(}\PY{o}{\PYZhy{}}\PY{p}{(}\PY{p}{(}\PY{n}{X}\PY{p}{[}\PY{n}{n}\PY{p}{]}\PY{o}{\PYZhy{}}\PY{n}{mu}\PY{p}{[}\PY{n}{i}\PY{p}{]}\PY{p}{)}\PY{o}{/}\PY{n}{math}\PY{o}{.}\PY{n}{sqrt}\PY{p}{(}\PY{n}{s2}\PY{p}{)}\PY{p}{)}\PY{p}{)}\PY{p}{)}\PY{p}{)}
        \PY{c+c1}{\PYZsh{} Luego se estima el mejor W que maximiza la verosimilitud utilizando minimos cuadrados}
            \PY{n}{PHIT} \PY{o}{=} \PY{n}{PHI}\PY{o}{.}\PY{n}{T}
            \PY{n}{w\PYZus{}ml} \PY{o}{=} \PY{n}{np}\PY{o}{.}\PY{n}{linalg}\PY{o}{.}\PY{n}{inv}\PY{p}{(}\PY{n}{lambdaI}\PY{o}{*}\PY{n}{I}\PY{o}{+}\PY{n}{PHIT}\PY{o}{.}\PY{n}{dot}\PY{p}{(}\PY{n}{PHI}\PY{p}{)}\PY{p}{)}\PY{o}{.}\PY{n}{dot}\PY{p}{(}\PY{n}{PHIT}\PY{o}{.}\PY{n}{dot}\PY{p}{(}\PY{n}{t}\PY{p}{)}\PY{p}{)}
            \PY{n}{yEst} \PY{o}{=} \PY{n}{PHI}\PY{o}{.}\PY{n}{dot}\PY{p}{(}\PY{n}{w\PYZus{}ml}\PY{p}{)}
            \PY{c+c1}{\PYZsh{}print w\PYZus{}ml}
            \PY{k}{return} \PY{n}{PHI}\PY{p}{,}\PY{n}{w\PYZus{}ml}\PY{p}{,}\PY{n}{yEst}
        
        \PY{k}{def} \PY{n+nf}{Erms}\PY{p}{(}\PY{n}{to}\PY{p}{,} \PY{n}{te}\PY{p}{)}\PY{p}{:}
            \PY{n}{N}\PY{p}{,} \PY{n}{d} \PY{o}{=} \PY{n}{to}\PY{o}{.}\PY{n}{shape}
            \PY{n}{eRMS} \PY{o}{=} \PY{n}{np}\PY{o}{.}\PY{n}{sqrt}\PY{p}{(}\PY{p}{(}\PY{l+m+mf}{1.0}\PY{o}{/}\PY{n}{N}\PY{p}{)}\PY{o}{*}\PY{n}{np}\PY{o}{.}\PY{n}{sum}\PY{p}{(}\PY{p}{(}\PY{n}{to}\PY{o}{\PYZhy{}}\PY{n}{te}\PY{p}{)}\PY{o}{*}\PY{o}{*}\PY{l+m+mi}{2}\PY{p}{)}\PY{p}{)}
            \PY{k}{return} \PY{n}{eRMS}
\end{Verbatim}


    \subsubsection{Regresion con regularización con funciones bases
polinomial}\label{regresion-con-regularizaciuxf3n-con-funciones-bases-polinomial}

    A la hora de hacer la regresion con regularizacion con funciones bases
polinomiales, se logra disminuir el error pero de una manera muy poco
visible, pero se logra que cuando se aumentan las funciones bases, no se
sobreentrena como pasa en la regresion lineal normal.

    \begin{Verbatim}[commandchars=\\\{\}]
{\color{incolor}In [{\color{incolor}10}]:} \PY{n}{lambdaI} \PY{o}{=} \PY{n}{math}\PY{o}{.}\PY{n}{exp}\PY{p}{(}\PY{o}{\PYZhy{}}\PY{l+m+mf}{18.0}\PY{p}{)}
         \PY{n}{PHIbTr}\PY{p}{,}\PY{n}{w\PYZus{}MLReg}\PY{p}{,}\PY{n}{yEstimadoTr} \PY{o}{=} \PY{n}{LS\PYZus{}Reg}\PY{p}{(}\PY{n}{xTr}\PY{p}{,}\PY{n}{GSRTr}\PY{p}{,}\PY{l+s+s1}{\PYZsq{}}\PY{l+s+s1}{pol}\PY{l+s+s1}{\PYZsq{}}\PY{p}{,}\PY{l+m+mi}{100}\PY{p}{,} \PY{n}{lambdaI}\PY{p}{)}
         \PY{n}{PHIbTe}\PY{p}{,}\PY{n}{w\PYZus{}MLRegTe}\PY{p}{,}\PY{n}{yEstimadoTe} \PY{o}{=} \PY{n}{LS\PYZus{}Reg}\PY{p}{(}\PY{n}{xTe}\PY{p}{,}\PY{n}{GSRTe}\PY{p}{,}\PY{l+s+s1}{\PYZsq{}}\PY{l+s+s1}{pol}\PY{l+s+s1}{\PYZsq{}}\PY{p}{,}\PY{l+m+mi}{100}\PY{p}{,} \PY{n}{lambdaI}\PY{p}{)}
         \PY{n}{yEstimadoTe} \PY{o}{=} \PY{n}{PHIbTe}\PY{o}{.}\PY{n}{dot}\PY{p}{(}\PY{n}{w\PYZus{}MLReg}\PY{p}{)}
         \PY{n}{plt}\PY{o}{.}\PY{n}{plot}\PY{p}{(}\PY{n}{xTr}\PY{p}{,}\PY{n}{yEstimadoTr}\PY{p}{,} \PY{l+s+s1}{\PYZsq{}}\PY{l+s+s1}{og}\PY{l+s+s1}{\PYZsq{}}\PY{p}{)}
         \PY{n}{plt}\PY{o}{.}\PY{n}{plot}\PY{p}{(}\PY{n}{xTe}\PY{p}{,}\PY{n}{yEstimadoTe}\PY{p}{,} \PY{l+s+s1}{\PYZsq{}}\PY{l+s+s1}{or}\PY{l+s+s1}{\PYZsq{}}\PY{p}{)}
         \PY{n}{plt}\PY{o}{.}\PY{n}{plot}\PY{p}{(}\PY{n}{x}\PY{p}{,}\PY{n}{GSR}\PY{p}{,} \PY{l+s+s1}{\PYZsq{}}\PY{l+s+s1}{\PYZhy{}k}\PY{l+s+s1}{\PYZsq{}}\PY{p}{)}
         
         \PY{n}{RMStest} \PY{o}{=} \PY{n}{Erms}\PY{p}{(}\PY{n}{GSRTr}\PY{p}{,} \PY{n}{yEstimadoTr}\PY{p}{)}
         \PY{n}{RMStrain} \PY{o}{=} \PY{n}{Erms}\PY{p}{(}\PY{n}{GSRTe}\PY{p}{,} \PY{n}{yEstimadoTe}\PY{p}{)}
         
         \PY{k}{print} \PY{l+s+s2}{\PYZdq{}}\PY{l+s+s2}{Error Test}\PY{l+s+s2}{\PYZdq{}}\PY{p}{,} \PY{n}{RMStest}
         \PY{k}{print} \PY{l+s+s2}{\PYZdq{}}\PY{l+s+s2}{Error Train}\PY{l+s+s2}{\PYZdq{}}\PY{p}{,} \PY{n}{RMStrain}
\end{Verbatim}


    \begin{Verbatim}[commandchars=\\\{\}]
Error Test 0.002042044465995395
Error Train 0.0020124087281880253

    \end{Verbatim}

    \begin{center}
    \adjustimage{max size={0.9\linewidth}{0.9\paperheight}}{output_28_1.png}
    \end{center}
    { \hspace*{\fill} \\}
    
    \subsubsection{Regresión con regularización con funciones
exponenciales}\label{regresiuxf3n-con-regularizaciuxf3n-con-funciones-exponenciales}

    A la hora de hacerlo con funciones bases exponenciales, sigue existiendo
el mismo que con la regresión lineal normal. Pero se puede observar que
tambien disminuyo el error.

    \begin{Verbatim}[commandchars=\\\{\}]
{\color{incolor}In [{\color{incolor}11}]:} \PY{n}{lambdaI} \PY{o}{=} \PY{n}{math}\PY{o}{.}\PY{n}{exp}\PY{p}{(}\PY{o}{\PYZhy{}}\PY{l+m+mf}{18.0}\PY{p}{)}
         \PY{n}{PHIbTr}\PY{p}{,}\PY{n}{w\PYZus{}MLReg}\PY{p}{,}\PY{n}{yEstimadoTr} \PY{o}{=} \PY{n}{LS\PYZus{}Reg}\PY{p}{(}\PY{n}{xTr}\PY{p}{,}\PY{n}{GSRTr}\PY{p}{,}\PY{l+s+s1}{\PYZsq{}}\PY{l+s+s1}{exp}\PY{l+s+s1}{\PYZsq{}}\PY{p}{,}\PY{l+m+mi}{100}\PY{p}{,} \PY{n}{lambdaI}\PY{p}{)}
         \PY{n}{PHIbTe}\PY{p}{,}\PY{n}{w\PYZus{}MLRegTe}\PY{p}{,}\PY{n}{yEstimadoTe} \PY{o}{=} \PY{n}{LS\PYZus{}Reg}\PY{p}{(}\PY{n}{xTe}\PY{p}{,}\PY{n}{GSRTe}\PY{p}{,}\PY{l+s+s1}{\PYZsq{}}\PY{l+s+s1}{exp}\PY{l+s+s1}{\PYZsq{}}\PY{p}{,}\PY{l+m+mi}{100}\PY{p}{,} \PY{n}{lambdaI}\PY{p}{)}
         \PY{n}{yEstimadoTe} \PY{o}{=} \PY{n}{PHIbTe}\PY{o}{.}\PY{n}{dot}\PY{p}{(}\PY{n}{w\PYZus{}MLReg}\PY{p}{)}
         \PY{n}{plt}\PY{o}{.}\PY{n}{plot}\PY{p}{(}\PY{n}{xTr}\PY{p}{,}\PY{n}{yEstimadoTr}\PY{p}{,} \PY{l+s+s1}{\PYZsq{}}\PY{l+s+s1}{og}\PY{l+s+s1}{\PYZsq{}}\PY{p}{)}
         \PY{n}{plt}\PY{o}{.}\PY{n}{plot}\PY{p}{(}\PY{n}{xTe}\PY{p}{,}\PY{n}{yEstimadoTe}\PY{p}{,} \PY{l+s+s1}{\PYZsq{}}\PY{l+s+s1}{or}\PY{l+s+s1}{\PYZsq{}}\PY{p}{)}
         \PY{n}{plt}\PY{o}{.}\PY{n}{plot}\PY{p}{(}\PY{n}{x}\PY{p}{,}\PY{n}{GSR}\PY{p}{,} \PY{l+s+s1}{\PYZsq{}}\PY{l+s+s1}{\PYZhy{}k}\PY{l+s+s1}{\PYZsq{}}\PY{p}{)}
         
         \PY{n}{RMStest} \PY{o}{=} \PY{n}{Erms}\PY{p}{(}\PY{n}{GSRTr}\PY{p}{,} \PY{n}{yEstimadoTr}\PY{p}{)}
         \PY{n}{RMStrain} \PY{o}{=} \PY{n}{Erms}\PY{p}{(}\PY{n}{GSRTe}\PY{p}{,} \PY{n}{yEstimadoTe}\PY{p}{)}
         
         \PY{k}{print} \PY{l+s+s2}{\PYZdq{}}\PY{l+s+s2}{Error Test}\PY{l+s+s2}{\PYZdq{}}\PY{p}{,} \PY{n}{RMStest}
         \PY{k}{print} \PY{l+s+s2}{\PYZdq{}}\PY{l+s+s2}{Error Train}\PY{l+s+s2}{\PYZdq{}}\PY{p}{,} \PY{n}{RMStrain}
\end{Verbatim}


    \begin{Verbatim}[commandchars=\\\{\}]
Error Test 0.0022077070283606937
Error Train 0.002183021685423015

    \end{Verbatim}

    \begin{center}
    \adjustimage{max size={0.9\linewidth}{0.9\paperheight}}{output_31_1.png}
    \end{center}
    { \hspace*{\fill} \\}
    
    \subsubsection{Regresión con regularización con funciones
sigmoidales}\label{regresiuxf3n-con-regularizaciuxf3n-con-funciones-sigmoidales}

    Lo mismo resulta con las funcionses sigmoidales, se puede apreciar que
se noto una gran mejoria a la hora de hacer la regresión con
regularizacion a la hora de hacer el Train, pero cuando se testea, la
función resultante no es nada parecida a lo que necesitamos.

    \begin{Verbatim}[commandchars=\\\{\}]
{\color{incolor}In [{\color{incolor}12}]:} \PY{n}{lambdaI} \PY{o}{=} \PY{n}{math}\PY{o}{.}\PY{n}{exp}\PY{p}{(}\PY{o}{\PYZhy{}}\PY{l+m+mf}{18.0}\PY{p}{)}
         \PY{n}{PHIbTr}\PY{p}{,}\PY{n}{w\PYZus{}MLReg}\PY{p}{,}\PY{n}{yEstimadoTr} \PY{o}{=} \PY{n}{LS\PYZus{}Reg}\PY{p}{(}\PY{n}{xTr}\PY{p}{,}\PY{n}{GSRTr}\PY{p}{,}\PY{l+s+s1}{\PYZsq{}}\PY{l+s+s1}{sig}\PY{l+s+s1}{\PYZsq{}}\PY{p}{,}\PY{l+m+mi}{100}\PY{p}{,} \PY{n}{lambdaI}\PY{p}{)}
         \PY{n}{PHIbTe}\PY{p}{,}\PY{n}{w\PYZus{}MLRegTe}\PY{p}{,}\PY{n}{yEstimadoTe} \PY{o}{=} \PY{n}{LS\PYZus{}Reg}\PY{p}{(}\PY{n}{xTe}\PY{p}{,}\PY{n}{GSRTe}\PY{p}{,}\PY{l+s+s1}{\PYZsq{}}\PY{l+s+s1}{sig}\PY{l+s+s1}{\PYZsq{}}\PY{p}{,}\PY{l+m+mi}{100}\PY{p}{,} \PY{n}{lambdaI}\PY{p}{)}
         \PY{n}{yEstimadoTe} \PY{o}{=} \PY{n}{PHIbTe}\PY{o}{.}\PY{n}{dot}\PY{p}{(}\PY{n}{w\PYZus{}MLReg}\PY{p}{)}
         \PY{n}{plt}\PY{o}{.}\PY{n}{plot}\PY{p}{(}\PY{n}{xTr}\PY{p}{,}\PY{n}{yEstimadoTr}\PY{p}{,} \PY{l+s+s1}{\PYZsq{}}\PY{l+s+s1}{og}\PY{l+s+s1}{\PYZsq{}}\PY{p}{)}
         \PY{n}{plt}\PY{o}{.}\PY{n}{plot}\PY{p}{(}\PY{n}{xTe}\PY{p}{,}\PY{n}{yEstimadoTe}\PY{p}{,} \PY{l+s+s1}{\PYZsq{}}\PY{l+s+s1}{or}\PY{l+s+s1}{\PYZsq{}}\PY{p}{)}
         \PY{n}{plt}\PY{o}{.}\PY{n}{plot}\PY{p}{(}\PY{n}{x}\PY{p}{,}\PY{n}{GSR}\PY{p}{,} \PY{l+s+s1}{\PYZsq{}}\PY{l+s+s1}{\PYZhy{}k}\PY{l+s+s1}{\PYZsq{}}\PY{p}{)}
         
         \PY{n}{RMStest} \PY{o}{=} \PY{n}{Erms}\PY{p}{(}\PY{n}{GSRTr}\PY{p}{,} \PY{n}{yEstimadoTr}\PY{p}{)}
         \PY{n}{RMStrain} \PY{o}{=} \PY{n}{Erms}\PY{p}{(}\PY{n}{GSRTe}\PY{p}{,} \PY{n}{yEstimadoTe}\PY{p}{)}
         
         \PY{k}{print} \PY{l+s+s2}{\PYZdq{}}\PY{l+s+s2}{Error Test}\PY{l+s+s2}{\PYZdq{}}\PY{p}{,} \PY{n}{RMStest}
         \PY{k}{print} \PY{l+s+s2}{\PYZdq{}}\PY{l+s+s2}{Error Train}\PY{l+s+s2}{\PYZdq{}}\PY{p}{,} \PY{n}{RMStrain}
\end{Verbatim}


    \begin{Verbatim}[commandchars=\\\{\}]
Error Test 0.0023224158927136382
Error Train 0.002290324228040209

    \end{Verbatim}

    \begin{center}
    \adjustimage{max size={0.9\linewidth}{0.9\paperheight}}{output_34_1.png}
    \end{center}
    { \hspace*{\fill} \\}
    
    Entre trabajar con regresion lineal y regresion lineal con
regularización, a la hora de buscar menor error, conviene trabajar con
funciones bases polinomial, por lo expuesto y verificado con el error.

    \subsection{Regresion Bayesiana}\label{regresion-bayesiana}

    \subsubsection{Regresión bayesiana con funciones base
polinomial}\label{regresiuxf3n-bayesiana-con-funciones-base-polinomial}

    A la hora de analizar nuestra señal GSR con la regresion bayesiana,
tenemos que tener tambien una matriz de funciones bases, en este caso lo
haremos con funciones bases polinomiales, despues de varias el alpha y
beta, no se pudo llegar a un ajuste como en la regresion lineal y
regresion lineal con regularización.

    \begin{Verbatim}[commandchars=\\\{\}]
{\color{incolor}In [{\color{incolor}13}]:} \PY{n}{M} \PY{o}{=} \PY{l+m+mi}{100} \PY{c+c1}{\PYZsh{} numero de funciones base}
         \PY{n}{PHI}\PY{p}{,}\PY{n}{w\PYZus{}MLReg}\PY{p}{,}\PY{n}{yEstimadoTr}\PY{p}{,} \PY{n}{s2} \PY{o}{=} \PY{n}{LS}\PY{p}{(}\PY{n}{xTr}\PY{p}{,}\PY{n}{GSRTr}\PY{p}{,}\PY{l+s+s1}{\PYZsq{}}\PY{l+s+s1}{pol}\PY{l+s+s1}{\PYZsq{}}\PY{p}{,}\PY{n}{M}\PY{o}{\PYZhy{}}\PY{l+m+mi}{1}\PY{p}{)}
         \PY{k}{print} \PY{n}{PHI}\PY{o}{.}\PY{n}{shape}
         
         \PY{n}{iT} \PY{o}{=} \PY{l+m+mi}{100} \PY{c+c1}{\PYZsh{} Numero de iteraciones}
         \PY{n}{alpha} \PY{o}{=} \PY{l+m+mf}{0.0}
         \PY{n}{beta} \PY{o}{=} \PY{l+m+mf}{0.1}
         \PY{n}{invbeta} \PY{o}{=} \PY{l+m+mi}{1}\PY{o}{/}\PY{n}{beta}
         \PY{n}{PHIT} \PY{o}{=} \PY{n}{PHI}\PY{o}{.}\PY{n}{T}
         \PY{n}{invSn} \PY{o}{=} \PY{n}{alpha}\PY{o}{*}\PY{n}{np}\PY{o}{.}\PY{n}{eye}\PY{p}{(}\PY{n}{M}\PY{p}{)}\PY{o}{+} \PY{n}{beta}\PY{o}{*}\PY{n}{PHIT}\PY{o}{.}\PY{n}{dot}\PY{p}{(}\PY{n}{PHI}\PY{p}{)}
         \PY{n}{Sn} \PY{o}{=} \PY{n}{np}\PY{o}{.}\PY{n}{linalg}\PY{o}{.}\PY{n}{inv}\PY{p}{(}\PY{n}{invSn}\PY{p}{)}
         \PY{n}{mn} \PY{o}{=} \PY{n}{beta}\PY{o}{*}\PY{n}{Sn}\PY{o}{.}\PY{n}{dot}\PY{p}{(}\PY{n}{PHIT}\PY{o}{.}\PY{n}{dot}\PY{p}{(}\PY{n}{GSRTr}\PY{p}{)}\PY{p}{)}
         \PY{k}{print} \PY{n}{mn}\PY{o}{.}\PY{n}{shape}\PY{p}{,}\PY{n}{Sn}\PY{o}{.}\PY{n}{shape}
         
         \PY{c+c1}{\PYZsh{} Probemos la estimacion con el mn inicial}
         \PY{n}{yEst} \PY{o}{=} \PY{n}{PHI}\PY{o}{.}\PY{n}{dot}\PY{p}{(}\PY{n}{mn}\PY{p}{)}
         \PY{n}{lambdaIp}\PY{p}{,}\PY{n}{vecI} \PY{o}{=} \PY{n}{np}\PY{o}{.}\PY{n}{linalg}\PY{o}{.}\PY{n}{eig}\PY{p}{(}\PY{n}{PHIT}\PY{o}{.}\PY{n}{dot}\PY{p}{(}\PY{n}{PHI}\PY{p}{)}\PY{p}{)}
         \PY{n}{lambdaI} \PY{o}{=} \PY{n}{beta}\PY{o}{*}\PY{n}{lambdaIp}
         \PY{k}{for} \PY{n}{j} \PY{o+ow}{in} \PY{n+nb}{range}\PY{p}{(}\PY{l+m+mi}{0}\PY{p}{,}\PY{n}{iT}\PY{p}{)}\PY{p}{:}
             \PY{n}{gamma} \PY{o}{=} \PY{n}{np}\PY{o}{.}\PY{n}{sum}\PY{p}{(}\PY{n}{lambdaI}\PY{o}{/}\PY{p}{(}\PY{n}{alpha}\PY{o}{*}\PY{n}{np}\PY{o}{.}\PY{n}{ones}\PY{p}{(}\PY{n}{lambdaI}\PY{o}{.}\PY{n}{shape}\PY{p}{)}\PY{o}{+}\PY{n}{lambdaI}\PY{p}{)}\PY{p}{)}
             \PY{n}{alpha} \PY{o}{=} \PY{n}{gamma}\PY{o}{/}\PY{p}{(}\PY{p}{(}\PY{n}{mn}\PY{o}{.}\PY{n}{T}\PY{p}{)}\PY{o}{.}\PY{n}{dot}\PY{p}{(}\PY{n}{mn}\PY{p}{)}\PY{p}{)}
             \PY{n}{invbeta} \PY{o}{=} \PY{p}{(}\PY{l+m+mi}{1}\PY{o}{/}\PY{p}{(}\PY{n}{N}\PY{o}{\PYZhy{}}\PY{n}{gamma}\PY{p}{)}\PY{p}{)}\PY{o}{*}\PY{p}{(}\PY{p}{(}\PY{p}{(}\PY{n}{GSRTr}\PY{o}{\PYZhy{}}\PY{n}{PHI}\PY{o}{.}\PY{n}{dot}\PY{p}{(}\PY{n}{mn}\PY{p}{)}\PY{p}{)}\PY{o}{.}\PY{n}{T}\PY{p}{)}\PY{o}{.}\PY{n}{dot}\PY{p}{(}\PY{n}{GSRTr}\PY{o}{\PYZhy{}}\PY{n}{PHI}\PY{o}{.}\PY{n}{dot}\PY{p}{(}\PY{n}{mn}\PY{p}{)}\PY{p}{)}\PY{p}{)}
             \PY{n}{beta} \PY{o}{=} \PY{l+m+mi}{1}\PY{o}{/}\PY{n}{invbeta}
             \PY{n}{lambdaI} \PY{o}{=} \PY{n}{beta}\PY{o}{*}\PY{n}{lambdaIp}
             \PY{n}{invSn} \PY{o}{=} \PY{n}{alpha}\PY{o}{*}\PY{n}{np}\PY{o}{.}\PY{n}{eye}\PY{p}{(}\PY{n}{M}\PY{p}{)}\PY{o}{+} \PY{n}{beta}\PY{o}{*}\PY{n}{PHIT}\PY{o}{.}\PY{n}{dot}\PY{p}{(}\PY{n}{PHI}\PY{p}{)}
             \PY{n}{Sn} \PY{o}{=} \PY{n}{np}\PY{o}{.}\PY{n}{linalg}\PY{o}{.}\PY{n}{inv}\PY{p}{(}\PY{n}{invSn}\PY{p}{)}
             \PY{n}{mn} \PY{o}{=} \PY{n}{beta}\PY{o}{*}\PY{n}{Sn}\PY{o}{.}\PY{n}{dot}\PY{p}{(}\PY{n}{PHIT}\PY{o}{.}\PY{n}{dot}\PY{p}{(}\PY{n}{GSRTr}\PY{p}{)}\PY{p}{)}
             
         \PY{c+c1}{\PYZsh{}print invSn }
         \PY{n}{yEstfin} \PY{o}{=} \PY{n}{PHI}\PY{o}{.}\PY{n}{dot}\PY{p}{(}\PY{n}{mn}\PY{p}{)}
         \PY{n}{plt}\PY{o}{.}\PY{n}{plot}\PY{p}{(}\PY{n}{xTr}\PY{p}{,}\PY{n}{GSRTr}\PY{p}{,}\PY{l+s+s1}{\PYZsq{}}\PY{l+s+s1}{or}\PY{l+s+s1}{\PYZsq{}}\PY{p}{)}
         \PY{n}{plt}\PY{o}{.}\PY{n}{plot}\PY{p}{(}\PY{n}{xTr}\PY{p}{,}\PY{n}{yEstfin}\PY{p}{,}\PY{l+s+s1}{\PYZsq{}}\PY{l+s+s1}{ok}\PY{l+s+s1}{\PYZsq{}}\PY{p}{)}
         
         
         \PY{n}{RMStrain} \PY{o}{=} \PY{n}{Erms}\PY{p}{(}\PY{n}{GSRTr}\PY{p}{,} \PY{n}{yEstfin}\PY{p}{)}
         \PY{k}{print} \PY{l+s+s2}{\PYZdq{}}\PY{l+s+s2}{Error Train}\PY{l+s+s2}{\PYZdq{}}\PY{p}{,} \PY{n}{RMStrain}
\end{Verbatim}


    \begin{Verbatim}[commandchars=\\\{\}]
(4570L, 100L)
(100L, 1L) (100L, 100L)

    \end{Verbatim}

    \begin{Verbatim}[commandchars=\\\{\}]
C:\textbackslash{}Users\textbackslash{}XxSoaD\textbackslash{}Anaconda2\textbackslash{}lib\textbackslash{}site-packages\textbackslash{}numpy\textbackslash{}core\textbackslash{}numeric.py:492: ComplexWarning: Casting complex values to real discards the imaginary part
  return array(a, dtype, copy=False, order=order)

    \end{Verbatim}

    \begin{Verbatim}[commandchars=\\\{\}]
Error Train (0.7947371627070415-0.4512346694689251j)

    \end{Verbatim}

    \begin{center}
    \adjustimage{max size={0.9\linewidth}{0.9\paperheight}}{output_39_3.png}
    \end{center}
    { \hspace*{\fill} \\}
    
    \subsubsection{Regresión bayesiana con funciones base
exponencial}\label{regresiuxf3n-bayesiana-con-funciones-base-exponencial}

    Con regresion bayesiana con funciones base exponencial, las cosas
mejoraron drasticamente comparandola con la regresion lineal. Con estas
funciones bases, son las que mas se ajusta a la señal en comparación con
las demas funciones bases, y esto se puede apreciar en las graficas.

    \begin{Verbatim}[commandchars=\\\{\}]
{\color{incolor}In [{\color{incolor}14}]:} \PY{n}{M} \PY{o}{=} \PY{l+m+mi}{100} \PY{c+c1}{\PYZsh{} numero de funciones base}
         \PY{n}{PHI}\PY{p}{,}\PY{n}{w\PYZus{}MLReg}\PY{p}{,}\PY{n}{yEstimadoTr}\PY{p}{,} \PY{n}{s2} \PY{o}{=} \PY{n}{LS}\PY{p}{(}\PY{n}{xTr}\PY{p}{,}\PY{n}{GSRTr}\PY{p}{,}\PY{l+s+s1}{\PYZsq{}}\PY{l+s+s1}{exp}\PY{l+s+s1}{\PYZsq{}}\PY{p}{,}\PY{n}{M}\PY{o}{\PYZhy{}}\PY{l+m+mi}{1}\PY{p}{)}
         \PY{k}{print} \PY{n}{PHI}\PY{o}{.}\PY{n}{shape}
         
         \PY{n}{iT} \PY{o}{=} \PY{l+m+mi}{100} \PY{c+c1}{\PYZsh{} Numero de iteraciones}
         \PY{n}{alpha} \PY{o}{=} \PY{l+m+mf}{0.35}
         \PY{n}{beta} \PY{o}{=} \PY{l+m+mf}{0.5}
         \PY{n}{invbeta} \PY{o}{=} \PY{l+m+mi}{1}\PY{o}{/}\PY{n}{beta}
         \PY{n}{PHIT} \PY{o}{=} \PY{n}{PHI}\PY{o}{.}\PY{n}{T}
         \PY{n}{invSn} \PY{o}{=} \PY{n}{alpha}\PY{o}{*}\PY{n}{np}\PY{o}{.}\PY{n}{eye}\PY{p}{(}\PY{n}{M}\PY{p}{)}\PY{o}{+} \PY{n}{beta}\PY{o}{*}\PY{n}{PHIT}\PY{o}{.}\PY{n}{dot}\PY{p}{(}\PY{n}{PHI}\PY{p}{)}
         \PY{n}{Sn} \PY{o}{=} \PY{n}{np}\PY{o}{.}\PY{n}{linalg}\PY{o}{.}\PY{n}{inv}\PY{p}{(}\PY{n}{invSn}\PY{p}{)}
         \PY{n}{mn} \PY{o}{=} \PY{n}{beta}\PY{o}{*}\PY{n}{Sn}\PY{o}{.}\PY{n}{dot}\PY{p}{(}\PY{n}{PHIT}\PY{o}{.}\PY{n}{dot}\PY{p}{(}\PY{n}{GSRTr}\PY{p}{)}\PY{p}{)}
         \PY{k}{print} \PY{n}{mn}\PY{o}{.}\PY{n}{shape}\PY{p}{,}\PY{n}{Sn}\PY{o}{.}\PY{n}{shape}
         
         \PY{c+c1}{\PYZsh{} Probemos la estimacion con el mn inicial}
         \PY{n}{yEst} \PY{o}{=} \PY{n}{PHI}\PY{o}{.}\PY{n}{dot}\PY{p}{(}\PY{n}{mn}\PY{p}{)}
         \PY{n}{lambdaIp}\PY{p}{,}\PY{n}{vecI} \PY{o}{=} \PY{n}{np}\PY{o}{.}\PY{n}{linalg}\PY{o}{.}\PY{n}{eig}\PY{p}{(}\PY{n}{PHIT}\PY{o}{.}\PY{n}{dot}\PY{p}{(}\PY{n}{PHI}\PY{p}{)}\PY{p}{)}
         \PY{n}{lambdaI} \PY{o}{=} \PY{n}{beta}\PY{o}{*}\PY{n}{lambdaIp}
         \PY{k}{for} \PY{n}{j} \PY{o+ow}{in} \PY{n+nb}{range}\PY{p}{(}\PY{l+m+mi}{0}\PY{p}{,}\PY{n}{iT}\PY{p}{)}\PY{p}{:}
             \PY{n}{gamma} \PY{o}{=} \PY{n}{np}\PY{o}{.}\PY{n}{sum}\PY{p}{(}\PY{n}{lambdaI}\PY{o}{/}\PY{p}{(}\PY{n}{alpha}\PY{o}{*}\PY{n}{np}\PY{o}{.}\PY{n}{ones}\PY{p}{(}\PY{n}{lambdaI}\PY{o}{.}\PY{n}{shape}\PY{p}{)}\PY{o}{+}\PY{n}{lambdaI}\PY{p}{)}\PY{p}{)}
             \PY{n}{alpha} \PY{o}{=} \PY{n}{gamma}\PY{o}{/}\PY{p}{(}\PY{p}{(}\PY{n}{mn}\PY{o}{.}\PY{n}{T}\PY{p}{)}\PY{o}{.}\PY{n}{dot}\PY{p}{(}\PY{n}{mn}\PY{p}{)}\PY{p}{)}
             \PY{n}{invbeta} \PY{o}{=} \PY{p}{(}\PY{l+m+mi}{1}\PY{o}{/}\PY{p}{(}\PY{n}{N}\PY{o}{\PYZhy{}}\PY{n}{gamma}\PY{p}{)}\PY{p}{)}\PY{o}{*}\PY{p}{(}\PY{p}{(}\PY{p}{(}\PY{n}{GSRTr}\PY{o}{\PYZhy{}}\PY{n}{PHI}\PY{o}{.}\PY{n}{dot}\PY{p}{(}\PY{n}{mn}\PY{p}{)}\PY{p}{)}\PY{o}{.}\PY{n}{T}\PY{p}{)}\PY{o}{.}\PY{n}{dot}\PY{p}{(}\PY{n}{GSRTr}\PY{o}{\PYZhy{}}\PY{n}{PHI}\PY{o}{.}\PY{n}{dot}\PY{p}{(}\PY{n}{mn}\PY{p}{)}\PY{p}{)}\PY{p}{)}
             \PY{n}{beta} \PY{o}{=} \PY{l+m+mi}{1}\PY{o}{/}\PY{n}{invbeta}
             \PY{n}{lambdaI} \PY{o}{=} \PY{n}{beta}\PY{o}{*}\PY{n}{lambdaIp}
             \PY{n}{invSn} \PY{o}{=} \PY{n}{alpha}\PY{o}{*}\PY{n}{np}\PY{o}{.}\PY{n}{eye}\PY{p}{(}\PY{n}{M}\PY{p}{)}\PY{o}{+} \PY{n}{beta}\PY{o}{*}\PY{n}{PHIT}\PY{o}{.}\PY{n}{dot}\PY{p}{(}\PY{n}{PHI}\PY{p}{)}
             \PY{n}{Sn} \PY{o}{=} \PY{n}{np}\PY{o}{.}\PY{n}{linalg}\PY{o}{.}\PY{n}{inv}\PY{p}{(}\PY{n}{invSn}\PY{p}{)}
             \PY{n}{mn} \PY{o}{=} \PY{n}{beta}\PY{o}{*}\PY{n}{Sn}\PY{o}{.}\PY{n}{dot}\PY{p}{(}\PY{n}{PHIT}\PY{o}{.}\PY{n}{dot}\PY{p}{(}\PY{n}{GSRTr}\PY{p}{)}\PY{p}{)}
             
         \PY{c+c1}{\PYZsh{}print invSn }
         \PY{n}{yEstfin} \PY{o}{=} \PY{n}{PHI}\PY{o}{.}\PY{n}{dot}\PY{p}{(}\PY{n}{mn}\PY{p}{)}
         \PY{n}{plt}\PY{o}{.}\PY{n}{plot}\PY{p}{(}\PY{n}{xTr}\PY{p}{,}\PY{n}{GSRTr}\PY{p}{,}\PY{l+s+s1}{\PYZsq{}}\PY{l+s+s1}{or}\PY{l+s+s1}{\PYZsq{}}\PY{p}{)}
         \PY{n}{plt}\PY{o}{.}\PY{n}{plot}\PY{p}{(}\PY{n}{xTr}\PY{p}{,}\PY{n}{yEstfin}\PY{p}{,}\PY{l+s+s1}{\PYZsq{}}\PY{l+s+s1}{ok}\PY{l+s+s1}{\PYZsq{}}\PY{p}{)}
         
         \PY{n}{RMStrain} \PY{o}{=} \PY{n}{Erms}\PY{p}{(}\PY{n}{GSRTr}\PY{p}{,} \PY{n}{yEstfin}\PY{p}{)}
         \PY{k}{print} \PY{l+s+s2}{\PYZdq{}}\PY{l+s+s2}{Error Train}\PY{l+s+s2}{\PYZdq{}}\PY{p}{,} \PY{n}{RMStrain}
\end{Verbatim}


    \begin{Verbatim}[commandchars=\\\{\}]
(4570L, 100L)
(100L, 1L) (100L, 100L)
Error Train (0.3909089640678495+0.004977291192292248j)

    \end{Verbatim}

    \begin{center}
    \adjustimage{max size={0.9\linewidth}{0.9\paperheight}}{output_42_1.png}
    \end{center}
    { \hspace*{\fill} \\}
    
    \subsubsection{Regresión bayesiana con funciones base
sigmoidal}\label{regresiuxf3n-bayesiana-con-funciones-base-sigmoidal}

    Con regresion bayesiana con funciones base sigmoidales, se ve mejorado
tambien como con la exponencial en comparativa con la regresion lineal.
Se ajusta a la señal pero hay mucha diferencia con la misma.

    \begin{Verbatim}[commandchars=\\\{\}]
{\color{incolor}In [{\color{incolor}15}]:} \PY{n}{M} \PY{o}{=} \PY{l+m+mi}{100} \PY{c+c1}{\PYZsh{} numero de funciones base}
         \PY{n}{PHI}\PY{p}{,}\PY{n}{w\PYZus{}MLReg}\PY{p}{,}\PY{n}{yEstimadoTr}\PY{p}{,} \PY{n}{s2} \PY{o}{=} \PY{n}{LS}\PY{p}{(}\PY{n}{xTr}\PY{p}{,}\PY{n}{GSRTr}\PY{p}{,}\PY{l+s+s1}{\PYZsq{}}\PY{l+s+s1}{sig}\PY{l+s+s1}{\PYZsq{}}\PY{p}{,}\PY{n}{M}\PY{o}{\PYZhy{}}\PY{l+m+mi}{1}\PY{p}{)}
         \PY{k}{print} \PY{n}{PHI}\PY{o}{.}\PY{n}{shape}
         
         \PY{n}{iT} \PY{o}{=} \PY{l+m+mi}{100} \PY{c+c1}{\PYZsh{} Numero de iteraciones}
         \PY{n}{alpha} \PY{o}{=} \PY{l+m+mf}{0.5}
         \PY{n}{beta} \PY{o}{=} \PY{l+m+mf}{0.55}
         \PY{n}{invbeta} \PY{o}{=} \PY{l+m+mi}{1}\PY{o}{/}\PY{n}{beta}
         \PY{n}{PHIT} \PY{o}{=} \PY{n}{PHI}\PY{o}{.}\PY{n}{T}
         \PY{n}{invSn} \PY{o}{=} \PY{n}{alpha}\PY{o}{*}\PY{n}{np}\PY{o}{.}\PY{n}{eye}\PY{p}{(}\PY{n}{M}\PY{p}{)}\PY{o}{+} \PY{n}{beta}\PY{o}{*}\PY{n}{PHIT}\PY{o}{.}\PY{n}{dot}\PY{p}{(}\PY{n}{PHI}\PY{p}{)}
         \PY{n}{Sn} \PY{o}{=} \PY{n}{np}\PY{o}{.}\PY{n}{linalg}\PY{o}{.}\PY{n}{inv}\PY{p}{(}\PY{n}{invSn}\PY{p}{)}
         \PY{n}{mn} \PY{o}{=} \PY{n}{beta}\PY{o}{*}\PY{n}{Sn}\PY{o}{.}\PY{n}{dot}\PY{p}{(}\PY{n}{PHIT}\PY{o}{.}\PY{n}{dot}\PY{p}{(}\PY{n}{GSRTr}\PY{p}{)}\PY{p}{)}
         \PY{k}{print} \PY{n}{mn}\PY{o}{.}\PY{n}{shape}\PY{p}{,}\PY{n}{Sn}\PY{o}{.}\PY{n}{shape}
         
         \PY{c+c1}{\PYZsh{} Probemos la estimacion con el mn inicial}
         \PY{n}{yEst} \PY{o}{=} \PY{n}{PHI}\PY{o}{.}\PY{n}{dot}\PY{p}{(}\PY{n}{mn}\PY{p}{)}
         \PY{n}{lambdaIp}\PY{p}{,}\PY{n}{vecI} \PY{o}{=} \PY{n}{np}\PY{o}{.}\PY{n}{linalg}\PY{o}{.}\PY{n}{eig}\PY{p}{(}\PY{n}{PHIT}\PY{o}{.}\PY{n}{dot}\PY{p}{(}\PY{n}{PHI}\PY{p}{)}\PY{p}{)}
         \PY{n}{lambdaI} \PY{o}{=} \PY{n}{beta}\PY{o}{*}\PY{n}{lambdaIp}
         \PY{k}{for} \PY{n}{j} \PY{o+ow}{in} \PY{n+nb}{range}\PY{p}{(}\PY{l+m+mi}{0}\PY{p}{,}\PY{n}{iT}\PY{p}{)}\PY{p}{:}
             \PY{n}{gamma} \PY{o}{=} \PY{n}{np}\PY{o}{.}\PY{n}{sum}\PY{p}{(}\PY{n}{lambdaI}\PY{o}{/}\PY{p}{(}\PY{n}{alpha}\PY{o}{*}\PY{n}{np}\PY{o}{.}\PY{n}{ones}\PY{p}{(}\PY{n}{lambdaI}\PY{o}{.}\PY{n}{shape}\PY{p}{)}\PY{o}{+}\PY{n}{lambdaI}\PY{p}{)}\PY{p}{)}
             \PY{n}{alpha} \PY{o}{=} \PY{n}{gamma}\PY{o}{/}\PY{p}{(}\PY{p}{(}\PY{n}{mn}\PY{o}{.}\PY{n}{T}\PY{p}{)}\PY{o}{.}\PY{n}{dot}\PY{p}{(}\PY{n}{mn}\PY{p}{)}\PY{p}{)}
             \PY{n}{invbeta} \PY{o}{=} \PY{p}{(}\PY{l+m+mi}{1}\PY{o}{/}\PY{p}{(}\PY{n}{N}\PY{o}{\PYZhy{}}\PY{n}{gamma}\PY{p}{)}\PY{p}{)}\PY{o}{*}\PY{p}{(}\PY{p}{(}\PY{p}{(}\PY{n}{GSRTr}\PY{o}{\PYZhy{}}\PY{n}{PHI}\PY{o}{.}\PY{n}{dot}\PY{p}{(}\PY{n}{mn}\PY{p}{)}\PY{p}{)}\PY{o}{.}\PY{n}{T}\PY{p}{)}\PY{o}{.}\PY{n}{dot}\PY{p}{(}\PY{n}{GSRTr}\PY{o}{\PYZhy{}}\PY{n}{PHI}\PY{o}{.}\PY{n}{dot}\PY{p}{(}\PY{n}{mn}\PY{p}{)}\PY{p}{)}\PY{p}{)}
             \PY{n}{beta} \PY{o}{=} \PY{l+m+mi}{1}\PY{o}{/}\PY{n}{invbeta}
             \PY{n}{lambdaI} \PY{o}{=} \PY{n}{beta}\PY{o}{*}\PY{n}{lambdaIp}
             \PY{n}{invSn} \PY{o}{=} \PY{n}{alpha}\PY{o}{*}\PY{n}{np}\PY{o}{.}\PY{n}{eye}\PY{p}{(}\PY{n}{M}\PY{p}{)}\PY{o}{+} \PY{n}{beta}\PY{o}{*}\PY{n}{PHIT}\PY{o}{.}\PY{n}{dot}\PY{p}{(}\PY{n}{PHI}\PY{p}{)}
             \PY{n}{Sn} \PY{o}{=} \PY{n}{np}\PY{o}{.}\PY{n}{linalg}\PY{o}{.}\PY{n}{inv}\PY{p}{(}\PY{n}{invSn}\PY{p}{)}
             \PY{n}{mn} \PY{o}{=} \PY{n}{beta}\PY{o}{*}\PY{n}{Sn}\PY{o}{.}\PY{n}{dot}\PY{p}{(}\PY{n}{PHIT}\PY{o}{.}\PY{n}{dot}\PY{p}{(}\PY{n}{GSRTr}\PY{p}{)}\PY{p}{)}
             
         \PY{c+c1}{\PYZsh{}print invSn }
         \PY{n}{yEstfin} \PY{o}{=} \PY{n}{PHI}\PY{o}{.}\PY{n}{dot}\PY{p}{(}\PY{n}{mn}\PY{p}{)}
         \PY{n}{plt}\PY{o}{.}\PY{n}{plot}\PY{p}{(}\PY{n}{xTr}\PY{p}{,}\PY{n}{GSRTr}\PY{p}{,}\PY{l+s+s1}{\PYZsq{}}\PY{l+s+s1}{or}\PY{l+s+s1}{\PYZsq{}}\PY{p}{)}
         \PY{n}{plt}\PY{o}{.}\PY{n}{plot}\PY{p}{(}\PY{n}{xTr}\PY{p}{,}\PY{n}{yEstfin}\PY{p}{,}\PY{l+s+s1}{\PYZsq{}}\PY{l+s+s1}{ok}\PY{l+s+s1}{\PYZsq{}}\PY{p}{)}
         
         \PY{n}{RMStrain} \PY{o}{=} \PY{n}{Erms}\PY{p}{(}\PY{n}{GSRTr}\PY{p}{,} \PY{n}{yEstfin}\PY{p}{)}
         \PY{k}{print} \PY{l+s+s2}{\PYZdq{}}\PY{l+s+s2}{Error Train}\PY{l+s+s2}{\PYZdq{}}\PY{p}{,} \PY{n}{RMStrain}
\end{Verbatim}


    \begin{Verbatim}[commandchars=\\\{\}]
(4570L, 100L)
(100L, 1L) (100L, 100L)
Error Train (0.19714707477102436-0.1907424892855899j)

    \end{Verbatim}

    \begin{center}
    \adjustimage{max size={0.9\linewidth}{0.9\paperheight}}{output_45_1.png}
    \end{center}
    { \hspace*{\fill} \\}
    
    El mayor problema que se logra percibir en este modelo es el ajuste del
alpha y beta, ya que toca hacerlo empiricamente, prueba tras prueba
ajustarlo y es un trabajo bastante arduo y dificil de hacer.

    \subsection{Analizando HR}\label{analizando-hr}

    Ahora continuaremos analizando la señal HR, se sigue partiendo la señal
en dos partes, una para entrenar nuestro modelo y la otra para
verificar, estas partes se dividiran en 70\% para entrenamiento y 30\%
para test como lo anteriormente visto. Pero se puede ver que esta señal
varia mas que la anteriormente analizad

\begin{itemize}
\tightlist
\item
  xTr = Representa el conjunto de las caracteristicas de entrenamiento
\item
  xTe = Representa el conjunto de las caracteristicas de testing
\item
  HRTr = Representa los datos recogidos de la respuesta galvanica de la
  piel para el entrenamiento
\item
  HRTe = Representa los datos recogidos de la respuesta galvanica de la
  piel para el testing
\end{itemize}

Graficamos los datos de entramiento y los de testing.

    \begin{Verbatim}[commandchars=\\\{\}]
{\color{incolor}In [{\color{incolor}16}]:} \PY{n}{plt}\PY{o}{.}\PY{n}{plot}\PY{p}{(}\PY{n}{x}\PY{p}{,}\PY{n}{HR}\PY{p}{,}\PY{l+s+s1}{\PYZsq{}}\PY{l+s+s1}{or}\PY{l+s+s1}{\PYZsq{}}\PY{p}{)}
         \PY{n}{plt}\PY{o}{.}\PY{n}{xlabel}\PY{p}{(}\PY{l+s+s1}{\PYZsq{}}\PY{l+s+s1}{x}\PY{l+s+s1}{\PYZsq{}}\PY{p}{)}
         \PY{n}{plt}\PY{o}{.}\PY{n}{legend}\PY{p}{(}\PY{p}{(}\PY{l+s+s1}{\PYZsq{}}\PY{l+s+s1}{Noise data (t)}\PY{l+s+s1}{\PYZsq{}}\PY{p}{,} \PY{l+s+s1}{\PYZsq{}}\PY{l+s+s1}{True data (y)}\PY{l+s+s1}{\PYZsq{}}\PY{p}{)}\PY{p}{)}
         
         \PY{n+nb}{id} \PY{o}{=} \PY{n}{np}\PY{o}{.}\PY{n}{random}\PY{o}{.}\PY{n}{permutation}\PY{p}{(}\PY{n}{N}\PY{p}{)}
         \PY{n}{perTrain} \PY{o}{=} \PY{l+m+mf}{0.7}
         \PY{n}{NTr} \PY{o}{=} \PY{n+nb}{int}\PY{p}{(}\PY{n+nb}{round}\PY{p}{(}\PY{n}{N}\PY{o}{*}\PY{n}{perTrain}\PY{p}{)}\PY{p}{)}
         \PY{n}{idTr} \PY{o}{=} \PY{n+nb}{id}\PY{p}{[}\PY{p}{:}\PY{n}{NTr}\PY{p}{]}
         \PY{n}{idTe} \PY{o}{=} \PY{n+nb}{id}\PY{p}{[}\PY{n}{NTr}\PY{p}{:}\PY{p}{]}
         \PY{n}{xTr} \PY{o}{=} \PY{n}{x}\PY{p}{[}\PY{n}{idTr}\PY{p}{]}
         \PY{n}{xTe} \PY{o}{=} \PY{n}{x}\PY{p}{[}\PY{n}{idTe}\PY{p}{]}
         \PY{n}{HRTr} \PY{o}{=} \PY{n}{HR}\PY{p}{[}\PY{n}{idTr}\PY{p}{]}
         \PY{n}{HRTe} \PY{o}{=} \PY{n}{HR}\PY{p}{[}\PY{n}{idTe}\PY{p}{]}
         
         \PY{n}{plt}\PY{o}{.}\PY{n}{plot}\PY{p}{(}\PY{n}{xTr}\PY{p}{,} \PY{n}{HRTr}\PY{p}{,} \PY{l+s+s1}{\PYZsq{}}\PY{l+s+s1}{or}\PY{l+s+s1}{\PYZsq{}}\PY{p}{)}
         \PY{n}{plt}\PY{o}{.}\PY{n}{xlabel}\PY{p}{(}\PY{l+s+s1}{\PYZsq{}}\PY{l+s+s1}{x}\PY{l+s+s1}{\PYZsq{}}\PY{p}{)}
         \PY{n}{HRTr}\PY{o}{.}\PY{n}{shape}
\end{Verbatim}


\begin{Verbatim}[commandchars=\\\{\}]
{\color{outcolor}Out[{\color{outcolor}16}]:} (4570L, 1L)
\end{Verbatim}
            
    \begin{center}
    \adjustimage{max size={0.9\linewidth}{0.9\paperheight}}{output_49_1.png}
    \end{center}
    { \hspace*{\fill} \\}
    
    \begin{Verbatim}[commandchars=\\\{\}]
{\color{incolor}In [{\color{incolor}17}]:} \PY{n}{plt}\PY{o}{.}\PY{n}{plot}\PY{p}{(}\PY{n}{xTe}\PY{p}{,} \PY{n}{HRTe}\PY{p}{,} \PY{l+s+s1}{\PYZsq{}}\PY{l+s+s1}{og}\PY{l+s+s1}{\PYZsq{}}\PY{p}{)}
         \PY{n}{plt}\PY{o}{.}\PY{n}{xlabel}\PY{p}{(}\PY{l+s+s1}{\PYZsq{}}\PY{l+s+s1}{x}\PY{l+s+s1}{\PYZsq{}}\PY{p}{)}
         \PY{n}{HRTe}\PY{o}{.}\PY{n}{shape}
\end{Verbatim}


\begin{Verbatim}[commandchars=\\\{\}]
{\color{outcolor}Out[{\color{outcolor}17}]:} (1958L, 1L)
\end{Verbatim}
            
    \begin{center}
    \adjustimage{max size={0.9\linewidth}{0.9\paperheight}}{output_50_1.png}
    \end{center}
    { \hspace*{\fill} \\}
    
    \subsection{Regresion con funciones bases
polinomial}\label{regresion-con-funciones-bases-polinomial}

    Volviendo a la regresion con funciones bases polinomial, pero hay mucho
error en nuestro modelo, pero se ajusta un poco a la señal, como en una
media entre las subidas y bajadas de la señal

    \begin{Verbatim}[commandchars=\\\{\}]
{\color{incolor}In [{\color{incolor}18}]:} \PY{n}{PHIbTr}\PY{p}{,}\PY{n}{w\PYZus{}ML}\PY{p}{,}\PY{n}{yEstimadoTr}\PY{p}{,} \PY{n}{s2} \PY{o}{=} \PY{n}{LS}\PY{p}{(}\PY{n}{xTr}\PY{p}{,}\PY{n}{HRTr}\PY{p}{,}\PY{l+s+s1}{\PYZsq{}}\PY{l+s+s1}{pol}\PY{l+s+s1}{\PYZsq{}}\PY{p}{,} \PY{l+m+mi}{30}\PY{p}{)}
         \PY{n}{PHIbTe}\PY{p}{,}\PY{n}{w\PYZus{}MLTe}\PY{p}{,}\PY{n}{yEstimadoTe}\PY{p}{,} \PY{n}{s2} \PY{o}{=} \PY{n}{LS}\PY{p}{(}\PY{n}{xTe}\PY{p}{,}\PY{n}{HRTe}\PY{p}{,}\PY{l+s+s1}{\PYZsq{}}\PY{l+s+s1}{pol}\PY{l+s+s1}{\PYZsq{}}\PY{p}{,} \PY{l+m+mi}{30}\PY{p}{)}
         \PY{n}{yEstimadoTe} \PY{o}{=} \PY{n}{PHIbTe}\PY{o}{.}\PY{n}{dot}\PY{p}{(}\PY{n}{w\PYZus{}ML}\PY{p}{)}
         \PY{n}{plt}\PY{o}{.}\PY{n}{plot}\PY{p}{(}\PY{n}{xTr}\PY{p}{,}\PY{n}{yEstimadoTr}\PY{p}{,} \PY{l+s+s1}{\PYZsq{}}\PY{l+s+s1}{og}\PY{l+s+s1}{\PYZsq{}}\PY{p}{)}
         \PY{n}{plt}\PY{o}{.}\PY{n}{plot}\PY{p}{(}\PY{n}{xTe}\PY{p}{,}\PY{n}{yEstimadoTe}\PY{p}{,} \PY{l+s+s1}{\PYZsq{}}\PY{l+s+s1}{or}\PY{l+s+s1}{\PYZsq{}}\PY{p}{)}
         \PY{n}{plt}\PY{o}{.}\PY{n}{plot}\PY{p}{(}\PY{n}{x}\PY{p}{,}\PY{n}{HR}\PY{p}{,} \PY{l+s+s1}{\PYZsq{}}\PY{l+s+s1}{\PYZhy{}k}\PY{l+s+s1}{\PYZsq{}}\PY{p}{)}
         
         \PY{n}{RMStest} \PY{o}{=} \PY{n}{Erms}\PY{p}{(}\PY{n}{HRTr}\PY{p}{,} \PY{n}{yEstimadoTr}\PY{p}{)}
         \PY{n}{RMStrain} \PY{o}{=} \PY{n}{Erms}\PY{p}{(}\PY{n}{HRTe}\PY{p}{,} \PY{n}{yEstimadoTe}\PY{p}{)}
         
         \PY{k}{print} \PY{n}{RMStest}\PY{p}{,} \PY{n}{RMStrain}
\end{Verbatim}


    \begin{Verbatim}[commandchars=\\\{\}]
3407.853662333171 3434.4630216585015

    \end{Verbatim}

    \begin{center}
    \adjustimage{max size={0.9\linewidth}{0.9\paperheight}}{output_53_1.png}
    \end{center}
    { \hspace*{\fill} \\}
    
    \subsubsection{Regresion con funciones bases
exponencial}\label{regresion-con-funciones-bases-exponencial}

    Con regresion lineal con funciones bases exponencial, por mas que se
subio la cantidad de funciones bases, no se logró modelar la señal,
dicha señal por ser tan variable en intervalos cortos es dificil de
ajustar con este tipo de modelo.

    \begin{Verbatim}[commandchars=\\\{\}]
{\color{incolor}In [{\color{incolor}19}]:} \PY{n}{PHIbTr}\PY{p}{,}\PY{n}{w\PYZus{}ML}\PY{p}{,}\PY{n}{yEstimadoTr}\PY{p}{,} \PY{n}{s2} \PY{o}{=} \PY{n}{LS}\PY{p}{(}\PY{n}{xTr}\PY{p}{,}\PY{n}{HRTr}\PY{p}{,}\PY{l+s+s1}{\PYZsq{}}\PY{l+s+s1}{exp}\PY{l+s+s1}{\PYZsq{}}\PY{p}{,}\PY{l+m+mi}{50}\PY{p}{)}
         \PY{n}{PHIbTe}\PY{p}{,}\PY{n}{w\PYZus{}MLTe}\PY{p}{,}\PY{n}{yEstimadoTe}\PY{p}{,} \PY{n}{s2} \PY{o}{=} \PY{n}{LS}\PY{p}{(}\PY{n}{xTe}\PY{p}{,}\PY{n}{HRTe}\PY{p}{,}\PY{l+s+s1}{\PYZsq{}}\PY{l+s+s1}{exp}\PY{l+s+s1}{\PYZsq{}}\PY{p}{,}\PY{l+m+mi}{50}\PY{p}{)}
         \PY{n}{yEstimadoTe} \PY{o}{=} \PY{n}{PHIbTe}\PY{o}{.}\PY{n}{dot}\PY{p}{(}\PY{n}{w\PYZus{}ML}\PY{p}{)}
         \PY{n}{plt}\PY{o}{.}\PY{n}{plot}\PY{p}{(}\PY{n}{xTr}\PY{p}{,}\PY{n}{yEstimadoTr}\PY{p}{,} \PY{l+s+s1}{\PYZsq{}}\PY{l+s+s1}{og}\PY{l+s+s1}{\PYZsq{}}\PY{p}{)}
         \PY{n}{plt}\PY{o}{.}\PY{n}{plot}\PY{p}{(}\PY{n}{xTe}\PY{p}{,}\PY{n}{yEstimadoTe}\PY{p}{,} \PY{l+s+s1}{\PYZsq{}}\PY{l+s+s1}{or}\PY{l+s+s1}{\PYZsq{}}\PY{p}{)}
         \PY{n}{plt}\PY{o}{.}\PY{n}{plot}\PY{p}{(}\PY{n}{x}\PY{p}{,}\PY{n}{HR}\PY{p}{,} \PY{l+s+s1}{\PYZsq{}}\PY{l+s+s1}{\PYZhy{}k}\PY{l+s+s1}{\PYZsq{}}\PY{p}{)}
         
         \PY{n}{RMStest} \PY{o}{=} \PY{n}{Erms}\PY{p}{(}\PY{n}{HRTr}\PY{p}{,} \PY{n}{yEstimadoTr}\PY{p}{)}
         \PY{n}{RMStrain} \PY{o}{=} \PY{n}{Erms}\PY{p}{(}\PY{n}{HRTe}\PY{p}{,} \PY{n}{yEstimadoTe}\PY{p}{)}
         
         \PY{k}{print} \PY{n}{RMStest}\PY{p}{,} \PY{n}{RMStrain}
\end{Verbatim}


    \begin{Verbatim}[commandchars=\\\{\}]
27260.902439489884 27234.4260128302

    \end{Verbatim}

    \begin{center}
    \adjustimage{max size={0.9\linewidth}{0.9\paperheight}}{output_56_1.png}
    \end{center}
    { \hspace*{\fill} \\}
    
    \subsubsection{Regresion con funciones bases
sigmoidal}\label{regresion-con-funciones-bases-sigmoidal}

    Con regresion lineal con funciones bases sigmoidal, esto no cambia con
respecto al analisis hecho con funciones bases exponenciales. Es muy
dificil al tener cambios tan brucos.

    \begin{Verbatim}[commandchars=\\\{\}]
{\color{incolor}In [{\color{incolor}20}]:} \PY{n}{PHIbTr}\PY{p}{,}\PY{n}{w\PYZus{}ML}\PY{p}{,}\PY{n}{yEstimadoTr}\PY{p}{,} \PY{n}{s2} \PY{o}{=} \PY{n}{LS}\PY{p}{(}\PY{n}{xTr}\PY{p}{,}\PY{n}{HRTr}\PY{p}{,}\PY{l+s+s1}{\PYZsq{}}\PY{l+s+s1}{sig}\PY{l+s+s1}{\PYZsq{}}\PY{p}{,}\PY{l+m+mi}{51}\PY{p}{)}
         \PY{n}{PHIbTe}\PY{p}{,}\PY{n}{w\PYZus{}MLTe}\PY{p}{,}\PY{n}{yEstimadoTe}\PY{p}{,} \PY{n}{s2} \PY{o}{=} \PY{n}{LS}\PY{p}{(}\PY{n}{xTe}\PY{p}{,}\PY{n}{HRTe}\PY{p}{,}\PY{l+s+s1}{\PYZsq{}}\PY{l+s+s1}{sig}\PY{l+s+s1}{\PYZsq{}}\PY{p}{,} \PY{l+m+mi}{51}\PY{p}{)}
         \PY{n}{yEstimadoTe} \PY{o}{=} \PY{n}{PHIbTe}\PY{o}{.}\PY{n}{dot}\PY{p}{(}\PY{n}{w\PYZus{}ML}\PY{p}{)}
         \PY{n}{plt}\PY{o}{.}\PY{n}{plot}\PY{p}{(}\PY{n}{xTr}\PY{p}{,}\PY{n}{yEstimadoTr}\PY{p}{,} \PY{l+s+s1}{\PYZsq{}}\PY{l+s+s1}{og}\PY{l+s+s1}{\PYZsq{}}\PY{p}{)}
         \PY{n}{plt}\PY{o}{.}\PY{n}{plot}\PY{p}{(}\PY{n}{xTe}\PY{p}{,}\PY{n}{yEstimadoTe}\PY{p}{,} \PY{l+s+s1}{\PYZsq{}}\PY{l+s+s1}{or}\PY{l+s+s1}{\PYZsq{}}\PY{p}{)}
         \PY{n}{plt}\PY{o}{.}\PY{n}{plot}\PY{p}{(}\PY{n}{x}\PY{p}{,}\PY{n}{HR}\PY{p}{,} \PY{l+s+s1}{\PYZsq{}}\PY{l+s+s1}{\PYZhy{}k}\PY{l+s+s1}{\PYZsq{}}\PY{p}{)}
         
         \PY{n}{RMStest} \PY{o}{=} \PY{n}{Erms}\PY{p}{(}\PY{n}{HRTr}\PY{p}{,} \PY{n}{yEstimadoTr}\PY{p}{)}
         \PY{n}{RMStrain} \PY{o}{=} \PY{n}{Erms}\PY{p}{(}\PY{n}{HRTe}\PY{p}{,} \PY{n}{yEstimadoTe}\PY{p}{)}
         
         \PY{k}{print} \PY{n}{RMStest}\PY{p}{,} \PY{n}{RMStrain}
\end{Verbatim}


    \begin{Verbatim}[commandchars=\\\{\}]
1257.7437579562497 1247.540371769818

    \end{Verbatim}

    \begin{center}
    \adjustimage{max size={0.9\linewidth}{0.9\paperheight}}{output_59_1.png}
    \end{center}
    { \hspace*{\fill} \\}
    
    \subsubsection{Regresion con regularización con funciones bases
polinomial}\label{regresion-con-regularizaciuxf3n-con-funciones-bases-polinomial}

    Con regresion lineal con regularización, como nuestra señal HR tiene
unos cambios tan bruscos, con las funciones bases polinomial no logra
describirla a un alto grado de eficiencia. No logra modelar los cambios
tan fuertes vistos en la señal.

    \begin{Verbatim}[commandchars=\\\{\}]
{\color{incolor}In [{\color{incolor}21}]:} \PY{n}{lambdaI} \PY{o}{=} \PY{n}{math}\PY{o}{.}\PY{n}{exp}\PY{p}{(}\PY{o}{\PYZhy{}}\PY{l+m+mf}{18.0}\PY{p}{)}
         \PY{n}{PHIbTr}\PY{p}{,}\PY{n}{w\PYZus{}MLReg}\PY{p}{,}\PY{n}{yEstimadoTr} \PY{o}{=} \PY{n}{LS\PYZus{}Reg}\PY{p}{(}\PY{n}{xTr}\PY{p}{,}\PY{n}{HRTr}\PY{p}{,}\PY{l+s+s1}{\PYZsq{}}\PY{l+s+s1}{pol}\PY{l+s+s1}{\PYZsq{}}\PY{p}{,}\PY{l+m+mi}{150}\PY{p}{,} \PY{n}{lambdaI}\PY{p}{)}
         \PY{n}{PHIbTe}\PY{p}{,}\PY{n}{w\PYZus{}MLRegTe}\PY{p}{,}\PY{n}{yEstimadoTe} \PY{o}{=} \PY{n}{LS\PYZus{}Reg}\PY{p}{(}\PY{n}{xTe}\PY{p}{,}\PY{n}{HRTe}\PY{p}{,}\PY{l+s+s1}{\PYZsq{}}\PY{l+s+s1}{pol}\PY{l+s+s1}{\PYZsq{}}\PY{p}{,}\PY{l+m+mi}{150}\PY{p}{,} \PY{n}{lambdaI}\PY{p}{)}
         \PY{n}{yEstimadoTe} \PY{o}{=} \PY{n}{PHIbTe}\PY{o}{.}\PY{n}{dot}\PY{p}{(}\PY{n}{w\PYZus{}MLReg}\PY{p}{)}
         \PY{n}{plt}\PY{o}{.}\PY{n}{plot}\PY{p}{(}\PY{n}{xTr}\PY{p}{,}\PY{n}{yEstimadoTr}\PY{p}{,} \PY{l+s+s1}{\PYZsq{}}\PY{l+s+s1}{og}\PY{l+s+s1}{\PYZsq{}}\PY{p}{)}
         \PY{n}{plt}\PY{o}{.}\PY{n}{plot}\PY{p}{(}\PY{n}{xTe}\PY{p}{,}\PY{n}{yEstimadoTe}\PY{p}{,} \PY{l+s+s1}{\PYZsq{}}\PY{l+s+s1}{or}\PY{l+s+s1}{\PYZsq{}}\PY{p}{)}
         \PY{n}{plt}\PY{o}{.}\PY{n}{plot}\PY{p}{(}\PY{n}{x}\PY{p}{,}\PY{n}{HR}\PY{p}{,} \PY{l+s+s1}{\PYZsq{}}\PY{l+s+s1}{\PYZhy{}k}\PY{l+s+s1}{\PYZsq{}}\PY{p}{)}
         
         \PY{n}{RMStest} \PY{o}{=} \PY{n}{Erms}\PY{p}{(}\PY{n}{HRTr}\PY{p}{,} \PY{n}{yEstimadoTr}\PY{p}{)}
         \PY{n}{RMStrain} \PY{o}{=} \PY{n}{Erms}\PY{p}{(}\PY{n}{HRTe}\PY{p}{,} \PY{n}{yEstimadoTe}\PY{p}{)}
         
         \PY{k}{print} \PY{n}{RMStest}\PY{p}{,} \PY{n}{RMStrain}
\end{Verbatim}


    \begin{Verbatim}[commandchars=\\\{\}]
824.402883777935 812.9624119339887

    \end{Verbatim}

    \begin{center}
    \adjustimage{max size={0.9\linewidth}{0.9\paperheight}}{output_62_1.png}
    \end{center}
    { \hspace*{\fill} \\}
    
    \subsection{Regresión con regularización con funciones
exponenciales}\label{regresiuxf3n-con-regularizaciuxf3n-con-funciones-exponenciales}

    Con funciones exponenciales tiene un mayor error que que con funciones
bases polinomiales, igual es muy dificil con este modelo modelar dicha
funcion por lo anteriormente expuesto.

    \begin{Verbatim}[commandchars=\\\{\}]
{\color{incolor}In [{\color{incolor}22}]:} \PY{n}{lambdaI} \PY{o}{=} \PY{n}{math}\PY{o}{.}\PY{n}{exp}\PY{p}{(}\PY{o}{\PYZhy{}}\PY{l+m+mf}{18.0}\PY{p}{)}
         \PY{n}{PHIbTr}\PY{p}{,}\PY{n}{w\PYZus{}MLReg}\PY{p}{,}\PY{n}{yEstimadoTr} \PY{o}{=} \PY{n}{LS\PYZus{}Reg}\PY{p}{(}\PY{n}{xTr}\PY{p}{,}\PY{n}{HRTr}\PY{p}{,}\PY{l+s+s1}{\PYZsq{}}\PY{l+s+s1}{exp}\PY{l+s+s1}{\PYZsq{}}\PY{p}{,}\PY{l+m+mi}{150}\PY{p}{,} \PY{n}{lambdaI}\PY{p}{)}
         \PY{n}{PHIbTe}\PY{p}{,}\PY{n}{w\PYZus{}MLRegTe}\PY{p}{,}\PY{n}{yEstimadoTe} \PY{o}{=} \PY{n}{LS\PYZus{}Reg}\PY{p}{(}\PY{n}{xTe}\PY{p}{,}\PY{n}{HRTe}\PY{p}{,}\PY{l+s+s1}{\PYZsq{}}\PY{l+s+s1}{exp}\PY{l+s+s1}{\PYZsq{}}\PY{p}{,}\PY{l+m+mi}{150}\PY{p}{,} \PY{n}{lambdaI}\PY{p}{)}
         \PY{n}{yEstimadoTe} \PY{o}{=} \PY{n}{PHIbTe}\PY{o}{.}\PY{n}{dot}\PY{p}{(}\PY{n}{w\PYZus{}MLReg}\PY{p}{)}
         \PY{n}{plt}\PY{o}{.}\PY{n}{plot}\PY{p}{(}\PY{n}{xTr}\PY{p}{,}\PY{n}{yEstimadoTr}\PY{p}{,} \PY{l+s+s1}{\PYZsq{}}\PY{l+s+s1}{og}\PY{l+s+s1}{\PYZsq{}}\PY{p}{)}
         \PY{n}{plt}\PY{o}{.}\PY{n}{plot}\PY{p}{(}\PY{n}{xTe}\PY{p}{,}\PY{n}{yEstimadoTe}\PY{p}{,} \PY{l+s+s1}{\PYZsq{}}\PY{l+s+s1}{or}\PY{l+s+s1}{\PYZsq{}}\PY{p}{)}
         \PY{n}{plt}\PY{o}{.}\PY{n}{plot}\PY{p}{(}\PY{n}{x}\PY{p}{,}\PY{n}{HR}\PY{p}{,} \PY{l+s+s1}{\PYZsq{}}\PY{l+s+s1}{\PYZhy{}k}\PY{l+s+s1}{\PYZsq{}}\PY{p}{)}
         
         \PY{n}{RMStest} \PY{o}{=} \PY{n}{Erms}\PY{p}{(}\PY{n}{HRTr}\PY{p}{,} \PY{n}{yEstimadoTr}\PY{p}{)}
         \PY{n}{RMStrain} \PY{o}{=} \PY{n}{Erms}\PY{p}{(}\PY{n}{HRTe}\PY{p}{,} \PY{n}{yEstimadoTe}\PY{p}{)}
         
         \PY{k}{print} \PY{n}{RMStest}\PY{p}{,} \PY{n}{RMStrain}
\end{Verbatim}


    \begin{Verbatim}[commandchars=\\\{\}]
847.7819412908274 837.5440816191737

    \end{Verbatim}

    \begin{center}
    \adjustimage{max size={0.9\linewidth}{0.9\paperheight}}{output_65_1.png}
    \end{center}
    { \hspace*{\fill} \\}
    
    \subsection{Regresión con regularización con funciones
sigmoidales}\label{regresiuxf3n-con-regularizaciuxf3n-con-funciones-sigmoidales}

    Pero de las tres pruebas esta es la que mas contiene error, ya que la
funcion sigmoidal, es la funcion mas suave a la hora de la curvatura, y
dicha señal que tenemos tiene cambios muy bruscos en pequeños
intervalos.

    \begin{Verbatim}[commandchars=\\\{\}]
{\color{incolor}In [{\color{incolor}23}]:} \PY{n}{lambdaI} \PY{o}{=} \PY{n}{math}\PY{o}{.}\PY{n}{exp}\PY{p}{(}\PY{o}{\PYZhy{}}\PY{l+m+mf}{2.0}\PY{p}{)}
         \PY{n}{PHIbTr}\PY{p}{,}\PY{n}{w\PYZus{}MLReg}\PY{p}{,}\PY{n}{yEstimadoTr} \PY{o}{=} \PY{n}{LS\PYZus{}Reg}\PY{p}{(}\PY{n}{xTr}\PY{p}{,}\PY{n}{HRTr}\PY{p}{,}\PY{l+s+s1}{\PYZsq{}}\PY{l+s+s1}{sig}\PY{l+s+s1}{\PYZsq{}}\PY{p}{,}\PY{l+m+mi}{150}\PY{p}{,} \PY{n}{lambdaI}\PY{p}{)}
         \PY{n}{PHIbTe}\PY{p}{,}\PY{n}{w\PYZus{}MLRegTe}\PY{p}{,}\PY{n}{yEstimadoTe} \PY{o}{=} \PY{n}{LS\PYZus{}Reg}\PY{p}{(}\PY{n}{xTe}\PY{p}{,}\PY{n}{HRTe}\PY{p}{,}\PY{l+s+s1}{\PYZsq{}}\PY{l+s+s1}{sig}\PY{l+s+s1}{\PYZsq{}}\PY{p}{,}\PY{l+m+mi}{150}\PY{p}{,} \PY{n}{lambdaI}\PY{p}{)}
         \PY{n}{yEstimadoTe} \PY{o}{=} \PY{n}{PHIbTe}\PY{o}{.}\PY{n}{dot}\PY{p}{(}\PY{n}{w\PYZus{}MLReg}\PY{p}{)}
         \PY{n}{plt}\PY{o}{.}\PY{n}{plot}\PY{p}{(}\PY{n}{xTr}\PY{p}{,}\PY{n}{yEstimadoTr}\PY{p}{,} \PY{l+s+s1}{\PYZsq{}}\PY{l+s+s1}{og}\PY{l+s+s1}{\PYZsq{}}\PY{p}{)}
         \PY{n}{plt}\PY{o}{.}\PY{n}{plot}\PY{p}{(}\PY{n}{xTe}\PY{p}{,}\PY{n}{yEstimadoTe}\PY{p}{,} \PY{l+s+s1}{\PYZsq{}}\PY{l+s+s1}{or}\PY{l+s+s1}{\PYZsq{}}\PY{p}{)}
         \PY{n}{plt}\PY{o}{.}\PY{n}{plot}\PY{p}{(}\PY{n}{x}\PY{p}{,}\PY{n}{HR}\PY{p}{,} \PY{l+s+s1}{\PYZsq{}}\PY{l+s+s1}{\PYZhy{}k}\PY{l+s+s1}{\PYZsq{}}\PY{p}{)}
         
         \PY{n}{RMStest} \PY{o}{=} \PY{n}{Erms}\PY{p}{(}\PY{n}{HRTr}\PY{p}{,} \PY{n}{yEstimadoTr}\PY{p}{)}
         \PY{n}{RMStrain} \PY{o}{=} \PY{n}{Erms}\PY{p}{(}\PY{n}{HRTe}\PY{p}{,} \PY{n}{yEstimadoTe}\PY{p}{)}
         
         \PY{k}{print} \PY{n}{RMStest}\PY{p}{,} \PY{n}{RMStrain}
\end{Verbatim}


    \begin{Verbatim}[commandchars=\\\{\}]
890.3697257524021 876.9950709731642

    \end{Verbatim}

    \begin{center}
    \adjustimage{max size={0.9\linewidth}{0.9\paperheight}}{output_68_1.png}
    \end{center}
    { \hspace*{\fill} \\}
    
    \subsection{Regresión bayesiana con funciones base
polinomial}\label{regresiuxf3n-bayesiana-con-funciones-base-polinomial}

    Con regresion bayesiana y funciones bases polinomiales, nuestro modelo
no logra describir la funcion HR a la perfeccion, el modelo tiene un
alto grado de error. Como se puede observar en la grafica siguiente.

    \begin{Verbatim}[commandchars=\\\{\}]
{\color{incolor}In [{\color{incolor}24}]:} \PY{n}{M} \PY{o}{=} \PY{l+m+mi}{100} \PY{c+c1}{\PYZsh{} numero de funciones base}
         \PY{n}{PHI}\PY{p}{,}\PY{n}{w\PYZus{}MLReg}\PY{p}{,}\PY{n}{yEstimadoTr}\PY{p}{,} \PY{n}{s2} \PY{o}{=} \PY{n}{LS}\PY{p}{(}\PY{n}{xTr}\PY{p}{,}\PY{n}{HRTr}\PY{p}{,}\PY{l+s+s1}{\PYZsq{}}\PY{l+s+s1}{pol}\PY{l+s+s1}{\PYZsq{}}\PY{p}{,}\PY{n}{M}\PY{o}{\PYZhy{}}\PY{l+m+mi}{1}\PY{p}{)}
         \PY{k}{print} \PY{n}{PHI}\PY{o}{.}\PY{n}{shape}
         
         \PY{n}{iT} \PY{o}{=} \PY{l+m+mi}{200} \PY{c+c1}{\PYZsh{} Numero de iteraciones}
         \PY{n}{alpha} \PY{o}{=} \PY{l+m+mf}{0.1}
         \PY{n}{beta} \PY{o}{=} \PY{l+m+mf}{1.0}
         \PY{n}{invbeta} \PY{o}{=} \PY{l+m+mi}{1}\PY{o}{/}\PY{n}{beta}
         \PY{n}{PHIT} \PY{o}{=} \PY{n}{PHI}\PY{o}{.}\PY{n}{T}
         \PY{n}{invSn} \PY{o}{=} \PY{n}{alpha}\PY{o}{*}\PY{n}{np}\PY{o}{.}\PY{n}{eye}\PY{p}{(}\PY{n}{M}\PY{p}{)}\PY{o}{+} \PY{n}{beta}\PY{o}{*}\PY{n}{PHIT}\PY{o}{.}\PY{n}{dot}\PY{p}{(}\PY{n}{PHI}\PY{p}{)}
         \PY{n}{Sn} \PY{o}{=} \PY{n}{np}\PY{o}{.}\PY{n}{linalg}\PY{o}{.}\PY{n}{inv}\PY{p}{(}\PY{n}{invSn}\PY{p}{)}
         \PY{n}{mn} \PY{o}{=} \PY{n}{beta}\PY{o}{*}\PY{n}{Sn}\PY{o}{.}\PY{n}{dot}\PY{p}{(}\PY{n}{PHIT}\PY{o}{.}\PY{n}{dot}\PY{p}{(}\PY{n}{HRTr}\PY{p}{)}\PY{p}{)}
         \PY{k}{print} \PY{n}{mn}\PY{o}{.}\PY{n}{shape}\PY{p}{,}\PY{n}{Sn}\PY{o}{.}\PY{n}{shape}
         
         \PY{c+c1}{\PYZsh{} Probemos la estimacion con el mn inicial}
         \PY{n}{yEst} \PY{o}{=} \PY{n}{PHI}\PY{o}{.}\PY{n}{dot}\PY{p}{(}\PY{n}{mn}\PY{p}{)}
         \PY{n}{lambdaIp}\PY{p}{,}\PY{n}{vecI} \PY{o}{=} \PY{n}{np}\PY{o}{.}\PY{n}{linalg}\PY{o}{.}\PY{n}{eig}\PY{p}{(}\PY{n}{PHIT}\PY{o}{.}\PY{n}{dot}\PY{p}{(}\PY{n}{PHI}\PY{p}{)}\PY{p}{)}
         \PY{n}{lambdaI} \PY{o}{=} \PY{n}{beta}\PY{o}{*}\PY{n}{lambdaIp}
         \PY{k}{for} \PY{n}{j} \PY{o+ow}{in} \PY{n+nb}{range}\PY{p}{(}\PY{l+m+mi}{0}\PY{p}{,}\PY{n}{iT}\PY{p}{)}\PY{p}{:}
             \PY{n}{gamma} \PY{o}{=} \PY{n}{np}\PY{o}{.}\PY{n}{sum}\PY{p}{(}\PY{n}{lambdaI}\PY{o}{/}\PY{p}{(}\PY{n}{alpha}\PY{o}{*}\PY{n}{np}\PY{o}{.}\PY{n}{ones}\PY{p}{(}\PY{n}{lambdaI}\PY{o}{.}\PY{n}{shape}\PY{p}{)}\PY{o}{+}\PY{n}{lambdaI}\PY{p}{)}\PY{p}{)}
             \PY{n}{alpha} \PY{o}{=} \PY{n}{gamma}\PY{o}{/}\PY{p}{(}\PY{p}{(}\PY{n}{mn}\PY{o}{.}\PY{n}{T}\PY{p}{)}\PY{o}{.}\PY{n}{dot}\PY{p}{(}\PY{n}{mn}\PY{p}{)}\PY{p}{)}
             \PY{n}{invbeta} \PY{o}{=} \PY{p}{(}\PY{l+m+mi}{1}\PY{o}{/}\PY{p}{(}\PY{n}{N}\PY{o}{\PYZhy{}}\PY{n}{gamma}\PY{p}{)}\PY{p}{)}\PY{o}{*}\PY{p}{(}\PY{p}{(}\PY{p}{(}\PY{n}{HRTr}\PY{o}{\PYZhy{}}\PY{n}{PHI}\PY{o}{.}\PY{n}{dot}\PY{p}{(}\PY{n}{mn}\PY{p}{)}\PY{p}{)}\PY{o}{.}\PY{n}{T}\PY{p}{)}\PY{o}{.}\PY{n}{dot}\PY{p}{(}\PY{n}{HRTr}\PY{o}{\PYZhy{}}\PY{n}{PHI}\PY{o}{.}\PY{n}{dot}\PY{p}{(}\PY{n}{mn}\PY{p}{)}\PY{p}{)}\PY{p}{)}
             \PY{n}{beta} \PY{o}{=} \PY{l+m+mi}{1}\PY{o}{/}\PY{n}{invbeta}
             \PY{n}{lambdaI} \PY{o}{=} \PY{n}{beta}\PY{o}{*}\PY{n}{lambdaIp}
             \PY{n}{invSn} \PY{o}{=} \PY{n}{alpha}\PY{o}{*}\PY{n}{np}\PY{o}{.}\PY{n}{eye}\PY{p}{(}\PY{n}{M}\PY{p}{)}\PY{o}{+} \PY{n}{beta}\PY{o}{*}\PY{n}{PHIT}\PY{o}{.}\PY{n}{dot}\PY{p}{(}\PY{n}{PHI}\PY{p}{)}
             \PY{n}{Sn} \PY{o}{=} \PY{n}{np}\PY{o}{.}\PY{n}{linalg}\PY{o}{.}\PY{n}{inv}\PY{p}{(}\PY{n}{invSn}\PY{p}{)}
             \PY{n}{mn} \PY{o}{=} \PY{n}{beta}\PY{o}{*}\PY{n}{Sn}\PY{o}{.}\PY{n}{dot}\PY{p}{(}\PY{n}{PHIT}\PY{o}{.}\PY{n}{dot}\PY{p}{(}\PY{n}{HRTr}\PY{p}{)}\PY{p}{)}
             
         \PY{c+c1}{\PYZsh{}print invSn }
         \PY{n}{yEstfin} \PY{o}{=} \PY{n}{PHI}\PY{o}{.}\PY{n}{dot}\PY{p}{(}\PY{n}{mn}\PY{p}{)}
         \PY{n}{plt}\PY{o}{.}\PY{n}{plot}\PY{p}{(}\PY{n}{xTr}\PY{p}{,}\PY{n}{HRTr}\PY{p}{,}\PY{l+s+s1}{\PYZsq{}}\PY{l+s+s1}{or}\PY{l+s+s1}{\PYZsq{}}\PY{p}{)}
         \PY{n}{plt}\PY{o}{.}\PY{n}{plot}\PY{p}{(}\PY{n}{xTr}\PY{p}{,}\PY{n}{yEstfin}\PY{p}{,}\PY{l+s+s1}{\PYZsq{}}\PY{l+s+s1}{ok}\PY{l+s+s1}{\PYZsq{}}\PY{p}{)}
\end{Verbatim}


    \begin{Verbatim}[commandchars=\\\{\}]
(4570L, 100L)
(100L, 1L) (100L, 100L)

    \end{Verbatim}

\begin{Verbatim}[commandchars=\\\{\}]
{\color{outcolor}Out[{\color{outcolor}24}]:} [<matplotlib.lines.Line2D at 0x9d7e7f0>]
\end{Verbatim}
            
    \begin{center}
    \adjustimage{max size={0.9\linewidth}{0.9\paperheight}}{output_71_2.png}
    \end{center}
    { \hspace*{\fill} \\}
    
    \subsection{Regresión bayesiana con funciones base
exponencial}\label{regresiuxf3n-bayesiana-con-funciones-base-exponencial}

    Con regresion bayesiana con funciones base exponencial, se ajusta mas a
nuestra funcion HR pero como la misma es tan variable, no se logra
modelar con un alto grado de acierto.

    \begin{Verbatim}[commandchars=\\\{\}]
{\color{incolor}In [{\color{incolor}25}]:} \PY{n}{M} \PY{o}{=} \PY{l+m+mi}{100} \PY{c+c1}{\PYZsh{} numero de funciones base}
         \PY{n}{PHI}\PY{p}{,}\PY{n}{w\PYZus{}MLReg}\PY{p}{,}\PY{n}{yEstimadoTr}\PY{p}{,} \PY{n}{s2} \PY{o}{=} \PY{n}{LS}\PY{p}{(}\PY{n}{xTr}\PY{p}{,}\PY{n}{HRTr}\PY{p}{,}\PY{l+s+s1}{\PYZsq{}}\PY{l+s+s1}{exp}\PY{l+s+s1}{\PYZsq{}}\PY{p}{,}\PY{n}{M}\PY{o}{\PYZhy{}}\PY{l+m+mi}{1}\PY{p}{)}
         \PY{k}{print} \PY{n}{PHI}\PY{o}{.}\PY{n}{shape}
         
         \PY{n}{iT} \PY{o}{=} \PY{l+m+mi}{200} \PY{c+c1}{\PYZsh{} Numero de iteraciones}
         \PY{n}{alpha} \PY{o}{=} \PY{l+m+mf}{0.8}
         \PY{n}{beta} \PY{o}{=} \PY{l+m+mf}{2.7}
         \PY{n}{invbeta} \PY{o}{=} \PY{l+m+mi}{1}\PY{o}{/}\PY{n}{beta}
         \PY{n}{PHIT} \PY{o}{=} \PY{n}{PHI}\PY{o}{.}\PY{n}{T}
         \PY{n}{invSn} \PY{o}{=} \PY{n}{alpha}\PY{o}{*}\PY{n}{np}\PY{o}{.}\PY{n}{eye}\PY{p}{(}\PY{n}{M}\PY{p}{)}\PY{o}{+} \PY{n}{beta}\PY{o}{*}\PY{n}{PHIT}\PY{o}{.}\PY{n}{dot}\PY{p}{(}\PY{n}{PHI}\PY{p}{)}
         \PY{n}{Sn} \PY{o}{=} \PY{n}{np}\PY{o}{.}\PY{n}{linalg}\PY{o}{.}\PY{n}{inv}\PY{p}{(}\PY{n}{invSn}\PY{p}{)}
         \PY{n}{mn} \PY{o}{=} \PY{n}{beta}\PY{o}{*}\PY{n}{Sn}\PY{o}{.}\PY{n}{dot}\PY{p}{(}\PY{n}{PHIT}\PY{o}{.}\PY{n}{dot}\PY{p}{(}\PY{n}{HRTr}\PY{p}{)}\PY{p}{)}
         \PY{k}{print} \PY{n}{mn}\PY{o}{.}\PY{n}{shape}\PY{p}{,}\PY{n}{Sn}\PY{o}{.}\PY{n}{shape}
         
         \PY{c+c1}{\PYZsh{} Probemos la estimacion con el mn inicial}
         \PY{n}{yEst} \PY{o}{=} \PY{n}{PHI}\PY{o}{.}\PY{n}{dot}\PY{p}{(}\PY{n}{mn}\PY{p}{)}
         \PY{n}{lambdaIp}\PY{p}{,}\PY{n}{vecI} \PY{o}{=} \PY{n}{np}\PY{o}{.}\PY{n}{linalg}\PY{o}{.}\PY{n}{eig}\PY{p}{(}\PY{n}{PHIT}\PY{o}{.}\PY{n}{dot}\PY{p}{(}\PY{n}{PHI}\PY{p}{)}\PY{p}{)}
         \PY{n}{lambdaI} \PY{o}{=} \PY{n}{beta}\PY{o}{*}\PY{n}{lambdaIp}
         \PY{k}{for} \PY{n}{j} \PY{o+ow}{in} \PY{n+nb}{range}\PY{p}{(}\PY{l+m+mi}{0}\PY{p}{,}\PY{n}{iT}\PY{p}{)}\PY{p}{:}
             \PY{n}{gamma} \PY{o}{=} \PY{n}{np}\PY{o}{.}\PY{n}{sum}\PY{p}{(}\PY{n}{lambdaI}\PY{o}{/}\PY{p}{(}\PY{n}{alpha}\PY{o}{*}\PY{n}{np}\PY{o}{.}\PY{n}{ones}\PY{p}{(}\PY{n}{lambdaI}\PY{o}{.}\PY{n}{shape}\PY{p}{)}\PY{o}{+}\PY{n}{lambdaI}\PY{p}{)}\PY{p}{)}
             \PY{n}{alpha} \PY{o}{=} \PY{n}{gamma}\PY{o}{/}\PY{p}{(}\PY{p}{(}\PY{n}{mn}\PY{o}{.}\PY{n}{T}\PY{p}{)}\PY{o}{.}\PY{n}{dot}\PY{p}{(}\PY{n}{mn}\PY{p}{)}\PY{p}{)}
             \PY{n}{invbeta} \PY{o}{=} \PY{p}{(}\PY{l+m+mi}{1}\PY{o}{/}\PY{p}{(}\PY{n}{N}\PY{o}{\PYZhy{}}\PY{n}{gamma}\PY{p}{)}\PY{p}{)}\PY{o}{*}\PY{p}{(}\PY{p}{(}\PY{p}{(}\PY{n}{HRTr}\PY{o}{\PYZhy{}}\PY{n}{PHI}\PY{o}{.}\PY{n}{dot}\PY{p}{(}\PY{n}{mn}\PY{p}{)}\PY{p}{)}\PY{o}{.}\PY{n}{T}\PY{p}{)}\PY{o}{.}\PY{n}{dot}\PY{p}{(}\PY{n}{HRTr}\PY{o}{\PYZhy{}}\PY{n}{PHI}\PY{o}{.}\PY{n}{dot}\PY{p}{(}\PY{n}{mn}\PY{p}{)}\PY{p}{)}\PY{p}{)}
             \PY{n}{beta} \PY{o}{=} \PY{l+m+mi}{1}\PY{o}{/}\PY{n}{invbeta}
             \PY{n}{lambdaI} \PY{o}{=} \PY{n}{beta}\PY{o}{*}\PY{n}{lambdaIp}
             \PY{n}{invSn} \PY{o}{=} \PY{n}{alpha}\PY{o}{*}\PY{n}{np}\PY{o}{.}\PY{n}{eye}\PY{p}{(}\PY{n}{M}\PY{p}{)}\PY{o}{+} \PY{n}{beta}\PY{o}{*}\PY{n}{PHIT}\PY{o}{.}\PY{n}{dot}\PY{p}{(}\PY{n}{PHI}\PY{p}{)}
             \PY{n}{Sn} \PY{o}{=} \PY{n}{np}\PY{o}{.}\PY{n}{linalg}\PY{o}{.}\PY{n}{inv}\PY{p}{(}\PY{n}{invSn}\PY{p}{)}
             \PY{n}{mn} \PY{o}{=} \PY{n}{beta}\PY{o}{*}\PY{n}{Sn}\PY{o}{.}\PY{n}{dot}\PY{p}{(}\PY{n}{PHIT}\PY{o}{.}\PY{n}{dot}\PY{p}{(}\PY{n}{HRTr}\PY{p}{)}\PY{p}{)}
             
         \PY{c+c1}{\PYZsh{}print invSn }
         \PY{n}{yEstfin} \PY{o}{=} \PY{n}{PHI}\PY{o}{.}\PY{n}{dot}\PY{p}{(}\PY{n}{mn}\PY{p}{)}
         \PY{n}{plt}\PY{o}{.}\PY{n}{plot}\PY{p}{(}\PY{n}{xTr}\PY{p}{,}\PY{n}{HRTr}\PY{p}{,}\PY{l+s+s1}{\PYZsq{}}\PY{l+s+s1}{or}\PY{l+s+s1}{\PYZsq{}}\PY{p}{)}
         \PY{n}{plt}\PY{o}{.}\PY{n}{plot}\PY{p}{(}\PY{n}{xTr}\PY{p}{,}\PY{n}{yEstfin}\PY{p}{,}\PY{l+s+s1}{\PYZsq{}}\PY{l+s+s1}{ok}\PY{l+s+s1}{\PYZsq{}}\PY{p}{)}
\end{Verbatim}


    \begin{Verbatim}[commandchars=\\\{\}]
(4570L, 100L)
(100L, 1L) (100L, 100L)

    \end{Verbatim}

\begin{Verbatim}[commandchars=\\\{\}]
{\color{outcolor}Out[{\color{outcolor}25}]:} [<matplotlib.lines.Line2D at 0x9d7e2e8>]
\end{Verbatim}
            
    \begin{center}
    \adjustimage{max size={0.9\linewidth}{0.9\paperheight}}{output_74_2.png}
    \end{center}
    { \hspace*{\fill} \\}
    
    \subsection{Regresión bayesiana con funciones base
sigmoidal}\label{regresiuxf3n-bayesiana-con-funciones-base-sigmoidal}

    Y con funciones base sigmoidal, lo mismo ocurre que con exponencial.
Esta funcion es muy dificil describirla con este tipo de regresion al
ser tan variables en un intervalo tan pequeño.

    \begin{Verbatim}[commandchars=\\\{\}]
{\color{incolor}In [{\color{incolor}26}]:} \PY{n}{M} \PY{o}{=} \PY{l+m+mi}{100} \PY{c+c1}{\PYZsh{} numero de funciones base}
         \PY{n}{PHI}\PY{p}{,}\PY{n}{w\PYZus{}MLReg}\PY{p}{,}\PY{n}{yEstimadoTr}\PY{p}{,} \PY{n}{s2} \PY{o}{=} \PY{n}{LS}\PY{p}{(}\PY{n}{xTr}\PY{p}{,}\PY{n}{HRTr}\PY{p}{,}\PY{l+s+s1}{\PYZsq{}}\PY{l+s+s1}{sig}\PY{l+s+s1}{\PYZsq{}}\PY{p}{,}\PY{n}{M}\PY{o}{\PYZhy{}}\PY{l+m+mi}{1}\PY{p}{)}
         \PY{k}{print} \PY{n}{PHI}\PY{o}{.}\PY{n}{shape}
         
         \PY{n}{iT} \PY{o}{=} \PY{l+m+mi}{200} \PY{c+c1}{\PYZsh{} Numero de iteraciones}
         \PY{n}{alpha} \PY{o}{=} \PY{l+m+mf}{0.2}
         \PY{n}{beta} \PY{o}{=} \PY{l+m+mf}{0.5}
         \PY{n}{invbeta} \PY{o}{=} \PY{l+m+mi}{1}\PY{o}{/}\PY{n}{beta}
         \PY{n}{PHIT} \PY{o}{=} \PY{n}{PHI}\PY{o}{.}\PY{n}{T}
         \PY{n}{invSn} \PY{o}{=} \PY{n}{alpha}\PY{o}{*}\PY{n}{np}\PY{o}{.}\PY{n}{eye}\PY{p}{(}\PY{n}{M}\PY{p}{)}\PY{o}{+} \PY{n}{beta}\PY{o}{*}\PY{n}{PHIT}\PY{o}{.}\PY{n}{dot}\PY{p}{(}\PY{n}{PHI}\PY{p}{)}
         \PY{n}{Sn} \PY{o}{=} \PY{n}{np}\PY{o}{.}\PY{n}{linalg}\PY{o}{.}\PY{n}{inv}\PY{p}{(}\PY{n}{invSn}\PY{p}{)}
         \PY{n}{mn} \PY{o}{=} \PY{n}{beta}\PY{o}{*}\PY{n}{Sn}\PY{o}{.}\PY{n}{dot}\PY{p}{(}\PY{n}{PHIT}\PY{o}{.}\PY{n}{dot}\PY{p}{(}\PY{n}{HRTr}\PY{p}{)}\PY{p}{)}
         \PY{k}{print} \PY{n}{mn}\PY{o}{.}\PY{n}{shape}\PY{p}{,}\PY{n}{Sn}\PY{o}{.}\PY{n}{shape}
         
         \PY{c+c1}{\PYZsh{} Probemos la estimacion con el mn inicial}
         \PY{n}{yEst} \PY{o}{=} \PY{n}{PHI}\PY{o}{.}\PY{n}{dot}\PY{p}{(}\PY{n}{mn}\PY{p}{)}
         \PY{n}{lambdaIp}\PY{p}{,}\PY{n}{vecI} \PY{o}{=} \PY{n}{np}\PY{o}{.}\PY{n}{linalg}\PY{o}{.}\PY{n}{eig}\PY{p}{(}\PY{n}{PHIT}\PY{o}{.}\PY{n}{dot}\PY{p}{(}\PY{n}{PHI}\PY{p}{)}\PY{p}{)}
         \PY{n}{lambdaI} \PY{o}{=} \PY{n}{beta}\PY{o}{*}\PY{n}{lambdaIp}
         \PY{k}{for} \PY{n}{j} \PY{o+ow}{in} \PY{n+nb}{range}\PY{p}{(}\PY{l+m+mi}{0}\PY{p}{,}\PY{n}{iT}\PY{p}{)}\PY{p}{:}
             \PY{n}{gamma} \PY{o}{=} \PY{n}{np}\PY{o}{.}\PY{n}{sum}\PY{p}{(}\PY{n}{lambdaI}\PY{o}{/}\PY{p}{(}\PY{n}{alpha}\PY{o}{*}\PY{n}{np}\PY{o}{.}\PY{n}{ones}\PY{p}{(}\PY{n}{lambdaI}\PY{o}{.}\PY{n}{shape}\PY{p}{)}\PY{o}{+}\PY{n}{lambdaI}\PY{p}{)}\PY{p}{)}
             \PY{n}{alpha} \PY{o}{=} \PY{n}{gamma}\PY{o}{/}\PY{p}{(}\PY{p}{(}\PY{n}{mn}\PY{o}{.}\PY{n}{T}\PY{p}{)}\PY{o}{.}\PY{n}{dot}\PY{p}{(}\PY{n}{mn}\PY{p}{)}\PY{p}{)}
             \PY{n}{invbeta} \PY{o}{=} \PY{p}{(}\PY{l+m+mi}{1}\PY{o}{/}\PY{p}{(}\PY{n}{N}\PY{o}{\PYZhy{}}\PY{n}{gamma}\PY{p}{)}\PY{p}{)}\PY{o}{*}\PY{p}{(}\PY{p}{(}\PY{p}{(}\PY{n}{HRTr}\PY{o}{\PYZhy{}}\PY{n}{PHI}\PY{o}{.}\PY{n}{dot}\PY{p}{(}\PY{n}{mn}\PY{p}{)}\PY{p}{)}\PY{o}{.}\PY{n}{T}\PY{p}{)}\PY{o}{.}\PY{n}{dot}\PY{p}{(}\PY{n}{HRTr}\PY{o}{\PYZhy{}}\PY{n}{PHI}\PY{o}{.}\PY{n}{dot}\PY{p}{(}\PY{n}{mn}\PY{p}{)}\PY{p}{)}\PY{p}{)}
             \PY{n}{beta} \PY{o}{=} \PY{l+m+mi}{1}\PY{o}{/}\PY{n}{invbeta}
             \PY{n}{lambdaI} \PY{o}{=} \PY{n}{beta}\PY{o}{*}\PY{n}{lambdaIp}
             \PY{n}{invSn} \PY{o}{=} \PY{n}{alpha}\PY{o}{*}\PY{n}{np}\PY{o}{.}\PY{n}{eye}\PY{p}{(}\PY{n}{M}\PY{p}{)}\PY{o}{+} \PY{n}{beta}\PY{o}{*}\PY{n}{PHIT}\PY{o}{.}\PY{n}{dot}\PY{p}{(}\PY{n}{PHI}\PY{p}{)}
             \PY{n}{Sn} \PY{o}{=} \PY{n}{np}\PY{o}{.}\PY{n}{linalg}\PY{o}{.}\PY{n}{inv}\PY{p}{(}\PY{n}{invSn}\PY{p}{)}
             \PY{n}{mn} \PY{o}{=} \PY{n}{beta}\PY{o}{*}\PY{n}{Sn}\PY{o}{.}\PY{n}{dot}\PY{p}{(}\PY{n}{PHIT}\PY{o}{.}\PY{n}{dot}\PY{p}{(}\PY{n}{HRTr}\PY{p}{)}\PY{p}{)}
             
         \PY{c+c1}{\PYZsh{}print invSn }
         \PY{n}{yEstfin} \PY{o}{=} \PY{n}{PHI}\PY{o}{.}\PY{n}{dot}\PY{p}{(}\PY{n}{mn}\PY{p}{)}
         \PY{n}{plt}\PY{o}{.}\PY{n}{plot}\PY{p}{(}\PY{n}{xTr}\PY{p}{,}\PY{n}{HRTr}\PY{p}{,}\PY{l+s+s1}{\PYZsq{}}\PY{l+s+s1}{or}\PY{l+s+s1}{\PYZsq{}}\PY{p}{)}
         \PY{n}{plt}\PY{o}{.}\PY{n}{plot}\PY{p}{(}\PY{n}{xTr}\PY{p}{,}\PY{n}{yEstfin}\PY{p}{,}\PY{l+s+s1}{\PYZsq{}}\PY{l+s+s1}{ok}\PY{l+s+s1}{\PYZsq{}}\PY{p}{)}
\end{Verbatim}


    \begin{Verbatim}[commandchars=\\\{\}]
(4570L, 100L)
(100L, 1L) (100L, 100L)

    \end{Verbatim}

\begin{Verbatim}[commandchars=\\\{\}]
{\color{outcolor}Out[{\color{outcolor}26}]:} [<matplotlib.lines.Line2D at 0xbd84cf8>]
\end{Verbatim}
            
    \begin{center}
    \adjustimage{max size={0.9\linewidth}{0.9\paperheight}}{output_77_2.png}
    \end{center}
    { \hspace*{\fill} \\}
    
    En conclusion, este tipos de señales queda muy dificil describirla con
este tipo de modelos y funciones bases, por mas que se probo subir o
bajar la cantidad de funciones bases y cambiar el alpha y beta, no se
logra al menos tener un modelo aceptable.


    % Add a bibliography block to the postdoc
    
    
    
    \end{document}
